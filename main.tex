\documentclass[12pt]{ociamthesis}  % default square logo 
%\documentclass[12pt,beltcrest]{ociamthesis} % use old belt crest logo
%\documentclass[12pt,shieldcrest]{ociamthesis} % use older shield crest logo

%load any additional packages
\usepackage{amssymb}
\usepackage{listings}
\usepackage{color}
 
\definecolor{codegreen}{rgb}{0,0.6,0}
\definecolor{codegray}{rgb}{0.5,0.5,0.5}
\definecolor{codepurple}{rgb}{0.58,0,0.82}
\definecolor{backcolour}{rgb}{0.95,0.95,0.92}

 
\lstdefinestyle{mystyle}{
    backgroundcolor=\color{backcolour},   
    commentstyle=\color{codegreen},
    keywordstyle=\color{magenta},
    numberstyle=\tiny\color{codegray},
    stringstyle=\color{codepurple},
    basicstyle=\footnotesize,
    breakatwhitespace=false,         
    breaklines=true,                 
    captionpos=b,                    
    keepspaces=true,                 
    numbers=left,                    
    numbersep=5pt,                  
    showspaces=false,                
    showstringspaces=false,
    showtabs=false,                  
    tabsize=2,
    language=sh
}

\lstset{style=mystyle}

%input macros (i.e. write your own macros file called mymacros.tex 
%and uncomment the next line)
%\include{mymacros}

\title{Modul Praktikum \\[1ex]     %your thesis title,
        Kecerdasan Buatan}   %note \\[1ex] is a line break in the title

\author{Rolly Maulana Awangga}             %your name
\college{0410118609\\[5ex]
Applied Bachelor of Informatics Engineering}  %your college

%\renewcommand{\submittedtext}{change the default text here if needed}
\degree{Politeknik Pos Indonesia}     %the degree
\degreedate{Bandung 2019}         %the degree date

%end the preamble and start the document
\begin{document}

%this baselineskip gives sufficient line spacing for an examiner to easily
%markup the thesis with comments
\baselineskip=18pt plus1pt

%set the number of sectioning levels that get number and appear in the contents
\setcounter{secnumdepth}{3}
\setcounter{tocdepth}{3}


\maketitle                  % create a title page from the preamble info
\include{section/dedication}        % include a dedication.tex file
\include{section/acknowlegements}   % include an acknowledgements.tex file
\include{section/abstract}          % include the abstract

\begin{romanpages}          % start roman page numbering
\tableofcontents            % generate and include a table of contents
\listoffigures              % generate and include a list of figures
\end{romanpages}            % end roman page numbering

%now include the files of latex for each of the chapters etc
\include{section/chapter1}
\include{section/chapter2}
\chapter{Methods}

\section{The data}
PLease tell where is the data come from, a little brief of company can be put here.

\section{Method 1}
Definition, steps, algoritm or equation of method 1 and how to apply into your data
\section{Method 2}
Definition, steps, algoritm or equation of method 2 and how to apply into your data

\section{Yusniar Nur Syarif Sidiq/1164089}
\begin{enumerate}

\item Random Forest merupakan algoritma yang digunakan terhadapap klasifikasi data dalam jumlah yang besar. Klasifikasi pada random forest dilakukan dengan penggabungan dicision tree dengan melakuakn training terhadap sempel data yang dimiliki. Semakin banyak dicision tree maka data yang di dapat akan semakin akurat. Untuk gambar Random Forest dapat dilihat pada figure \ref{YNRF}

	\begin{figure}[ht]
	\centerline{\includegraphics[width=1\textwidth]{figures/YN/RF.PNG}}
	\caption{Random Forest.}
	\label{YNRF}
	\end{figure}

\item Pertama download dataset terlebih dahulu lalu buka dengan menggunakan software spyder guna melihat isi dari dataset tersebut. Data tersebut memiliki extensi file bernama .txt dan didalamnya terdapat class dari field. Misalnya saja pada data jenis burung memiliki file index dan angka, dimana index berisi angka yang memiliki makna berupa jenis burung atau bahkan nama burung sedangkan field memiliki isi nilai berupa 0 dan 1 yang dimana sifatnya boolean atau Ya dan Tidak. Hal ini dikarenakan komputer hanya dapat membaca bilangan biner maka dari itu field yang di isikan berupa angka. Artinya angka 0 berarti tidak dan angka 1 berarti Ya.

\item Cross Validation adalah sebuah teknik validasi model yang digunakan untuk menilai bagaimana hasil analisis statistik akan digeneralisasi ke kumpulan data independen. Cross validation digunakan dengan tujuan prediksi, dan bila kita ingin memperkirakan seberapa akurat model model prediksi yang dilakukan dalam sebuah praktek. Tujuan dari cross validation yaitu untuk mendefinisikan dataset guna menguju dalam fase pelatihan untuk membatasi masalah seperti overfitting dan underfitting serta mendapatkan wawasan tentang bagaimana model akan digeneralisasikan ke set data independen.

\item Dimana Score 44 \% diperoleh dari hasil pengelohan dataset jenis burung. Dimana akan dilakukan proses pembagian data testing dan data training lalu diproses dan menghasilkan score sebanyak 44 \% dimana menjelaskan bahwa score tersebut digunakan sebagai pembanding dalam tingkat keakuratannya. Pada dicision tree akan memperoleh data lebih kecil yaitu sebanyak 27 \% hal ini dikarenakan data yang diolah menggunakan dicision tree dibagi menjadi beberapa tree dan lalu disimpulkan untuk mendapatkan data yang akurat. Pada SVM akan memperoleh score sebanyak 29 \% hal ini dikarenakan data yang dimiliki masih bernilai netral sehingga tingkat keakuratannya masih belum jelas.

\item Untuk membaca confusion matriks dapat menggunakan source code sebagai berikut,
	\begin{verbatim}
		import numpy as np
		np.set_printoptions(precision=2)
		plt.figure(figsize=(60,60), dpi=300)
		plot_confusion_matrix(cm, classes=birds, normalize=True)
		plt.show()
	\end{verbatim}

Dimana numpy akan mengurus semua data yang berhubungan dengan matrix. Pada source code tersebut digunakan dalam melakukan read pada dataset burung dengan menggunakan metode confusion matrix. Dalam confusion matrix memiliki 4 istilah yaitu True Positive yang merupakan data posotif yang terditeksi benar, True Negatif yang merupakan data negatif akan tetapi terditeksi benar, False Positif merupakan data negatif namun terditeksi sebagai data positif, False Negatif merupakan data posotif namun terditeksi sebagai data negatif. Adapun contoh hasil read dataset menggunakan confusion matrix dapat dilihat pada figure \ref{YNCM}
	
	\begin{figure}[ht]
	\centerline{\includegraphics[width=1\textwidth]{figures/YN/YNCM.PNG}}
	\caption{Confusion Matrix.}
	\label{YNCM}
	\end{figure}

\item Voting merupakan proses pemilihan dari tree yang dimana akan dimunculkan hasilnya dan disimpulkan menjadi informasi yang pasti. Untuk kebih jelasnya saya akan memberikan sebuah contoh bagaimana voting beerja.
	
	\begin{figure}[ht]
	\centerline{\includegraphics[width=1\textwidth]{figures/YN/YNVoting.PNG}}
	\caption{Voting.}
	\label{YNV}
	\end{figure}

Dimana ditunjukkan pada figure \ref{YNV} terdapat 3 tree. Dalam tree tersebut akan dilakukan proses voting. Saya akan memberikan contoh kasus, dimana akan diadakan voting untuk menentukan sebuah mobil. Dalam tree akan diberikan sejumlah data misalnya saja data tersebut berupa gambar, yang dimana data tersebut akan dipilih dengan cara voting. Hasil voting akhir dari setiap tree menunjukkan mobil jazz, yang berarti kesimpulan dari data yang telah diberikan menyatakan gambar tersebut adalah mobil jazz. Bagaimana apabila terjadi perbedaan data misalnya saja pada tree 1 dan 2 menunyatakan mobil jazz sedangkan pada tree 3 menyatakan mobil yaris, maka kesimpulan yang di ambil adalah mobil jazz dikarenakan hasil voting terbanyak adalah mobil jazz.

\end{enumerate}


\section{Imron Sumadireja/1164076}
\subsection{Teori}
\begin{enumerate}
\item Random Forest Beserta Ilustrasinya \par
Random Forest adalah salah satu algoritma yang digunakan pada klasifikasi data dalam jumlah yang besar. Klasifikasi random forest ini dilakukan melalui penggabungan decision tree dengan melakukan training pada sampel data yang dimiliki atau biasa disebut dengan supervised learning. Semakin banyak menggunakan decision tree maka akan mempengaruhi akurasi yang didapatkan menjadi lebih baik. Setiap decision tree memiliki atribut yang berbeda, serta decision tree tersebut spesifik terhadap atributnya yang merupakan bagian kecil dari keseluruhan atribut pada data set. Contoh sederhananya bisa dilihat pada gambar berikut \ref{R1}
		\begin{figure}[ht]
		\centerline{\includegraphics[width=1\textwidth]{figures/im/R1.png}}
		\caption{Random Forest.}
		\label{R1}
		\end{figure}

\item Membaca Dataset, Makna Setiap File Serta Field Masing-Masing File\par
Pertama download terlebih dahulu datasetnya kemudian buka menggunakan spyder untuk mengetahui isi dari dataset tersebut. Untuk menjalankan code tersebut tinggal blok bagian yang akan di jalankan. Dataset tersebut di dalamnya terdapat class dari field atau data. Sebagai contoh pada data burung terdapat field index dan angka, untuk index biasanya berupa angka, angka tersebut memiliki makna sebagai pengganti nama atau jenis burung. Sedangkan field berisi nilai 0 dan 1 maknanya untuk memberikan penilaian ya atau tidak pada setiap suatu data namun pada kasus ini field di ganti dari ya atau tidak menjadi 0 dan 1 karena komputer kesulitan membaca ya atau tidak dan hanya bisa membaca dengan 0 dan 1 saja.

\item Cross Validation \par
Cross validation adalah metode statistik yang dapat digunakan untuk mengevaluasi kinerja model atau algoritma dengan data dipisahkan menjadi dua subset yaitu data testing dan data training. Selain itu cross validation digunakan untuk memperkirakan seberapa akurat sebuah model prediktif ketika dijalankan. Untuk melakukan proses cross validation ini dibutuhkan sebuah data. Cross validation mengambil data dari output yang telah di eksekusi oleh algoritma sebelumnya. Hasil tersebut akan dipisahkan menjadi dua subset berdasarkan ukuran dataset. Selanjutnya dataset tersebut akan di test secara bergantian hingga seluruh bagian terpenuhi.

\item Arti Score 44\% Pada Random Forest, 27\% Pada Decision Tree dan 29\% Dari SVM \par
\begin{enumerate}
\item Arti Score 44\% Pada Random Forest \par
Score tersebut merupakan hasil prediksi dari data yang telah dieksekusi sebelumnya dengan algoritma random forest, score tersebut menandakan bahwa akurasi yang didapatkan tidak terlalu baik karena data yang diujinya cukup banyak. Tetapi itu jauh lebih baik daripada menebak secara acak.
\item Arti Score 27\% Pada Decision Tree \par
Score tersebut merupakan hasil prediksi dari data yang dieksekusi sebelumnya dengan algoritma decision tree, selain itu pada decision tree menggunakan library sklearn sebagai acuan untuk melakukan prediksi. Untuk decision tree ini hasil yang didapatkan ialah 27\%. Hasil tersebut lebih buruk dibandingkan dengan menggunakan algoritma random forest.
\item Arti Score 29\% Pada SVM \par
Score tersebut merupakan hasil prediksi dari data yang dieksekusi sebelumnya dengan algoritma Support Vector Machine, score tersebut lebih baik daripada hasil yang di prediksi oleh decision tree namun score yang dimiliki oleh SVM tidak lebih baik dari hasil random forest.
\end{enumerate}

\item Cara Membaca Confusion Matriks Beserta Ilustrasinya \par
Cara untuk membaca confusion matriks yakni dengan cara memasukan parameter nilai yang tersedia pada datasets. Data tersebut akan  menghasilkan 0.5, 0.2 dan lain seterusnya sampai mendekati angka 1 atau akurasi yang sempurna. Pada confusion matriks terdapat 4 istilah sebagai representasi hasil proses klasifikasi, seperti gambar berikut \ref{R2}
		\begin{figure}[ht]
		\centerline{\includegraphics[width=1\textwidth]{figures/im/R2.png}}
		\caption{Confusion Matrix.}
		\label{R2}
		\end{figure}

\item Jelaskan Voting Pada Random Forest Beserta Ilustrasinya \par
Voting pada random forest berguna untuk mengambil nilai pada masing-masing tree yang akan digunakan untuk menentukan hasil final dengan akurasi yang lebih baik. Untuk ilustrasi sederhananya sebagai berikut \ref{R3}
		\begin{figure}[ht]
		\centerline{\includegraphics[width=1\textwidth]{figures/im/R3.png}}
		\caption{Voting.}
		\label{R3}
		\end{figure}
\end{enumerate}


\subsection{Praktik Program / Imron Sumadireja / 1164076}
\begin{enumerate}
\item Aplikasi Sederhana Menggunakan Pandas \par
	\begin{verbatim}
		import pandas as pd
		ron = {'Nama' : ['Arya','Razan','Bagja','MZ'], 'Umur' : [19,22,21,23], 
		'NPM' : [1145032,1145031,1145065,1145098]}
		df = pd.DataFrame(ron)
		print (df)
	\end{verbatim}
\begin{itemize}
\item Pada baris pertama menjelaskan bahwa code tersebut mengimport library pandas
\item Pada baris kedua itu merupakan sekumpulan data yang hasilnya akan membentuk seperti ndarray
\item Pada baris ketiga itu merupakan dataframe atau kerangka data yang berisi variable
\item Pada baris keempat itu untuk melihat hasil dari code tersebut.
\end{itemize}
Untuk hasilnya bisa dilihat seperti gambar berikut \ref{ron1}
		\begin{figure}[ht]
		\centerline{\includegraphics[width=1\textwidth]{figures/im/ron1.png}}
		\caption{Hasil dari pandas.}
		\label{ron1}
		\end{figure}

\item Aplikasi Sederhana Menggunakan Numpy \par
	\begin{verbatim}
		import numpy as np
		a = np.arange(24)
		a.ndim
		b = a.reshape(2,3,4)
		print (b)
	\end{verbatim}
\begin{itemize}
\item Pada baris pertama untuk import library numpy
\item Pada baris kedua untuk menampilkan angka sebanyak 24 dimulai dari 0
\item Pada baris ketiga merupakan nomor dari array dimensi
\item Pada baris keempat tersebut akan merubah tampilannya menjadi 2 bagian dengan 4 kolom dan 3 baris
\item Pada baris kelima untuk melihat hasil dari code tersebut.
\end{itemize}
Untuk hasilnya bisa dilihat seperti gambar berikut \ref{ron2}
		\begin{figure}[ht]
		\centerline{\includegraphics[width=1\textwidth]{figures/im/ron2.png}}
		\caption{Hasil dari numpy.}
		\label{ron2}
		\end{figure}

\item Aplikasi Sederhana Menggunakan Matplotlib \par
	\begin{verbatim}
		import pandas as pd
		import matplotlib.pyplot as plot
		data = pd.read_csv("F:/Imron/Kuliah/Semester 6/Artificial Intelegence/penjualan.csv")
		data.plot()
		data.show()
	\end{verbatim}
\begin{itemize}
\item Pada baris pertama untuk import library pandas
\item Pada baris kedua untuk import library matplotlib dengan inisialiasasi plt
\item Pada baris ketiga untuk membaca file .csv pada direktori tersebut
\item Pada baris keempat untuk membaca file .csv dengan library matplotlib
\item Pada baris kelima untuk menampilkan grafik dari hasil data pada .csv
\end{itemize}
Untuk hasilnya bisa dilihat seperti gambar berikut \ref{ron3}
		\begin{figure}[ht]
		\centerline{\includegraphics[width=1\textwidth]{figures/im/ron3.png}}
		\caption{Hasil dari matplotlib.}
		\label{ron3}
		\end{figure}

\item Menjalankan Program Klasifikasi Random Forest \par
Berikut ini adalah keluaran dari percobaan Klasifikasi Random Forest
\begin{itemize}
\item Hasil pada gambar \ref{rons1} tersebut menampilkan data dari file image attribute label. File tersebut berisi kategori, attribut pada setiap gambarnya dengan jumlah data sekitar 3 juta dan dibagi menjadi 3 kolom, pada code tersebut digunakan syntax error bad lines itu berfungsi untuk melewatkan data yang mengandung bad lines agar tidak terjadi errpr pada saat pembacaan file.
		\begin{figure}[ht]
		\centerline{\includegraphics[width=1\textwidth]{figures/im/rons1.png}}
		\caption{Klasifikasi Random Forest1.}
		\label{rons1}
		\end{figure}

\item Hasil pada gambar \ref{rons2} tersebut menampilkan 5 data teratas pada dataframe secara default.
		\begin{figure}[ht]
		\centerline{\includegraphics[width=1\textwidth]{figures/im/rons2.png}}
		\caption{Klasifikasi Random Forest2.}
		\label{rons2}
		\end{figure}

\item Hasil pada gambar \ref{rons3} menampilkan jumlah dari baris dan kolom pada file image attribute label atau dataframe
 		\begin{figure}[ht]
		\centerline{\includegraphics[width=1\textwidth]{figures/im/rons3.png}}
		\caption{Klasifikasi Random Forest3.}
		\label{rons3}
		\end{figure}

\item Hasil pada gambar \ref{rons4} merubah kolom menjadi baris, dan baris menjadi kolom dengan menggunakan fungsi dari pivot pada file sebelumnya
 		\begin{figure}[ht]
		\centerline{\includegraphics[width=1\textwidth]{figures/im/rons4.png}}
		\caption{Klasifikasi Random Forest4.}
		\label{rons4}
		\end{figure}

\item Hasil pada gambar \ref{rons5} tersebut menampilkan 5 data teratas secara default pada dataframe imgatt2
 		\begin{figure}[ht]
		\centerline{\includegraphics[width=1\textwidth]{figures/im/rons5.png}}
		\caption{Klasifikasi Random Forest5.}
		\label{rons5}
		\end{figure}

\item Hasil pada gambar \ref{rons6} menampilkan jumlah kolom dan baris pada dataframe imgatt2
 		\begin{figure}[ht]
		\centerline{\includegraphics[width=1\textwidth]{figures/im/rons6.png}}
		\caption{Klasifikasi Random Forest6.}
		\label{rons6}
		\end{figure}

\item Hasil pada gambar \ref{rons7} mengganti imgid menjadi index yang artinya unik untuk setiap datanya
 		\begin{figure}[ht]
		\centerline{\includegraphics[width=1\textwidth]{figures/im/rons7.png}}
		\caption{Klasifikasi Random Forest7.}
		\label{rons7}
		\end{figure}

\item Hasil pada gambar \ref{rons8} menampilkan 5 data teratas yang berisi apakah burung itu termasuk pada spesies yang mana. Kolom imgid ialah jenis burungnya dan kolom label itu spesies burungnya.
 		\begin{figure}[ht]
		\centerline{\includegraphics[width=1\textwidth]{figures/im/rons8.png}}
		\caption{Klasifikasi Random Forest8.}
		\label{rons8}
		\end{figure}

\item Hasil pada gambar \ref{rons9} menampilkan 11788 baris dan 1 kolom, dimana kolom tersebut merupakan spesies burungnya
 		\begin{figure}[ht]
		\centerline{\includegraphics[width=1\textwidth]{figures/im/rons9.png}}
		\caption{Klasifikasi Random Forest9.}
		\label{rons9}
		\end{figure}

\item Hasil pada gambar \ref{rons10} melakukan join antara imgatt2 dengan imglabels yang sebelumnya memiliki 312 kolom kini menjadi 313 kolom. Penggabungan ini termasuk ke dalam supervised learning karena kategorinya sudah tersedia.
 		\begin{figure}[ht]
		\centerline{\includegraphics[width=1\textwidth]{figures/im/rons10.png}}
		\caption{Klasifikasi Random Forest10.}
		\label{rons10}
		\end{figure}

\item Hasil pada gambar \ref{rons11} akan menghilangkan kolom pertama pada dataframe sebelumnya dan di rubah dengan kolom yang baru di join pada step sebelumnya
 		\begin{figure}[ht]
		\centerline{\includegraphics[width=1\textwidth]{figures/im/rons11.png}}
		\caption{Klasifikasi Random Forest11.}
		\label{rons11}
		\end{figure}

\item Hasil pada gambar \ref{rons12} menampilkan 5 data teratas dari dataframe att
 		\begin{figure}[ht]
		\centerline{\includegraphics[width=1\textwidth]{figures/im/rons12.png}}
		\caption{Klasifikasi Random Forest12.}
		\label{rons12}
		\end{figure}

\item Hasil pada gambar \ref{rons13} menampilkan 5 data teratas dari dataframe label
 		\begin{figure}[ht]
		\centerline{\includegraphics[width=1\textwidth]{figures/im/rons13.png}}
		\caption{Klasifikasi Random Forest13.}
		\label{rons13}
		\end{figure}

\item Hasil pada gambar \ref{rons14} membagi data menjadi 4 bagian, 8000 row pertama untuk data training atribut, 8000 row kedua untuk data training label, 8000 row ketiga untuk data testing atribut, dan 8000 row keempat untuk data testing label
 		\begin{figure}[ht]
		\centerline{\includegraphics[width=1\textwidth]{figures/im/rons14.png}}
		\caption{Klasifikasi Random Forest14.}
		\label{rons14}
		\end{figure}

\item Hasil pada gambar \ref{rons15} mengimport library sklearn ensemble untuk memanggil RandomForestClassifier, max features itu sebagai tanda ada berapa kolom untuk setiap tree nya pada kali ini setiap tree memiliki 50 kolom dengan estimasi 100 tree.
 		\begin{figure}[ht]
		\centerline{\includegraphics[width=1\textwidth]{figures/im/rons15.png}}
		\caption{Klasifikasi Random Forest15.}
		\label{rons15}
		\end{figure}

\item Hasil pada gambar \ref{rons16} hasil dari fit untuk membuat random forest dengan kategori yang sudah ditentukan dengan maksimum fitur sebanyak 50 kolom untuk setiap tree nya dengan estimasi 100 tree
 		\begin{figure}[ht]
		\centerline{\includegraphics[width=1\textwidth]{figures/im/rons16.png}}
		\caption{Klasifikasi Random Forest16.}
		\label{rons16}
		\end{figure}

\item Hasil pada gambar \ref{rons17} menampilkan hasil prediksi pada step sebelumnya pada random forest
 		\begin{figure}[ht]
		\centerline{\includegraphics[width=1\textwidth]{figures/im/rons17.png}}
		\caption{Klasifikasi Random Forest17.}
		\label{rons17}
		\end{figure}

\item Hasil pada gambar \ref{rons18} menampilkan hasil presentasi akurasi dengan menggunakan algoritma random forest
 		\begin{figure}[ht]
		\centerline{\includegraphics[width=1\textwidth]{figures/im/rons18.png}}
		\caption{Klasifikasi Random Forest18.}
		\label{rons18}
		\end{figure}
\end{itemize}

\item Menjalankan Program Confusion Matrix \par
Berikut ini adalah keluaran dari hasil percobaan Confusion Matrix
\begin{itemize}
\item Hasil pada gambar \ref{mat1} untuk memetakan data dari random forest ke dalam confusion matrix
 		\begin{figure}[ht]
		\centerline{\includegraphics[width=1\textwidth]{figures/im/mat1.png}}
		\caption{Confusion Matrix1.}
		\label{mat1}
		\end{figure}

\item Hasil pada gambar \ref{mat2} untuk menampilkan beberapa hasil dari data sebelumnya
 		\begin{figure}[ht]
		\centerline{\includegraphics[width=1\textwidth]{figures/im/mat2.png}}
		\caption{Confusion Matrix2.}
		\label{mat2}
		\end{figure}

\item Hasil pada gambar \ref{mat3} untuk merencanakan confusuin matrix dengan matplotlib sebelum di normalisasikan
 		\begin{figure}[ht]
		\centerline{\includegraphics[width=1\textwidth]{figures/im/mat3.png}}
		\caption{Confusion Matrix3.}
		\label{mat3}
		\end{figure}

\item Hasil pada gambar \ref{mat4} menampilkan file classes yang berisi nama-nama spesies burung 
 		\begin{figure}[ht]
		\centerline{\includegraphics[width=1\textwidth]{figures/im/mat4.png}}
		\caption{Confusion Matrix4.}
		\label{mat4}
		\end{figure}

\item Hasil pada gambar \ref{mat5} merupakan dari proses normalisasi yang pada step sebelumnya sudah direncanakan
 		\begin{figure}[ht]
		\centerline{\includegraphics[width=1\textwidth]{figures/im/mat5.png}}
		\caption{Confusion Matrix5.}
		\label{mat5}
		\end{figure}
\end{itemize}

\item Menjalankan Klasifikasi SVM dan Decision Tree \par
Berikut ini adalah hasil dari percobaan yang telah dilakukan
\begin{itemize}
\item Hasil pada gambar \ref{tree1} presentase prediksi yang dilakukan dengan menggunakan klasifikasi decision tree, dan hasilnya lebih buruk dibandingkan dengan random forest sebelumnya.
 		\begin{figure}[ht]
		\centerline{\includegraphics[width=1\textwidth]{figures/im/tree1.png}}
		\caption{Decision Tree1.}
		\label{tree1}
		\end{figure}

\item Hasil pada gambar \ref{svm1} presentase prediksi yang dilakukan dengan menggunakan klasifikasi SVM, dan hasilnya lebih baik di bandingkan dengan decision tree dan lebih buruk di bandingkan random forest
 		\begin{figure}[ht]
		\centerline{\includegraphics[width=1\textwidth]{figures/im/svm1.png}}
		\caption{SVM1.}
		\label{svm1}
		\end{figure}
\end{itemize}

\item Menjalankan Program Cross Validation \par
Berikut ini adalah hasil keluaran dari percobaan yang telah dilakukan
\begin{itemize}
\item Hasil pada gambar \ref{cross1} merupakan akurasi yang di dapatkan pada cross validation untuk random forest. Hasil tersebut masih warning dikarenakan mesinnya tidak kuat untuk melakukan prediksi secara menyeluruh
 		\begin{figure}[ht]
		\centerline{\includegraphics[width=1\textwidth]{figures/im/cross1.png}}
		\caption{Cross Validation1.}
		\label{cross1}
		\end{figure}

\item Hasil pada gambar \ref{cross2} merupakan akurasi yang didapatkan pada cross validation untuk decision tree. Hasil tersebut sama seperti step sebelumnya masih memiliki warning.
 		\begin{figure}[ht]
		\centerline{\includegraphics[width=1\textwidth]{figures/im/cross2.png}}
		\caption{Cross Validation2.}
		\label{cross2}
		\end{figure}

\item Hasil pada gambar \ref{cross3} merupakan akurasi yang didapatkan pada cross validation untuk SVM. Hasil tersebut masih sama seperti step sebelumnya masih memiliki tanda warning.
 		\begin{figure}[ht]
		\centerline{\includegraphics[width=1\textwidth]{figures/im/cross3.png}}
		\caption{Cross Validation3.}
		\label{cross3}
		\end{figure}
\end{itemize}

\item Menjalankan Program Pengamatan Komponen Informasi \par
Berikut ini adalah keluaran dari hasil percobaan yang telah saya lakukan
\begin{itemize}
\item Hasil pada gambar \ref{comp1} seharusnya mengeluarkan beberapa informasi mengenai banyaknya tree dan atribut lainnya. Namun yang terjadi pada percobaan saya hanya mengeluarkan akurasi saja. Dikarenakan masih terdapat warning
 		\begin{figure}[ht]
		\centerline{\includegraphics[width=1\textwidth]{figures/im/comp1.png}}
		\caption{Pengamatan Komponen Informasi1.}
		\label{comp1}
		\end{figure}

\item Hasil pada gambar \ref{comp2} ini merupakan hasil dari plotting komponen informasi, namun dikarenakan pada step sebelumnya terdapat warning jadi data-data yang terdapat pada gambar tersebut, terlihat sedikit acak
 		\begin{figure}[ht]
		\centerline{\includegraphics[width=1\textwidth]{figures/im/comp2.png}}
		\caption{Pengamatan Komponen Informasi2.}
		\label{comp2}
		\end{figure}
\end{itemize}
\end{enumerate}

\subsection{Penanganan Error / Imron Sumadireja / 1164076}
Dari percobaan yang telah saya lakukan, saya menemukan beberapa error, diantaranya sebagai berikut
\begin{itemize}
\item Screenshot error \ref{error1}
 		\begin{figure}[ht]
		\centerline{\includegraphics[width=1\textwidth]{figures/im/error1.png}}
		\caption{Error1.}
		\label{error1}
		\end{figure}

\item Code error \ref{error2}
 		\begin{figure}[ht]
		\centerline{\includegraphics[width=1\textwidth]{figures/im/error2.png}}
		\caption{Error2.}
		\label{error2}
		\end{figure}

\item Solusi Pemecahan Masalah \ref{sol1} saya coba rubah data training dan data testingnya menjadi 1000 dan hasilnya teratasi dari pada yang sebelumnya. Error tersebut dikarenakan memorinya tidak muat untuk melakukan running dengan data yang begitu banyak. Bahkan laptop saya pun sudah coba di restart dan hasilnya tetap sama.
 		\begin{figure}[ht]
		\centerline{\includegraphics[width=1\textwidth]{figures/im/sol1.png}}
		\caption{Solusi1.}
		\label{sol1}
		\end{figure}

\end{itemize}


\section{Andri Fajar Sunandhar/1164065}
\subsection{Teori}
\begin{enumerate}
\item Apa itu Random Forest Serta Gambar Ilustrasinya \par
Random Forest adalah suatu algoritma yang digunakan pada klasifikasi data dalam jumlah yang besar. Klasifikasi random forest dilakukan melalui penggabungan pohon  dengan melakukan training pada sampel data yang dimiliki. Penggunaan tree yang semakin banyak akan mempengaruhi akurasi yang akan didapatkan menjadi lebih baik. Penentuan klasifikasi dengan random forest diambil berdasarkan hasil voting dari pohon yang terbentuk. Pemenang dari pohon yang terbentuk ditentukan dengan vote terbanyak. Pembangunan pohon  pada random forest sampai dengan mencapai ukuran maksimum dari pohon data. Akan tetapi, pembangunan pohon Random Forest tidak dilakukan pemangkasan  yang merupakan sebuah metode untuk mengurangi kompleksitas ruang. Contoh Ilustrasi sederhana Gambar Random Forest. 
		\begin{figure}[ht]
		\centerline{\includegraphics[width=1\textwidth]{figures/AFS/AFS1.png}}
		\caption{Random Forest.}
		\label{AFS1}
		\end{figure}

\item Cara Membaca Dataset
	

		\begin{enumerate}
			\item Buka Anaconda Navigator.
			\item Jalankan Spyder
			\item Import libraries yang dibutuhkan
			\item Masukan kode berikut untuk membaca file Data.csv.
				\begin{figure}[ht]
				\centering
				\includegraphics[scale=0.8]{figures/AFS/2.png}
				\caption{Kode membaca file.csv}
				\label{contoh}
				\end{figure}
			\item Jalankan kode tersebut, maka di windiws console akan muncul pesan :
				\begin{figure}[ht]
				\centering
				\includegraphics[scale=0.9]{figures/AFS/3.png}
				\caption{ Window Console}
				\label{contoh}
				\end{figure}
			\item Klik variable explorer, maka akan terlihat dataset yang baru ter-import.
				\begin{figure}[ht]
				\centering
				\includegraphics[scale=0.6]{figures/AFS/4.png}
				\caption{Variable Explorer}
				\label{contoh}
				\end{figure}
			\item Kemudian double klik pada dataset cell, maka akan muncul pop-up windows seperti berikut: 
				\begin{figure}[ht]
				\centering
				\includegraphics[scale=0.7]{figures/AFS/5.png}
				\caption{ Dataset Cell}
				\label{contoh}
				\end{figure}
			\item Seperti yang terlihat pada gambar tersebut dataset ini memiliki Kolom Country, Age, dan Salary sebagai 		   				independent variable-nya dan kolom Purchased sebagai dependent variable-nya.
			
		\end{enumerate}
	

\item Cross Validation \par
Cross validation adalah metode statistik yang digunakan untuk memperkirakan keterampilan model pembelajaran mesin. Ini biasanya digunakan dalam pembelajaran mesin yang diterapkan untuk membandingkan dan memilih model untuk masalah pemodelan prediktif yang diberikan karena mudah dipahami, mudah diimplementasikan, dan menghasilkan estimasi keterampilan yang umumnya memiliki bias lebih rendah daripada metode lainnya.

\item Arti Score 44\% Pada Random Forest, 27\% Pada Decision Tree dan 29\% Dari SVM \par
\begin{enumerate}
\item Arti Score 44\% \par
Pada Random Forest, Score tersebut merupakan hasil dari akurasi.
\item Arti Score 27\% \par
Pada decission tree adalah presentasi hasil dari perhitungan dataset.
\item Arti Score 29\% Pada SVM \par
merupakan hasil pendekatan jaringan saraf. Jaringan saraf sendiri merupakan komponen jaringan utama dari sistem saraf. Sistem tersebut mengatur dan mengontrol fungsi tubuh dan aktivitas dan terdiri dari dua bagian:  (SSP) yang terdiri dari otak dan sumsum tulang belakang, dan percabangan saraf perifer dari sistem saraf tepi (SST) yang terdapat dalam pengolahan dataset terkait. 
\end{enumerate}

\begin {enumerate}
\item Confusion Matrix Dan Ilustrasinya
\begin{enumerate}
\item Perhitungan confusion matrix adalah sebagai berikut, akan saya beri contoh sederhana yaitu pengambilan keputusan untuk mendapatkan bantuan beasiswa. Saya menggunakan dua atribut, yaitu rekening listrik dan gaji. Ini adalah pohon keputusannya:
 
\begin{figure}[ht]
\centering
\includegraphics[scale=0.5]{figures/AFS/7.jpg}
\caption{Pohon Keputusan}
\label{contoh}
\end{figure}
\end{enumerate}


Kemudian data testingnya adalah

\begin{figure}[ht]
\centering
\includegraphics[scale=0.5]{figures/AFS/8.jpg}
\caption{Data Testing}
\label{contoh}
\end{figure}

Yang pertama kita lakukan yaitu mencari 4 nilai yaitu a,b,c, dan d:

 a= 5

 b= 1

 c= 1

 d= 3

Kemudian kita dapat mencari nilai Recall, Precision, accuracy dan Error Rate

 Recall =3/(1+3) = 0,75

 Precision = 3/(1+3) = 0,75

 Accuracy =(5+3)/(5+1+1+3) = 0,8

 Error Rate =(1+1)/(5+1+1+3) = 0,2

\end {enumerate}

\item Jelaskan Voting Pada Random Forest Beserta Ilustrasinya 
\par Voting merupakan metode yang paling umum digunakan dalam random forest. Ketika classifier membuat keputusan, Anda dapat memanfaatkan yang terbaik keputusan umum dan rata-rata yang didefinisikan ke dalam bentuk "voting".
\par Setelah pohon terbentuk,maka akan dilakukan voting pada setiap kelas dari data sampel. Kemudian, mengkombinasikan vote dari setiap kelas kemudian diambil vote yang paling banyak.Dengan menggunakan random forest pada klasifikasi data maka, akan menghasilkan vote yang paling baik. \ref{AFS6}
		\begin{figure}[ht]
		\centerline{\includegraphics[width=1\textwidth]{figures/AFS/6.png}}
		\caption{Voting.}
		\label{AFS6}
		\end{figure}
\end{enumerate}

\chapter{Methods}

\section{The data}
PLease tell where is the data come from, a little brief of company can be put here.

\section{Method 1}
Definition, steps, algoritm or equation of method 1 and how to apply into your data
\section{Method 2}
Definition, steps, algoritm or equation of method 2 and how to apply into your data

\section{Yusniar Nur Syarif Sidiq/1164089}
\begin{enumerate}

\item Random Forest merupakan algoritma yang digunakan terhadapap klasifikasi data dalam jumlah yang besar. Klasifikasi pada random forest dilakukan dengan penggabungan dicision tree dengan melakuakn training terhadap sempel data yang dimiliki. Semakin banyak dicision tree maka data yang di dapat akan semakin akurat. Untuk gambar Random Forest dapat dilihat pada figure \ref{YNRF}

	\begin{figure}[ht]
	\centerline{\includegraphics[width=1\textwidth]{figures/YN/RF.PNG}}
	\caption{Random Forest.}
	\label{YNRF}
	\end{figure}

\item Pertama download dataset terlebih dahulu lalu buka dengan menggunakan software spyder guna melihat isi dari dataset tersebut. Data tersebut memiliki extensi file bernama .txt dan didalamnya terdapat class dari field. Misalnya saja pada data jenis burung memiliki file index dan angka, dimana index berisi angka yang memiliki makna berupa jenis burung atau bahkan nama burung sedangkan field memiliki isi nilai berupa 0 dan 1 yang dimana sifatnya boolean atau Ya dan Tidak. Hal ini dikarenakan komputer hanya dapat membaca bilangan biner maka dari itu field yang di isikan berupa angka. Artinya angka 0 berarti tidak dan angka 1 berarti Ya.

\item Cross Validation adalah sebuah teknik validasi model yang digunakan untuk menilai bagaimana hasil analisis statistik akan digeneralisasi ke kumpulan data independen. Cross validation digunakan dengan tujuan prediksi, dan bila kita ingin memperkirakan seberapa akurat model model prediksi yang dilakukan dalam sebuah praktek. Tujuan dari cross validation yaitu untuk mendefinisikan dataset guna menguju dalam fase pelatihan untuk membatasi masalah seperti overfitting dan underfitting serta mendapatkan wawasan tentang bagaimana model akan digeneralisasikan ke set data independen.

\item Dimana Score 44 \% diperoleh dari hasil pengelohan dataset jenis burung. Dimana akan dilakukan proses pembagian data testing dan data training lalu diproses dan menghasilkan score sebanyak 44 \% dimana menjelaskan bahwa score tersebut digunakan sebagai pembanding dalam tingkat keakuratannya. Pada dicision tree akan memperoleh data lebih kecil yaitu sebanyak 27 \% hal ini dikarenakan data yang diolah menggunakan dicision tree dibagi menjadi beberapa tree dan lalu disimpulkan untuk mendapatkan data yang akurat. Pada SVM akan memperoleh score sebanyak 29 \% hal ini dikarenakan data yang dimiliki masih bernilai netral sehingga tingkat keakuratannya masih belum jelas.

\item Untuk membaca confusion matriks dapat menggunakan source code sebagai berikut,
	\begin{verbatim}
		import numpy as np
		np.set_printoptions(precision=2)
		plt.figure(figsize=(60,60), dpi=300)
		plot_confusion_matrix(cm, classes=birds, normalize=True)
		plt.show()
	\end{verbatim}

Dimana numpy akan mengurus semua data yang berhubungan dengan matrix. Pada source code tersebut digunakan dalam melakukan read pada dataset burung dengan menggunakan metode confusion matrix. Dalam confusion matrix memiliki 4 istilah yaitu True Positive yang merupakan data posotif yang terditeksi benar, True Negatif yang merupakan data negatif akan tetapi terditeksi benar, False Positif merupakan data negatif namun terditeksi sebagai data positif, False Negatif merupakan data posotif namun terditeksi sebagai data negatif. Adapun contoh hasil read dataset menggunakan confusion matrix dapat dilihat pada figure \ref{YNCM}
	
	\begin{figure}[ht]
	\centerline{\includegraphics[width=1\textwidth]{figures/YN/YNCM.PNG}}
	\caption{Confusion Matrix.}
	\label{YNCM}
	\end{figure}

\item Voting merupakan proses pemilihan dari tree yang dimana akan dimunculkan hasilnya dan disimpulkan menjadi informasi yang pasti. Untuk kebih jelasnya saya akan memberikan sebuah contoh bagaimana voting beerja.
	
	\begin{figure}[ht]
	\centerline{\includegraphics[width=1\textwidth]{figures/YN/YNVoting.PNG}}
	\caption{Voting.}
	\label{YNV}
	\end{figure}

Dimana ditunjukkan pada figure \ref{YNV} terdapat 3 tree. Dalam tree tersebut akan dilakukan proses voting. Saya akan memberikan contoh kasus, dimana akan diadakan voting untuk menentukan sebuah mobil. Dalam tree akan diberikan sejumlah data misalnya saja data tersebut berupa gambar, yang dimana data tersebut akan dipilih dengan cara voting. Hasil voting akhir dari setiap tree menunjukkan mobil jazz, yang berarti kesimpulan dari data yang telah diberikan menyatakan gambar tersebut adalah mobil jazz. Bagaimana apabila terjadi perbedaan data misalnya saja pada tree 1 dan 2 menunyatakan mobil jazz sedangkan pada tree 3 menyatakan mobil yaris, maka kesimpulan yang di ambil adalah mobil jazz dikarenakan hasil voting terbanyak adalah mobil jazz.

\end{enumerate}


\section{Imron Sumadireja/1164076}
\subsection{Teori}
\begin{enumerate}
\item Random Forest Beserta Ilustrasinya \par
Random Forest adalah salah satu algoritma yang digunakan pada klasifikasi data dalam jumlah yang besar. Klasifikasi random forest ini dilakukan melalui penggabungan decision tree dengan melakukan training pada sampel data yang dimiliki atau biasa disebut dengan supervised learning. Semakin banyak menggunakan decision tree maka akan mempengaruhi akurasi yang didapatkan menjadi lebih baik. Setiap decision tree memiliki atribut yang berbeda, serta decision tree tersebut spesifik terhadap atributnya yang merupakan bagian kecil dari keseluruhan atribut pada data set. Contoh sederhananya bisa dilihat pada gambar berikut \ref{R1}
		\begin{figure}[ht]
		\centerline{\includegraphics[width=1\textwidth]{figures/im/R1.png}}
		\caption{Random Forest.}
		\label{R1}
		\end{figure}

\item Membaca Dataset, Makna Setiap File Serta Field Masing-Masing File\par
Pertama download terlebih dahulu datasetnya kemudian buka menggunakan spyder untuk mengetahui isi dari dataset tersebut. Untuk menjalankan code tersebut tinggal blok bagian yang akan di jalankan. Dataset tersebut di dalamnya terdapat class dari field atau data. Sebagai contoh pada data burung terdapat field index dan angka, untuk index biasanya berupa angka, angka tersebut memiliki makna sebagai pengganti nama atau jenis burung. Sedangkan field berisi nilai 0 dan 1 maknanya untuk memberikan penilaian ya atau tidak pada setiap suatu data namun pada kasus ini field di ganti dari ya atau tidak menjadi 0 dan 1 karena komputer kesulitan membaca ya atau tidak dan hanya bisa membaca dengan 0 dan 1 saja.

\item Cross Validation \par
Cross validation adalah metode statistik yang dapat digunakan untuk mengevaluasi kinerja model atau algoritma dengan data dipisahkan menjadi dua subset yaitu data testing dan data training. Selain itu cross validation digunakan untuk memperkirakan seberapa akurat sebuah model prediktif ketika dijalankan. Untuk melakukan proses cross validation ini dibutuhkan sebuah data. Cross validation mengambil data dari output yang telah di eksekusi oleh algoritma sebelumnya. Hasil tersebut akan dipisahkan menjadi dua subset berdasarkan ukuran dataset. Selanjutnya dataset tersebut akan di test secara bergantian hingga seluruh bagian terpenuhi.

\item Arti Score 44\% Pada Random Forest, 27\% Pada Decision Tree dan 29\% Dari SVM \par
\begin{enumerate}
\item Arti Score 44\% Pada Random Forest \par
Score tersebut merupakan hasil prediksi dari data yang telah dieksekusi sebelumnya dengan algoritma random forest, score tersebut menandakan bahwa akurasi yang didapatkan tidak terlalu baik karena data yang diujinya cukup banyak. Tetapi itu jauh lebih baik daripada menebak secara acak.
\item Arti Score 27\% Pada Decision Tree \par
Score tersebut merupakan hasil prediksi dari data yang dieksekusi sebelumnya dengan algoritma decision tree, selain itu pada decision tree menggunakan library sklearn sebagai acuan untuk melakukan prediksi. Untuk decision tree ini hasil yang didapatkan ialah 27\%. Hasil tersebut lebih buruk dibandingkan dengan menggunakan algoritma random forest.
\item Arti Score 29\% Pada SVM \par
Score tersebut merupakan hasil prediksi dari data yang dieksekusi sebelumnya dengan algoritma Support Vector Machine, score tersebut lebih baik daripada hasil yang di prediksi oleh decision tree namun score yang dimiliki oleh SVM tidak lebih baik dari hasil random forest.
\end{enumerate}

\item Cara Membaca Confusion Matriks Beserta Ilustrasinya \par
Cara untuk membaca confusion matriks yakni dengan cara memasukan parameter nilai yang tersedia pada datasets. Data tersebut akan  menghasilkan 0.5, 0.2 dan lain seterusnya sampai mendekati angka 1 atau akurasi yang sempurna. Pada confusion matriks terdapat 4 istilah sebagai representasi hasil proses klasifikasi, seperti gambar berikut \ref{R2}
		\begin{figure}[ht]
		\centerline{\includegraphics[width=1\textwidth]{figures/im/R2.png}}
		\caption{Confusion Matrix.}
		\label{R2}
		\end{figure}

\item Jelaskan Voting Pada Random Forest Beserta Ilustrasinya \par
Voting pada random forest berguna untuk mengambil nilai pada masing-masing tree yang akan digunakan untuk menentukan hasil final dengan akurasi yang lebih baik. Untuk ilustrasi sederhananya sebagai berikut \ref{R3}
		\begin{figure}[ht]
		\centerline{\includegraphics[width=1\textwidth]{figures/im/R3.png}}
		\caption{Voting.}
		\label{R3}
		\end{figure}
\end{enumerate}




\section{Andri Fajar Sunandhar/1164065}
\subsection{Teori}
\begin{enumerate}
\item Apa itu Random Forest Serta Gambar Ilustrasinya \par
Random Forest adalah suatu algoritma yang digunakan pada klasifikasi data dalam jumlah yang besar. Klasifikasi random forest dilakukan melalui penggabungan pohon  dengan melakukan training pada sampel data yang dimiliki. Penggunaan tree yang semakin banyak akan mempengaruhi akurasi yang akan didapatkan menjadi lebih baik. Penentuan klasifikasi dengan random forest diambil berdasarkan hasil voting dari pohon yang terbentuk. Pemenang dari pohon yang terbentuk ditentukan dengan vote terbanyak. Pembangunan pohon  pada random forest sampai dengan mencapai ukuran maksimum dari pohon data. Akan tetapi, pembangunan pohon Random Forest tidak dilakukan pemangkasan  yang merupakan sebuah metode untuk mengurangi kompleksitas ruang. Contoh Ilustrasi sederhana Gambar Random Forest. 
		\begin{figure}[ht]
		\centerline{\includegraphics[width=1\textwidth]{figures/AFS/AFS1.png}}
		\caption{Random Forest.}
		\label{AFS1}
		\end{figure}

\item Cara Membaca Dataset
	

		\begin{enumerate}
			\item Buka Anaconda Navigator.
			\item Jalankan Spyder
			\item Import libraries yang dibutuhkan
			\item Masukan kode berikut untuk membaca file Data.csv.
				\begin{figure}[ht]
				\centering
				\includegraphics[scale=0.8]{figures/AFS/2.png}
				\caption{Kode membaca file.csv}
				\label{contoh}
				\end{figure}
			\item Jalankan kode tersebut, maka di windiws console akan muncul pesan :
				\begin{figure}[ht]
				\centering
				\includegraphics[scale=0.9]{figures/AFS/3.png}
				\caption{ Window Console}
				\label{contoh}
				\end{figure}
			\item Klik variable explorer, maka akan terlihat dataset yang baru ter-import.
				\begin{figure}[ht]
				\centering
				\includegraphics[scale=0.6]{figures/AFS/4.png}
				\caption{Variable Explorer}
				\label{contoh}
				\end{figure}
			\item Kemudian double klik pada dataset cell, maka akan muncul pop-up windows seperti berikut: 
				\begin{figure}[ht]
				\centering
				\includegraphics[scale=0.7]{figures/AFS/5.png}
				\caption{ Dataset Cell}
				\label{contoh}
				\end{figure}
			\item Seperti yang terlihat pada gambar tersebut dataset ini memiliki Kolom Country, Age, dan Salary sebagai 		   				independent variable-nya dan kolom Purchased sebagai dependent variable-nya.
			
		\end{enumerate}
	

\item Cross Validation \par
Cross validation adalah metode statistik yang digunakan untuk memperkirakan keterampilan model pembelajaran mesin. Ini biasanya digunakan dalam pembelajaran mesin yang diterapkan untuk membandingkan dan memilih model untuk masalah pemodelan prediktif yang diberikan karena mudah dipahami, mudah diimplementasikan, dan menghasilkan estimasi keterampilan yang umumnya memiliki bias lebih rendah daripada metode lainnya.

\item Arti Score 44\% Pada Random Forest, 27\% Pada Decision Tree dan 29\% Dari SVM \par
\begin{enumerate}
\item Arti Score 44\% \par
Pada Random Forest, Score tersebut merupakan hasil dari akurasi.
\item Arti Score 27\% \par
Pada decission tree adalah presentasi hasil dari perhitungan dataset.
\item Arti Score 29\% Pada SVM \par
merupakan hasil pendekatan jaringan saraf. Jaringan saraf sendiri merupakan komponen jaringan utama dari sistem saraf. Sistem tersebut mengatur dan mengontrol fungsi tubuh dan aktivitas dan terdiri dari dua bagian:  (SSP) yang terdiri dari otak dan sumsum tulang belakang, dan percabangan saraf perifer dari sistem saraf tepi (SST) yang terdapat dalam pengolahan dataset terkait. 
\end{enumerate}

\begin {enumerate}
\item Confusion Matrix Dan Ilustrasinya
\begin{enumerate}
\item Perhitungan confusion matrix adalah sebagai berikut, akan saya beri contoh sederhana yaitu pengambilan keputusan untuk mendapatkan bantuan beasiswa. Saya menggunakan dua atribut, yaitu rekening listrik dan gaji. Ini adalah pohon keputusannya:
 
\begin{figure}[ht]
\centering
\includegraphics[scale=0.5]{figures/AFS/7.jpg}
\caption{Pohon Keputusan}
\label{contoh}
\end{figure}
\end{enumerate}


Kemudian data testingnya adalah

\begin{figure}[ht]
\centering
\includegraphics[scale=0.5]{figures/AFS/8.jpg}
\caption{Data Testing}
\label{contoh}
\end{figure}

Yang pertama kita lakukan yaitu mencari 4 nilai yaitu a,b,c, dan d:

 a= 5

 b= 1

 c= 1

 d= 3

Kemudian kita dapat mencari nilai Recall, Precision, accuracy dan Error Rate

 Recall =3/(1+3) = 0,75

 Precision = 3/(1+3) = 0,75

 Accuracy =(5+3)/(5+1+1+3) = 0,8

 Error Rate =(1+1)/(5+1+1+3) = 0,2

\end {enumerate}

\item Jelaskan Voting Pada Random Forest Beserta Ilustrasinya 
\par Voting merupakan metode yang paling umum digunakan dalam random forest. Ketika classifier membuat keputusan, Anda dapat memanfaatkan yang terbaik keputusan umum dan rata-rata yang didefinisikan ke dalam bentuk "voting".
\par Setelah pohon terbentuk,maka akan dilakukan voting pada setiap kelas dari data sampel. Kemudian, mengkombinasikan vote dari setiap kelas kemudian diambil vote yang paling banyak.Dengan menggunakan random forest pada klasifikasi data maka, akan menghasilkan vote yang paling baik. \ref{AFS6}
		\begin{figure}[ht]
		\centerline{\includegraphics[width=1\textwidth]{figures/AFS/6.png}}
		\caption{Voting.}
		\label{AFS6}
		\end{figure}
\end{enumerate}



\subsection{Praktek Program}
\begin{enumerate}
\item Aplikasi Sederhana Menggunakan Pandas
	\begin{figure}[ht]
	\centering
	\includegraphics[scale=0.5]{figures/AFS/praktek1.jpg}
	\caption{Aplikasi Pandas}
	\label{contoh}
	\end{figure}
	\par Penjelasan kodingan :
		\begin{enumerate}
		\item Memanggil library.
		\item Membuat variable dengan data frame.
		\item Menampilkan hasil
		\end{enumerate}
	\par Sehingga menghasilkan :
	\begin{figure}[ht]
	\centering
	\includegraphics[scale=0.5]{figures/AFS/praktek2.jpg}
	\caption{Hasil Pandas}
	\label{contoh}
	\end{figure}
\item Aplikasi Sederhana Menggunakan Numpy
	\begin{figure}[ht]
	\centering
	\includegraphics[scale=0.5]{figures/AFS/praktek3.png}
	\caption{Aplikasi Numpy}
	\label{contoh}
	\end{figure}
	\par Penjelasan kodingan :
		\begin{enumerate}
		\item Memanggil library numpy
		\item Membuat variable dengan value eye dengan size10
		\item Menampilkan hasil value
		\end{enumerate}
	\par Sehingga menghasilkan :
	\begin{figure}[ht]
	\centering
	\includegraphics[scale=0.5]{figures/AFS/praktek4.png}
	\caption{Hasil Numpy}
	\label{contoh}
	\end{figure}
\item Aplikasi Sederhana Menggunakan Matplotlib
	\begin{figure}[ht]
	\centering
	\includegraphics[scale=0.5]{figures/AFS/praktek5.png}
	\caption{Aplikasi Matplotlib}
	\label{contoh}
	\end{figure}
	\par Penjelasan kodingan :
		\begin{enumerate}
		\item Memanggil library matplotlib.pyplot
		\item Membuat variable yang berisi 10,20,30,40,50,60,70
		\item Membuat garis koordinat
		\item Menampilkan hasil plt
		\end{enumerate}
	\par Sehingga menghasilkan :
	\begin{figure}[ht]
	\centering
	\includegraphics[scale=0.5]{figures/AFS/praktek6.png}
	\caption{Hasil Matplotlib}
	\label{contoh}
	\end{figure}
\par
\par
\item Program Klasifikasi Random Forest :
\begin{itemize}
\item Code Random Forest 1 :
\par
\begin{figure}[ht]
\centering
\includegraphics[scale=0.7]{figures/AFS/4a.jpg}
\caption{Gambar1}
\label{contoh}
\end{figure}
\par
\begin{itemize}
\item Penjelasan : Membaca dataset. Codingan di atas menghasilkan variabel baru yaitu imgatt. Terdapat 3 kolom dan 3677856 baris data.
\par 
\par
\end{itemize}
\item Code Random Forest 2 :
\par
\begin{figure}[ht]
\centering
\includegraphics[scale=0.7]{figures/AFS/4b.jpg}
\caption{Gambar2}
\label{contoh}
\end{figure}
\par
\begin{itemize}
\item Penjelasan : Codingan di atas berfungsi untuk melihat sebagian data awal dari dataset. Hasilnya terdapat pada gambar di atas setelah di eksekusi.
\par
\par
\end{itemize}
\item Code Random Forest 3 :
\par
\begin{figure}[ht]
\centering
\includegraphics[scale=0.7]{figures/AFS/4c.jpg}
\caption{Gambar3}
\label{contoh}
\end{figure}
\par
\begin{itemize}
\item Penjelasan : Codingan di atas merupakan tampilan untuk menampilkan hasil dari dataset yang telah di run atau di eksekusi. Dimana pada gambar di atas 3677856 merupakan baris dan 3 adalah kolom.
\par
\par
\end{itemize}
\item Code Random Forest 4 :
\par
\begin{figure}[ht]
\centering
\includegraphics[scale=0.7]{figures/AFS/4d.jpg}
\caption{Gambar 4}
\label{contoh}
\end{figure}
\par
\begin{itemize}
\item Penjelasan : Pada gambar di atas menmapilkan hasil dari variabel imgatt2. Dimana index nya 'imgid', kolom berisi 'attid' dan values atau nilainya berisi 'present'.
\par
\par
\end{itemize}
\item Code Random Forest 5 :
\par
\begin{figure}[ht]
\centering
\includegraphics[scale=0.7]{figures/AFS/4e.jpg}
\caption{Gambar 5}
\label{contoh}
\end{figure}
\par
\begin{itemize}
\item Penjelasan : Pada gambar di atas menmapilkan hasil dari variabel imgatt2.head. Dimana dataset nya ada 5 baris dan 312 kolom.
\par
\par
\end{itemize}
\item Code Random Forest 6 :
\par
\begin{figure}[ht]
\centering
\includegraphics[scale=0.7]{figures/AFS/4f.jpg}
\caption{Gambar 6}
\label{contoh}
\end{figure}
\par
\begin{itemize}
\item Penjelasan : Pada gambar di atas menampilkan jumlah dari baris dan kolom dari variabel imgatt2. Dimana 11788 adalah baris dan 312 adalah kolom.
\par
\par
\end{itemize}
\item Code Random Forest 7 :
\par
\begin{figure}[ht]
\centering
\includegraphics[scale=0.7]{figures/AFS/4g.jpeg}
\caption{Gambar 7}
\label{contoh}
\end{figure}
\par
\begin{itemize}
\item Penjelasan : Pada gambar di atas menunjukkan load dari  jawabannya yang berisi " apakah burung tersebut ( subjek pada dataset ) termasuk dalam spesies yang mana ?. Kolom yang digunakan adalah imgid dan label, kemudian melakukan pivot yang mana imgid menjadi index yang artinya unik sehubungan dengan dataset yang telah dieksekusi.
\par
\par
\end{itemize}
\item Code Random Forest 8 :
\par
\begin{figure}[ht]
\centering
\includegraphics[scale=0.2]{figures/AFS/4h.jpg}
\caption{Gambar 8}
\label{contoh}
\end{figure}
\par
\begin{itemize}
\item Penjelasan : Pada gambar di atas menunjukkan hasil dari variabel imglabels. Dimana menampilkan dataset dari imgid dan label. Dan dapat dilihat hasilnya dari gambar di atas.
\par
\par
\end{itemize}
\item Code Random Forest 9 :
\par
\begin{figure}[ht]
\centering
\includegraphics[scale=0.7]{figures/AFS/4i.jpg}
\caption{Gambar 9}
\label{contoh}
\end{figure}
\par
\begin{itemize}
\item Penjelasan : Pada gambar di atas menunjukkan jumlah baris dan kolom dari variabel imglabels. Dimana hasil dari kodingan tersebut dapat dilihat setelah di run. 
\par
\par
\end{itemize}
\item Code Random Forest 10 :
\par
\begin{figure}[ht]
\centering
\includegraphics[scale=0.7]{figures/AFS/4j.jpg}
\caption{Gambar 10}
\label{contoh}
\end{figure}
\par
\begin{itemize}
\item Penjelasan : Pada gambar diatas dikarenakan isinya sama, maka bisa melakukan join antara dua data yang diesekusi ( yaitu ada imgatt2 dan imglabels ), sehingga pada hasilnya akan didapatkan data ciri dan data jawaban atau labelnya sehingga bisa dikategorikan/dikelompokkan sebagai supervised learning. Jadi perintah untuk menggabungkan kedua data, kemudian dilakukan pemisahan antara data set untuk training dan test pada dataset yang dieksekusi.
\par
\par
\end{itemize}
\item Code Random Forest 11 :
\par
\begin{figure}[ht]
\centering
\includegraphics[scale=0.7]{figures/AFS/4k.jpg}
\caption{Gambar 11}
\label{contoh}
\end{figure}
\par
\begin{itemize}
\item Penjelasan :Pada gambar di atas menghasilkan pemisahan dan pemilihan tabel ( memisahkan dan memilih tabel ). 
\par
\par
\end{itemize}
\item Code Random Forest 12 :
\par
\begin{figure}[ht]
\centering
\includegraphics[scale=0.7]{figures/AFS/4l.jpg}
\caption{Gambar 12}
\label{contoh}
\end{figure}
\par
\begin{itemize}
\item Penjelasan : Pada gambar di atas menunjukkan hasil dari variabel dtatthead. Dimana data nya dapat dilihat pada gambar diatas. Dan dataset nya terdiri dari 5 baris dan 312 kolom.
\par
\par
\end{itemize}
\item Code Random Forest 13 :
\par
\begin{figure}[ht]
\centering
\includegraphics[scale=0.7]{figures/AFS/4m.jpg}
\caption{Gambar 13}
\label{contoh}
\end{figure}
\par
\begin{itemize}
\item Penjelasan : Pada gambar di atas menunjukkan hasil dari variabel dflabel.head. Dimana berisikan data dari imgid dan label. Dan hasilnya dapat dilihat pada gambar di atas.
\par
\par
\end{itemize}
\item Code Random Forest 14 :
\par
\begin{figure}[ht]
\centering
\includegraphics[scale=0.7]{figures/AFS/4n.jpg}
\caption{Gambar 14}
\label{contoh}
\end{figure}
\par
\begin{itemize}
\item Penjelasan : Pada gambar di atas merupakan pembagian dari data training dan dataset
\par
\par
\end{itemize}
\item Code Random Forest 15 :
\par
\begin{figure}[ht] 
\centering
\includegraphics[scale=0.7]{figures/AFS/4o.jpg}
\caption{Gambar 15}
\label{contoh}
\end{figure}
\par
\begin{itemize} 
\item Penjelasan : Pada gambar di atas merupakan pemanggilan kelas RandomForestClassifier. max features yang diartikan berapa banyak kolom pada setiap tree.
\par
\par
\end{itemize}
\item Code Random Forest 16 :
\par
\begin{figure}[ht]
\centering
\includegraphics[scale=0.7]{figures/AFS/4p.jpg}
\caption{Gambar 16}
\label{contoh}
\end{figure}
\par
\begin{itemize}
\item Penjelasan : Pada gambar di atas merupaka perintah untuk melakukan fit untuk membangun random forest yang sudah ditentukan dengan maksimum fitur sebanyak 50.
\par
\par
\end{itemize}
\item Code Random Forest 17 :
\par
\begin{figure}[ht]
\centering
\includegraphics[scale=0.7]{figures/AFS/4q.jpg}
\caption{Gambar 17}
\label{contoh}
\end{figure}
\par
\begin{itemize}
\item Penjelasan : Pada gambar di atas menunjukkan hasil dari cetakan variabel dftrainatt.head.
\par
\par
\end{itemize}
\item Code Random Forest 18 :
\par
\begin{figure}[ht]
\centering
\includegraphics[scale=0.7]{figures/AFS/4r.jpg}
\caption{Gambar 18}
\label{contoh}
\end{figure}
\par
\begin{itemize}
\item Penjelasan : Pada gambar di atas merupakan hasil dari variabel dftestatt da dftsetlabel. Dimana hasilnya dapat dilihat dari pada gambar di atas
\par
\par
\end{itemize}

\end{itemize}

\item Program Klasifikasi Confusion Matrix
	\begin{itemize}
		\item Setelah melakukan random forest kemudian dipetakan ke dalam confusion matrix.
			\begin{figure}[ht]
			\centering
			\includegraphics[scale=0.5]{figures/AFS/abc1.jpg}
			\caption{Memetakan ke confusion matrix}
			\label{contoh}
			\end{figure}
		\item Lalu melihat hasilnya.
			\begin{figure}[ht]
			\centering
			\includegraphics[scale=0.5]{figures/AFS/abc2.jpg}
			\caption{Melihat hasil}
			\label{contoh}
			\end{figure}
		\item Kemudian dilakukan perintah plot.
			\begin{figure}[ht]
			\centering
			\includegraphics[scale=0.5]{figures/AFS/abc3.jpg}
			\caption{Melakukan Plot}
			\label{contoh}
			\end{figure}
		\item Selanjutnya nama data akan di set agar plot sumbunya sesuai.
			\begin{figure}[ht]
			\centering
			\includegraphics[scale=0.5]{figures/AFS/abc4.jpg}
			\caption{Plotting nama data}
			\label{contoh}
			\end{figure}
		\item Setelah label berubah, maka dilakukan perintah plot.
		\begin{figure}[ht]
			\centering
			\includegraphics[scale=0.5]{figures/AFS/abc5.jpg}
			\caption{Melakukan perintah plot}
			\label{contoh}
			\end{figure}
	\end{itemize}
\par
\par
\item Program Klasifikasi SVM dan Decision Tree Beserta Penjelasan Keluarannya :
\begin{itemize}
\item Code SVM :
\par
\begin{figure}[ht]
\centering
\includegraphics[scale=0.7]{figures/AFS/t2.jpg}
\caption{SVM}
\label{contoh}
\end{figure}
\par
\begin{itemize}
\item Penjelasan : Pada gambar di atas cara untuk mencoba klasikasi dengan SVM dengan dataset yang sama.
\par 
\par
\end{itemize}
\item Code Decision Tree :
\par
\begin{figure}[ht]
\centering
\includegraphics[scale=0.7]{figures/AFS/t1.jpg}
\caption{Decission Tree}
\label{contoh}
\end{figure}
\par
\begin{itemize}
\item Penjelasan : Pada gambar di atas merupakan cara untuk mencoba klasikasi dengan decission tree dengan dataset yang sama.
\par
\par
\end{itemize}
\end{itemize}


\item Program Cross Validation
	\begin{itemize}
		\item Melakukan pengecekan cross validation untuk random forest.
			\begin{figure}[ht]
			\centering
			\includegraphics[scale=0.5]{figures/AFS/fajar1.jpg}
			\caption{Pengecekan cross validation random forest}
			\label{contoh}
			\end{figure}
		\item Melakukan pengecekan cross validation untuk decission tree.
			\begin{figure}[ht]
			\centering
			\includegraphics[scale=0.5]{figures/AFS/fajar2.jpg}
			\caption{Pengecekan cross validation decision tree}
			\label{contoh}
			\end{figure}
		\item Melakukan pengecekan cross validation untuk SVM.
			\begin{figure}[ht]
			\centering
			\includegraphics[scale=0.5]{figures/AFS/fajar3.jpg}
			\caption{Pengecekan cross validation SVM}
			\label{contoh}
			\end{figure}
	\end{itemize}
\item Program Pengamatan Komponen Informasi
	\begin{itemize}
		\item Melakukan pengamatan komponen informasi untuk menetahui berapa banyak tree yang dibuat, atribut yang dipakai, dan informasi lainnya.
			\begin{figure}[ht]
			\centering
			\includegraphics[scale=0.5]{figures/AFS/sunandhar1.jpg}
			\caption{Pengamatan Komponen}
			\label{contoh}
			\end{figure}
		\item Melakukan plot informasi agar bisa dibaca.
			\begin{figure}[ht]
			\centering
			\includegraphics[scale=0.5]{figures/AFS/sunandhar2.jpg}
			\caption{Plot informasi}
			\label{contoh}
			\end{figure}
	\end{itemize}
\end{enumerate}

\par
\par
\subsection{Penanganan Eror}
Penyelesaian Tugas Harian  ( Penanganan Error )
\begin{enumerate}
\item Menyelesaikan dan Membahas Penanganan Error :
\begin{itemize}
\item Skrinsut Error
\par
\begin{figure}[ht]

\centering
\includegraphics[scale=0.7]{figures/AFS/Error.jpg}
\caption{Error}
\label{contoh}
\end{figure}
\par
\begin{itemize}
\item Kode Error: file b'data/CUB 200 2011/attributes/image attributes labels.txt'
\par 
\item Solusi Pemecahan Error : Hapus Direktori data pada kode pastikan satu folder.
\par 
\par
\end{itemize}
\end{itemize}
\end{enumerate}


\chapter{Experiment and Result}
brief of experiment and result.
\section{Experiment}
Please tell how the experiment conducted from method.

\extrafloats{100}
\maxdeadcycles=200


\section{Result}
Please provide the result of experiment

\section{Imron Sumadireja/1164076}
\subsection{Teori}
\begin{enumerate}
\item Klasifikasi Teks dan Gambar Ilustrasi \par
Klasifikasi teks merupakan sebuah model yang digunakan untuk mengkategorikan teks ke dalam kelompok-kelompok yang lebih terorganisir. Jadi untuk setiap kalimat yang di masukan ke dalam mesin, mesin tersebut akan menjadikan setiap kata dari kalimat tersebut menjadi sebuah kolom. Untuk ilustrasinya bisa dilihat pada gambar berikut \ref{Teks1}
		\begin{figure}[ht]
		\centerline{\includegraphics[width=0.5\textwidth]{figures/im/teks1.png}}
		\caption{Klasifikasi teks.}
		\label{Teks1}
		\end{figure}

\item Mengapa klasifikasi bunga tidak bisa menggunakan machine learning \par
Karena machine learning tidak dapat menampilkan inputan sesuai dengan apa yang kita inputkan. Karena inputan tersebut serupa namun mesin memberikan output yang berbeda, biasanya output atau error ini disebut dengan istilah noise. Untuk contoh sederhananya misalkan kita inputkan salah satu label yang terdapat pada bunga, output yang dihasilkan oleh mesin tersebut ialah label yang lain. Itu dikarenakan bunga banyak jenis yang serupa namun tidak sama. Untuk ilustrasinya bisa dilihat pada gambar berikut \ref{Teks2}
		\begin{figure}[ht]
		\centerline{\includegraphics[width=0.5\textwidth]{figures/im/teks2.png}}
		\caption{Klasifikasi Bunga.}
		\label{Teks2}
		\end{figure}

\item Teknik pembelajaran mesin pada untuk kata-kata yang digunakan pada Youtube \par
Teknik yang digunakan pada youtube salah satunya ialah keywords. Dengan keywords tersebut mesin dapat memberikan video sesuai dengan keyword yang kita inputkan pada kolom pencarian. Teknik pembelajarannya tergantung user memberikan input teks seperti apa, karena pada youtube itu sendiri akan menyesuaikan dengan apa yang biasa kita inputkan dan akan memfilter video secara otomatis seuai dengan keyword yang biasa kita inputkan. Contoh ilustrasi sederhananya seperti berikut \ref{Teks3}
		\begin{figure}[ht]
		\centerline{\includegraphics[width=0.5\textwidth]{figures/im/teks3.png}}
		\caption{Klasifikasi teks Youtube.}
		\label{Teks3}
		\end{figure}

\item Vektorisasi data \par
Vektorisasi data merupakan pemecahan atau pembagian data berupa teks, sebagai contoh terdapat 5 paragraf, data teks tersebut di pecah menjadi kalimat-kalimat yang lebih sederhana, lalu di pecah lagi menjadi kata untuk setiap kalimatnya. 

\item Bag of Words \par
Representasi penyederhanaan sebuah kalimat atau perhitungan setiap kata pada suatu kalimat dengan presentase berapa kali muncul kata tersebut untuk setiap kalimatnya. Contoh ilustrasi sederhananya seperti berikut \ref{Teks5}
		\begin{figure}[ht]
		\centerline{\includegraphics[width=0.5\textwidth]{figures/im/teks5.png}}
		\caption{Bag of Words.}
		\label{Teks5}
		\end{figure}

\item Apa itu TF-IDF \par
TF-IDF merupakan metode untuk menghitung bobot setiap kata pada suatu kalimat yang paling sering digunakan. TF-IDF ini akan menghitung nilai Term Frequency dan Inverse Document Frequency pada setiap kata dalam setiap kalimat yang muncul dengan diimbangi dengan jumlah dokumen dalam korpus yang mengandung kata. Contoh ilustrasi sederhananya seperti gambar berikut \ref{Teks6}
		\begin{figure}[ht]
		\centerline{\includegraphics[width=1\textwidth]{figures/im/teks6.png}}
		\caption{TF-IDF.}
		\label{Teks6}
		\end{figure}
\end{enumerate}

\subsection{Praktikum / Imron Sumadireja / 1164076}
\begin{enumerate}
\item Buat aplikasi sederhana menggunakan pandas dengan format csv sebanyak 500 baris \par
\begin{verbatim}
import pandas as pd
d = pd.read_csv("F:/Imron/../praktikum/PraktikumChapter4/trial.csv")
\end{verbatim}
\begin{itemize}
\item Baris pertama menjelaskan import library pandas dengan inisialisasi pd untuk mengelola dataframe
\item Baris kedua untuk membaca file dengan format csv pada direktori tertentu
\item Hasilnya seperti gambar berikut \ref{df1}
\end{itemize}
		\begin{figure}[ht]
		\centerline{\includegraphics[width=0.5\textwidth]{figures/im/df1.png}}
		\caption{Data Frame.}
		\label{df1}
		\end{figure}

\item Memecah dataframe tersebut menjadi dua bagian yaitu 450 row pertama dan 50 row kedua \par
\begin{verbatim}
d_train=d[:450]
d_test=d[450:]
\end{verbatim}
\begin{itemize}
\item Baris pertama membagi data training menjadi 450
\item Baris kedua membagi data menjadi 50 atau sisa dari data yang tersedia
\item Hasilnya seperti gambar berikut \ref{df2}
\end{itemize}
		\begin{figure}[ht]
		\centerline{\includegraphics[width=0.5\textwidth]{figures/im/df2.png}}
		\caption{Data Frame.}
		\label{df2}
		\end{figure}

\item Praktik vektorisasi dan klasifikasi dari data katty perry dan tunjukan keluarannya
\begin{verbatim}
import pandas as pd
d = pd.read_csv("F:/Imron/../Chapter03/Youtube02-KatyPerry.csv")
\end{verbatim}
\begin{itemize}
\item Baris pertama untuk import library pandas berguna untuk mengelola dataframe
\item Baris kedua membaca file dengan format csv pada direktori tersebut
\item Untuk hasilnya seperti gambar berikut \ref{yt1}
\end{itemize}
		\begin{figure}[ht]
		\centerline{\includegraphics[width=0.5\textwidth]{figures/im/yt1.png}}
		\caption{Vektorisasi dan Klasifikasi.}
		\label{yt1}
		\end{figure}

\begin{verbatim}
spam=d.query('CLASS == 1')
nospam=d.query('CLASS == 0')
\end{verbatim}
\begin{itemize}
\item Coding tersebut untuk membagi menjadi 2 tabel spam dan nospam, untuk hasilnya seperti berikut \ref{yt2}
\end{itemize}
		\begin{figure}[ht]
		\centerline{\includegraphics[width=0.5\textwidth]{figures/im/yt2.png}}
		\caption{Spam dan NoSpam.}
		\label{yt2}
		\end{figure}

\begin{verbatim}
from sklearn.feature_extraction.text import CountVectorizer
vectorizer = CountVectorizer ()
\end{verbatim}
\begin{itemize}
\item Baris pertama untuk import countvectorizer berfungsi untuk memecah data tersebut menjadi sebuah kata yang lebih sederhana
\item Baris kedua untuk menjalankan fungsi tersebut, pada code ini tidak ada hasilnya dikarenakan spyder tidak mendukung hasil dari instasiasi.
\end{itemize}

\begin{verbatim}
dvec = vectorizer.fit_transform(d['CONTENT'])
dvec
\end{verbatim}
\begin{itemize}
\item Baris pertama untuk melakukan pemecahan data pada dataframe yang terdapat pada kolom konten
\item Untuk menampilkan hasil dari code sebelumnya, hasilnya seperti gambar berikut \ref{yt3}
\end{itemize}
		\begin{figure}[ht]
		\centerline{\includegraphics[width=0.5\textwidth]{figures/im/yt3.png}}
		\caption{Vektorisasi.}
		\label{yt3}
		\end{figure}

\begin{verbatim}
Daptarkata= vectorizer.get_feature_names()
\end{verbatim}
\begin{itemize}
\item Code tersebut untuk menampilkan setiap kata pada dataframe yang sudah di vektorisasi, untuk hasilnya seperti gambar berikut \ref{yt4}
\end{itemize}
		\begin{figure}[ht]
		\centerline{\includegraphics[width=0.5\textwidth]{figures/im/yt4.png}}
		\caption{Daftar Kata.}
		\label{yt4}
		\end{figure}

\begin{verbatim}
dshuf = d.sample(frac=1)
\end{verbatim}
\begin{itemize}
\item Code tersebut untuk mengacak dataframe tersebut agar tidak berurutan lagi sesuai dengan waktu, untuk hasilnya seperti gambar berikut \ref{yt5}
\end{itemize}
		\begin{figure}[ht]
		\centerline{\includegraphics[width=0.5\textwidth]{figures/im/yt5.png}}
		\caption{Vektorisasi.}
		\label{yt5}
		\end{figure}

\begin{verbatim}
d_train=dshuf[:300]
d_test=dshuf[300:]
\end{verbatim}
\begin{itemize}
\item Code tersebut untuk membagi data menjadi data training dan data testing, hasilnya seperti gambar berikut \ref{yt6}
\end{itemize}
		\begin{figure}[ht]
		\centerline{\includegraphics[width=0.5\textwidth]{figures/im/yt6.png}}
		\caption{Pembagian data training dan testing.}
		\label{yt6}
		\end{figure}

\begin{verbatim}
d_train_att = vectorizer.fit_transform(d_train['CONTENT'])
d_train_att
\end{verbatim}
\begin{itemize}
\item Baris pertama untuk memecah data pada tabel training di kolom content dari setiap kalimat menjadi kata
\item Untuk menampilkan hasil dari code tersebut
\item Hasilnya seperti gambar berikut \ref{yt7}
\end{itemize}
		\begin{figure}[ht]
		\centerline{\includegraphics[width=0.5\textwidth]{figures/im/yt7.png}}
		\caption{Pemecahan data pada table training.}
		\label{yt7}
		\end{figure}

\begin{verbatim}
d_test_att=vectorizer.transform(d_test['CONTENT'])
d_test_att
\end{verbatim}
\begin{itemize}
\item Baris pertama untuk memecah data pada table testing di kolom content dari setiap kalimatnya menjadi kata
\item Untuk menampilkan hasil dari code tersebut
\item Hasilnya seperti gambar berikut \ref{yt8}
\end{itemize}
		\begin{figure}[ht]
		\centerline{\includegraphics[width=0.5\textwidth]{figures/im/yt8.png}}
		\caption{Pemecahan data pada table testing.}
		\label{yt8}
		\end{figure}

\begin{verbatim}
d_train_label=d_train['CLASS']
d_test_label=d_test['CLASS']
\end{verbatim}
\begin{itemize}
\item Baris pertam untuk menampilkan kolom class dari data training
\item Baris kedua untuk menampilkan kolom class dari data testing
\item Hasilnya seperti gambar berikut \ref{yt9} dan \ref{yt10}
\end{itemize}
		\begin{figure}[ht]
		\centerline{\includegraphics[width=0.5\textwidth]{figures/im/yt9.png}}
		\caption{Menampilkan kolom class.}
		\label{yt9}
		\end{figure}

		\begin{figure}[ht]
		\centerline{\includegraphics[width=0.5\textwidth]{figures/im/yt10.png}}
		\caption{Menampilkan kolom class.}
		\label{yt10}
		\end{figure}

\item Klasifikasi dari data vektorisasi menggunakan klasifikasi SVM \par
\begin{verbatim}
from sklearn import svm
clfsvm = svm.SVR(gamma = 'auto')
clfsvm.fit(d_train_att, d_train_label)
clfsvm.score(d_test_att, d_test_label)
\end{verbatim}
\begin{itemize}
\item Baris pertama untuk import method svm dari library sklearn
\item Nilai gamma ini berfungsi untuk memperhitungkan perolehan presentase akhir agar lebih baik
\item Untuk menyatukan data training atribut dengan label untuk di latih menggunakan metode svm
\item Untuk memunculkan score dari hasil latihan tersebut, latihan ini sama seperti k-fold dengan menggunakan 2 data
\item Hasilnya seperti gambar berikut \ref{yt11}
\end{itemize}
		\begin{figure}[ht]
		\centerline{\includegraphics[width=0.5\textwidth]{figures/im/yt11.png}}
		\caption{SVM.}
		\label{yt11}
		\end{figure}

\item Klasifikasi dari data vektorisasi menggunakan klasifikasi Decision Tree \par
\begin{verbatim}
from sklearn import tree
clftree = tree.DecisionTreeClassifier()
clftree.fit(d_train_att, d_train_label)
clftree.score(d_test_att, d_test_label)
\end{verbatim}
\begin{itemize}
\item Baris pertama untuk import metode tree pada library sklearn
\item Baris kedua menjalankan decision tree classifier pada metode tree
\item Untuk menyatukan data training label dan atribut untuk dilatih menggunakan metode decision tree
\item Untuk memunculkan score dari hasil latihan dengan menggunakan metode decision tree
\item Hasilnya seperti berikut \ref{yt12}
\end{itemize}
		\begin{figure}[ht]
		\centerline{\includegraphics[width=0.5\textwidth]{figures/im/yt12.png}}
		\caption{Decision Tree.}
		\label{yt12}
		\end{figure}

\item Plotlah confusion matrix dari praktik modul ini menggunakan matplotlib \par
\begin{verbatim}
from sklearn.metrics import confusion_matrix
pred_labels=clf.predict(d_test_att)
cm=confusion_matrix(d_test_label,pred_labels)
\end{verbatim}
\begin{itemize}
\item Baris pertama untuk import metode confusion matrix pada library sklearn.metrics
\item Baris kedua menjelaskan bahwa data testing atribut akan di jalankan untuk di normalisasikan
\item Baris ketiga akan menjalankan data testing label dan data testing atribut untuk di normalisasikan pada step selanjutnya, dan code tersebut tidak mengeluarkan hasil apa-apa dikarenakan code tersebut hanya mempersiapkan data
\end{itemize}

\begin{verbatim}
import matplotlib.pyplot as plt

def plot_confusion_matrix(cm, classes,
                          normalize=False,
                          title='Confusion matrix',
                          cmap=plt.cm.Blues):
 
    if normalize:
        cm = cm.astype('float') / cm.sum(axis=1)[:, np.newaxis]
        print("Normalized confusion matrix")
    else:
        print('Confusion matrix, without normalization')

    print(cm)
\end{verbatim}
\begin{itemize}
\item Baris pertama untuk import library matplotlib dengan inisalisasi plt
\item Fungsi dari plot confusion matrix ini menormalisasikan atau menyiapkan data untuk ditampilkan berupa grafik, untuk hasilnya seperti berikut \ref{yt13} maksudnya data tersebut di prediksi ada 26 data yang termasuk bukan spam dan memang bukan spam namun ada 1 data yang di prediksi bukan spam tetapi hasilnya spam, begitupun sebaliknya terdapat 22 prediksi data spam dan itu memang termasuk dalam kategori spam tetapi ada 1 data yang di prediksi merupakan spam namun hasilnya bukan spam.
\end{itemize}
		\begin{figure}[ht]
		\centerline{\includegraphics[width=0.5\textwidth]{figures/im/yt13.png}}
		\caption{Confusion Matrix.}
		\label{yt13}
		\end{figure}

\item Menjalankan program cross validation
\begin{verbatim}
from sklearn.model_selection import cross_val_score
scores=cross_val_score(clf,d_train_att,d_train_label,cv=5)

skor_rata2=scores.mean()
skoresd=scores.std()
\end{verbatim}
\begin{itemize}
\item Baris pertama untuk import metode cross validation
\item Melatih data training atribut dan label dengan menggunakan cross validation atau hampir sama dengan k-vold
\item Untuk menampilkan hasil dari cross validation tersebur, seperti gambar berikut \ref{yt14}
\end{itemize}
		\begin{figure}[ht]
		\centerline{\includegraphics[width=0.5\textwidth]{figures/im/yt14.png}}
		\caption{Cross Validation.}
		\label{yt14}
		\end{figure}

\begin{verbatim}
from sklearn.model_selection import cross_val_score
scores = cross_val_score(clf, d_train_att, d_train_label, cv=5)
# show average score and +/- two standard deviations away (covering 95 \% of scores)
print("Accuracy: \%0.2f (+/-\%0.2f)" \% (scores.mean(), scores.std() * 2))
\end{verbatim}
\begin{itemize}
\item Code tersebut akan menentukan akurasi menggunakan cross validation dari hasil akurasi yang model decision tree, hasilnya seperti gambar berikut \ref{yt15}
\end{itemize}
		\begin{figure}[ht]
		\centerline{\includegraphics[width=0.5\textwidth]{figures/im/yt15.png}}
		\caption{Cross Validation.}
		\label{yt15}
		\end{figure}

\item Buatlah program pengamatan komponen informasi\par
\begin{verbatim}
max_features_opts = range(1, 10, 1)
n_estimators_opts = range(2, 40, 4)
rf_params = np.empty((len(max_features_opts)*len(n_estimators_opts),4), float)
i = 0
for max_features in max_features_opts:
    for n_estimators in n_estimators_opts:
        clf = RandomForestClassifier(max_features=max_features, n_estimators=n_estimators)
        scores = cross_val_score(clf, d_train_att, d_train_label, cv=5)
        rf_params[i,0] = max_features
        rf_params[i,1] = n_estimators
        rf_params[i,2] = scores.mean()
        rf_params[i,3] = scores.std() * 2
        i += 1
        print("Max features: \%d, num estimators: \%d, accuracy: \%0.2f (+/- \%0.2f)" \%               
              (max_features, n_estimators, scores.mean(), scores.std() * 2))
\end{verbatim}
\begin{itemize}
\item Code tersebut menjelaskan bahwa hasil dari data cross validation sebelumnya yang telah dilakukan pelatihan. Untuk range pada max features ini berfungsi untuk menampilkan parameter dari data sebelumnya dan untuk yang estimator berfungsi untuk mengatur akurasi pada gambar yang akan ditampilkan. Code tersebut tidak menampilkan keluaran
\end{itemize}

\begin{verbatim}
import matplotlib.pyplot as plt
from mpl_toolkits.mplot3d import Axes3D
from matplotlib import cm
fig = plt.figure()
fig.clf()
ax = fig.gca(projection='3d')
x = rf_params[:,0]
y = rf_params[:,1]
z = rf_params[:,2]
ax.scatter(x, y, z)
ax.set_zlim(0.9, 1)
ax.set_xlabel('Max features')
ax.set_ylabel('Num estimators')
ax.set_zlabel('Avg accuracy')
plt.show()
\end{verbatim}
\begin{itemize}
\item Code tersebut akan menampilkan grafik seperti berikut \ref{yt16} grafik tersebut dihasilkan dari data-data yang telah dilatih sebelumnya dengan menggunakan algoritma decision tree, cross validation, svm. Untuk codingan diatas pada baris kedua itu untuk import axes 3D yang berfungsi untuk menampilkan grafik 3D lalu untuk baris ke 11 itu diartikan rata-rata akurasi yang didapatkan pada pelatihan sebelumnya yang sudah dilakukan.
\end{itemize}
		\begin{figure}[ht]
		\centerline{\includegraphics[width=0.5\textwidth]{figures/im/yt16.png}}
		\caption{Komponen Informasi.}
		\label{yt16}
		\end{figure}
\end{enumerate}

\section{Penanganan Error}
\subsection{Imron Sumadireja /1164076}
Hasil praktikum yang telah saya lakukan terdapat beberapa kendala error diantaranya sebagai berikut: \par
\begin{enumerate}
\item Screenshot error \ref{ytError1}
		\begin{figure}[ht]
		\centerline{\includegraphics[width=0.5\textwidth]{figures/im/yterror1.png}}
		\caption{Screenshot error.}
		\label{ytError1}
		\end{figure}

\item Error tersebut bermasalah dengan codingan yang seharusnya SVR, itu dikarenakan pada versi spyder yang saya gunakan SVC ini tidak dapat dirunning, walaupun di running itu akan menampilkan warning. Untuk menghilangkan warning maka solusi yang saya dapatkan seperti berikut \ref{ytSolusi1}
		\begin{figure}[ht]
		\centerline{\includegraphics[width=0.5\textwidth]{figures/im/ytsolusi.png}}
		\caption{Solusi error.}
		\label{ytSolusi1}
		\end{figure}
\end{enumerate}

\section{Yusniar Nur Syarif Sidiq/1164089}
\subsection{Teori / Yusniar Nur Syarif Sidiq / 1164089}

\begin{enumerate}
\item Jelaskan apa itu klasifikasi teks, sertakan gambar ilustrasi buatan sendiri.
Klasifikasi teks merupakan sebuah proses pemberian tag atau kategori kedalam teks sesuai dengan isinya. Fungsi dari klasifikasi teks yaitu untuk melakukan klasifikasi atau pengelompokkan teks ke dalam sebuah label tertentu. Untuk contoh klasifikasi teks dapat dilihat pada figure \ref{YNC4-1}

	\begin{figure}[ht]
		\centering{\includegraphics[scale=0.5]{figures/YN/Chapter4/YNC4-1.png}}
		\caption{Contoh Klasifikasi Teks YN}
		\label{YNC4-1}
	\end{figure}

Dimana dalam figure \ref{YNC4-1} menjelaskan terdapat daftar kata berupa kata-kata yang cukup banyak muncul pada email anda dengan tujuan menawarkan sesuatu barang atau hal lainnya, maka dapat dikategorikan bahwa kata tersebut merupakan spam.

\item Jelaskan mengapa klarifikasi bunga tidak bisa menggunakan machine learning, sertakan ilustrasi sendiri.
Karena semua bunga belum tentu memiliki ciri-ciri yang sama, atau bisa dibilang adanya data noise dalam klasifikasi bunga yang dapat menyebabkan tidak bisa menggunakan Machine Learning. Akan saya beri contoh berdasarkan ilustrasi saya sendiri, terdapat bunga mawar bewarna merah yang memiliki jumlah 5 kelopak, lalu terdapat bunga selain mawar yang  berwarna merah dan memiliki jumlah kelopak yang sama yaitu 5 serta memiliki kategori yang cukup banyak. Lalu terdapat bunga yang tidak cukup jelas datanya baik warnanya maupun jumlah kelopaknya, data tersebut akan menyababkan data noise. Untuk ilustrasi gambar akan saya berikan 2 buah gambar bunga dengan warna yang sama akan tetapi jenis yang berbeda, dapat dilihat pada figure \ref{YNC4-2}

	\begin{figure}[ht]
		\centering{\includegraphics[scale=0.5]{figures/YN/Chapter4/YNC4-2.png}}
		\caption{Contoh Klasifikasi Bunga YN}
		\label{YNC4-2}
	\end{figure}

\item Jelaskan bagaimana teknik pembelajaran mesin pada teks pada kata-kata yang digunakan di youtube
Dapat menggunakan teknik bag-of-words pada klasifikasi berbasis text dan kata guna mengklasifikasikan sebuah komentar yang ada dalam internet sebagai kata spam atau bukan. Contohnya pada kolom komentar dapat di cek seberapa sering kata yang muncul dalam kalimat. Setiap kata bisa disebut sebagai baris dan kolomnya, hal ini merupakan dimana kategori kata spam atau tidak. Untuk ilustrasi gambarnya dapat dilihat pada figure \ref{YNC4-3}

	\begin{figure}[ht]
		\centering{\includegraphics[scale=0.5]{figures/YN/Chapter4/YNC4-3.png}}
		\caption{Teknik Pembelajaran Mesi Pada Teks Youtube YN}
		\label{YNC4-3}
	\end{figure}


\item Jelaskan apa yang dimaksud vektorisasi data.
Vektorisasi data merupakan pembagian dan pemecahan data lalu data tersebut akan dilakukan perhitungan. Vektorisasi dapat kita maksudkan setiap data yang mungkin kita petakan ke integer tertentu. Misalnya saja kita memiliki data array yang cukup besar maka setiap kata cocok dengan slot unik dalam array. Contoh kita memiliki banyak kata yang tersesun dengan beberapa paragraf, data tersebut nantinya akan kita pecah mejadi bebera kata dalam tiap kalimatnya.

\item Jelaskan apa itu bag of words dengan kata-kata yang sederhana dan ilustrasi sendiri
bag-of-words adalah representasi penyederhanaan yang digunakan dalam pemrosesan bahasa alami dan pengambilan informasi. Model bag-of-words sederhana untuk dipahami dan diterapkan dan telah melihat kesuksesan besar dalam masalah seperti pemodelan bahasa dan klasikasi dokumen. Ilustrasinya yaitu terdapat satu kalimat dan kalimat tersebut akan dipecah menjadi kata per kata, untuk lebih jelasnya dapat dilihat pada figure \ref{YNC4-4}

	\begin{figure}[ht]
		\centering{\includegraphics[scale=0.5]{figures/YN/Chapter4/YNC4-4.png}}
		\caption{Bag Of Words YN}
		\label{YNC4-4}
	\end{figure}

\item Jelaskan apa itu TF-IDF, ilustrasikan dengan gambar sendiri
TF-IDF dapat memberikan kita frekuensi kata dalam setiap dokumen sehingga dapat menggantikan data menjadi number. TD-IDF merupakan sebuah metode untuk dapat menghitung bobot setiap kata dalam kalimat yang sering digunakan. Untuk lebih jelasnya dapat dilihat dari figure \ref{TNC4-5}

	\begin{figure}[ht]
		\centering{\includegraphics[scale=0.5]{figures/YN/Chapter4/YNC4-5.png}}
		\caption{TF-IDF YN}
		\label{YNC4-5}
	\end{figure}


\end{enumerate}

\subsection{Praktek Program / Yusniar Nur Syarif Sidiq / 1164089}
\begin{enumerate}

\item Buat data dummy dengan format csv sebanyak 500 baris dan melakukan load ke dataframe pandas. Jelaskan arti setiap baris kode yang dibuat.

	\begin{verbatim}
		import pandas as pd
		yn = pd.read_csv("AttributeDataSet.csv")
	\end{verbatim}

Dimana pada baris pertama akan melakukan import pandas yang di rename menjadi pd. Pada baris kedua kita akan membuat variabel yn lalu di isikan dengan file csv nya. Untuk hasil output dalam spyder bisa dilihat pada figure \ref{YNC4-6}

	\begin{figure}[ht]
		\centering{\includegraphics[scale=0.5]{figures/YN/Chapter4/No1-2/YNC4-6.png}}
		\caption{Import DataFrame 500 Baris}
		\label{YNC4-6}
	\end{figure}

\item Dari DataFrame tersebut dipecah menjadi dua DataFrame yaitu 450 row pertama dan 50 row sisanya.

	\begin{verbatim}
		d_train=yn[:450]
		d_test=yn[450:]
	\end{verbatim}

Dimana pada baris pertama menjelaskan bahwa akan membagi data traning sebanyak 450. Pada baris kedua akan membuat data testing sebanyak 50 yang merupakan sisanya. Untuk hasil output dalam spyder bisa dilihat pada figure \ref{YNC4-7}

	\begin{figure}[ht]
		\centering{\includegraphics[scale=0.5]{figures/YN/Chapter4/No1-2/YNC4-7.png}}
		\caption{Membagi Data Menjadi 2 DataFrame}
		\label{YNC4-7}
	\end{figure}

\item Praktekan vektorisasi dan klarifikasi dari data (NPM mod 4, jika 0 maka katty perry, 1 LMFAO, 2 Eminem, 3 Shakira). Tunjukan keluarannya dari komputer sendiri dan artikan maksud setiap keluaran yang di dapatkan.

	\begin{verbatim}
		import pandas as pd
		yn=pd.read_csv("Youtube03-LMFAO.csv")
	\end{verbatim}

Dimana hasil dari output tersebut akan membaca data csv yaitu Youtube03-LMFAO. Untuk hasil output dalam spyder bisa dilihat pada figure \ref{YNC4-8}

	\begin{figure}[ht]
		\centering{\includegraphics[scale=0.5]{figures/YN/Chapter4/C4No3/YNC4-8.png}}
		\caption{Membaca File csv}
		\label{YNC4-8}
	\end{figure}	

Selanjutnya perhatikan source code dibawah ini,

	\begin{verbatim}
		spam=yn.query('CLASS == 1')
		nospam=yn.query('CLASS == 0')
	\end{verbatim}	

Dimana pada source code tersebut akan membagi data spam dan bukan spam, yang berarti data spam akan diberi tanda 1 dan data yang bukan spam akan diberi tanda 0. Untuk hasil output dalam spyder bisa dilihat pada figure \ref{YNC4-9}

	\begin{figure}[ht]
		\centering{\includegraphics[scale=0.5]{figures/YN/Chapter4/C4No3/YNC4-9.png}}
		\caption{Membagi Data Spam Dan Bukan Spam}
		\label{YNC4-9}
	\end{figure}	

Lanjut ke source code berikutnya,

	\begin{verbatim}
		from sklearn.feature_extraction.text import CountVectorizer
		vectorizer = CountVectorizer()
	\end{verbatim}	

Dimana source code diatas akan memanggil lib vektorisasi dan akan melakukan fungsi bag of word yaitu menghitung kata yang muncul perkalimat. Untuk hasil output dalam spyder bisa dilihat pada figure \ref{YNC4-10}

	\begin{figure}[ht]
		\centering{\includegraphics[scale=0.5]{figures/YN/Chapter4/C4No3/YNC4-10.png}}
		\caption{Fungsi Bag Of Word}
		\label{YNC4-10}
	\end{figure}

Selanjutnya perhatikan source code berikut,

	\begin{verbatim}
		dvec = vectorizer.fit_transform(yn['CONTENT'])
		dvec
	\end{verbatim}	

Dimana pada source code tersebut akan melakukan vektorisasi pada colom content dan akan ditampilkan hasilnya. Untuk output dalam spyder dapat dilihat pada figure \ref{YNC4-11}

	\begin{figure}[ht]
		\centering{\includegraphics[scale=0.5]{figures/YN/Chapter4/C4No3/YNC4-11.png}}
		\caption{Melakukan Vektorisasi Pada Colom Content}
		\label{YNC4-11}
	\end{figure}

Lanjut ke source code berikutnya,

	\begin{verbatim}
		dk=vectorizer.get_feature_names()
	\end{verbatim}

Dimana pada source code tersebut akan menampilkan data yang sudah di vektorisasi. Untuk output dalam spyder dapat dilihat pada figure \ref{YNC4-12}

	\begin{figure}[ht]
		\centering{\includegraphics[scale=0.5]{figures/YN/Chapter4/C4No3/YNC4-12.png}}
		\caption{Data Yang Sudah Di Vektorisasi}
		\label{YNC4-12}
	\end{figure}

Selanjutnya perhatikan source code dibawah,

	\begin{verbatim}
		yeay = yn.sample(frac=1)
	\end{verbatim}

Dimana pada source code tersebut akan melakukan pengacakan data pada database agar sempurna saat akan melakukan klasifikasi. Untuk output dalam spyder dapat dilihat pada figure \ref{YNC4-13}

	\begin{figure}[ht]
		\centering{\includegraphics[scale=0.5]{figures/YN/Chapter4/C4No3/YNC4-13.png}}
		\caption{Pengacakan Data}
		\label{YNC4-13}
	\end{figure}

Lanjut ke source code berikutnya,

	\begin{verbatim}
		yn_train=yeay[:300]
		yn_test=yeay[300:]
	\end{verbatim}

Dimana pada source code tersebut akan membagi menjadi 2 DataFrame yaitu 300 pada data training dan sisanya data testing. Untuk output dalam spyder dapat dilihat pada figure \ref{YNC4-14}

	\begin{figure}[ht]
		\centering{\includegraphics[scale=0.5]{figures/YN/Chapter4/C4No3/YNC4-14.png}}
		\caption{Membagi Data Yang Sudah Di Vektorisasi}
		\label{YNC4-14}
	\end{figure}

Selanjutnya perhatikan source code dibawah,

	
	\begin{verbatim}
		yn_train_att=vectorizer.fit_transform(yn_train['CONTENT'])
		yn_train_att
	\end{verbatim}

Dimana akan dilakukannya training pada data training dan akan di vektorisasi.  Untuk output dalam spyder dapat dilihat pada figure \ref{YNC4-15}

	\begin{figure}[ht]
		\centering{\includegraphics[scale=0.5]{figures/YN/Chapter4/C4No3/YNC4-15.png}}
		\caption{Vektorisasi Pada Data Training}
		\label{YNC4-15}
	\end{figure}

Selanjutnya perhatikan source code dibawah,

	
	\begin{verbatim}
		yn_test_att=vectorizer.transform(yn_test['CONTENT'])
		yn_test_att
	\end{verbatim}

Dimana akan dilakukannya testing pada data testing dan akan di vektorisasi.  Untuk output dalam spyder dapat dilihat pada figure \ref{YNC4-16}

	\begin{figure}[ht]
		\centering{\includegraphics[scale=0.5]{figures/YN/Chapter4/C4No3/YNC4-16.png}}
		\caption{Vektorisasi Pada Data Testing}
		\label{YNC4-16}
	\end{figure}

Lanjut ke source code berikut ini,

	\begin{verbatim}
		yn_train_label=yn_train['CLASS']
		yn_test_label=yn_test['CLASS']
	\end{verbatim}

Dimana pada source code tersebut akan mengambil data spam dan bukan spam dari data training dan testing. Untuk output dalam spyder dapat dilihat pada figure \ref{YNC4-17}

	\begin{figure}[ht]
		\centering{\includegraphics[scale=0.5]{figures/YN/Chapter4/C4No3/YNC4-17.png}}
		\caption{Mengambil Data Spam Dan Bukan Dari Traning Dan Testing Data}
		\label{YNC4-17}
	\end{figure}

\item Cobalah klasifikasikan dari data vektorisasi yang di tentukan di nomor sebelumnya dengan klasifikasi SVM.

	\begin{verbatim}
		from sklearn import svm
		clfsvm = svm.SVR(gamma='auto')
		clfsvm.fit(yn_train_att, yn_train_label)
		clfsvm.score(yn_test_att, yn_test_label)
	\end{verbatim}

Dimana pada source code tersebut akan melakukan import svm dari library sklearn dan akan melakukan klasifikasi dari data yang sudah di vektorisasikan. Hasil output dari source code tersebut berupa score prediksi dari svm. Untuk output dalam spyder dapat dilihat pada figure \ref{YNC4-18}

	\begin{figure}[ht]
		\centering{\includegraphics[scale=0.5]{figures/YN/Chapter4/C4No4/YNC4-18.png}}
		\caption{Score Prediksi Dari SVM}
		\label{YNC4-18}
	\end{figure}

\item Cobalah klasifikasikan dari data vektorisasi yang di tentukan nomor sebelumnya dengan klasifikasi Decision Tree.

	\begin{verbatim}
		from sklearn import tree
		clftree = tree.DecisionTreeClassifier()
		clftree.fit(yn_train_att, yn_train_label)
		clftree.score(yn_test_att, yn_test_label)
	\end{verbatim}

Dimana source code tersebut akan melakukan import modul tree dari library sklearn dan akan melakukan klasifikasi dari data yang sudah di vektorisasikan. Hasil dari output source code tersebut akan berupa score dari prediksi Decision Tree. Untuk output dalam spyder dapat dilihat pada figure \ref{YNC4-19}

	\begin{figure}[ht]
		\centering{\includegraphics[scale=0.5]{figures/YN/Chapter4/C4No5/YNC4-19.png}}
		\caption{Score Prediksi Dari Decision Tree}
		\label{YNC4-19}
	\end{figure}

\item Plotlah confusion matrix dari praktek modul ini menggunakan matplotlib.

	\begin{verbatim}
		import matplotlib.pyplot as plt

		def plot_confusion_matrix(cm, classes,
                         	 normalize=False,
                         	 title='Confusion matrix',
                         	 cmap=plt.cm.Blues):
  		  """
   		 This function prints and plots the confusion matrix.
   		 Normalization can be applied by setting `normalize=True`.
 		   """
   		 if normalize:
       			 cm = cm.astype('float') / cm.sum(axis=1)[:, np.newaxis]
     	   		 print("Normalized confusion matrix")
    		else:
        			print('Confusion matrix, without normalization')

   		 print(cm)
	\end{verbatim}

Dalam source code tersebut akan melakukan import library matplotlib dan di rename menjadi plt lalu memasukkan fungsi confusion matrix. Untuk output dalam spyder dapat dilihat pada figure \ref{YNC4-20}

	\begin{figure}[ht]
		\centering{\includegraphics[scale=0.5]{figures/YN/Chapter4/C4No6/YNC4-20.png}}
		\caption{Import Matplotlib}
		\label{YNC4-20}
	\end{figure}

Selanjutnya perhatikan source code berikut ini,

	\begin{verbatim}
		import numpy as np
		np.set_printoptions(precision=2)
		plot_confusion_matrix(cm, classes=yn, normalize=True)
		plt.show()
	\end{verbatim}

Dimana pada soirce code tersebut akan melakukan import pada library numpy yang dimana akan di rename menjadi np. Numpy disini berfungsi sebagai pengolah data matix arti dari numpy sendiri yaitu numarical matrix. Hasil dari source code tersebut akan memunculkan data matrix yang sudah di normalisasi. Untuk output dalam spyder dapat dilihat pada figure \ref{YNC4-21}

	\begin{figure}[ht]
		\centering{\includegraphics[scale=0.5]{figures/YN/Chapter4/C4No6/YNC4-21.png}}
		\caption{Matrix Normalisasi}
		\label{YNC4-21}
	\end{figure}

\item Jalankan program cross validation.

	\begin{verbatim}
		from sklearn.model_selection import cross_val_score
		scores = cross_val_score(clf, yn_train_att, yn_train_label, cv=5)
		print("Accurac")
	\end{verbatim}

Dimana pada source code diatas akan menampilkan score akurasi pada Random Forest. Untuk output dalam spyder dapat dilihat pada figure \ref{YNC4-22}

	\begin{figure}[ht]
		\centering{\includegraphics[scale=0.5]{figures/YN/Chapter4/C4No7/YNC4-22.png}}
		\caption{Score Prediksi Akurasi RF}
		\label{YNC4-22}
	\end{figure}

Selanjutnya perhatikan source code dibawah ini,

	\begin{verbatim}
		scorestree = cross_val_score(clftree, yn_train_att, yn_train_label, cv=5)
		print("Accuracy")
	\end{verbatim}

Dimana pada source code diatas akan menampilkan score akurasi pada Decision Tree. Untuk output dalam spyder dapat dilihat pada figure \ref{YNC4-23}

	\begin{figure}[ht]
		\centering{\includegraphics[scale=0.5]{figures/YN/Chapter4/C4No7/YNC4-23.png}}
		\caption{Score Prediksi Akurasi DT}
		\label{YNC4-23}
	\end{figure}

Selanjutnya perhatikan source code dibawah ini,

	\begin{verbatim}
		scoressvm = cross_val_score(clfsvm, yn_train_att, yn_train_label, cv=5)
		print("Accuracy")
	\end{verbatim}

Dimana pada source code diatas akan menampilkan score akurasi pada SVM. Untuk output dalam spyder dapat dilihat pada figure \ref{YNC4-24}

	\begin{figure}[ht]
		\centering{\includegraphics[scale=0.5]{figures/YN/Chapter4/C4No7/YNC4-24.png}}
		\caption{Score Prediksi Akurasi SVM}
		\label{YNC4-24}
	\end{figure}

\item Buatlah program pengamatan komponen informasi.

	\begin{verbatim}
		max_features_opts = range(1, 10, 1)
		n_estimators_opts = range(2, 40, 4)
		rf_params = np.empty((len(max_features_opts)*len(n_estimators_opts),4), float)
		i = 0
		for max_features in max_features_opts:
    		for n_estimators in n_estimators_opts:
        		clf = RandomForestClassifier(max_features=max_features, n_estimators=n_estimators)
       		 scores = cross_val_score(clf, yn_train_att, yn_train_label, cv=5)
      		 rf_params[i,0] = max_features
       		 rf_params[i,1] = n_estimators
        		rf_params[i,2] = scores.mean()
       		 rf_params[i,3] = scores.std() * 2
       		 i += 1
       		 print("Max features"
	\end{verbatim}

Output dari source code diatas akan melakukan pengulangan dari data-data sebelumnya yang sudah di vektorisasi.  Untuk output dalam spyder dapat dilihat pada figure \ref{YNC4-25}

	\begin{figure}[ht]
		\centering{\includegraphics[scale=0.5]{figures/YN/Chapter4/C4No8/YNC4-25.png}}
		\caption{Pengulangan Data Vektorisasi}
		\label{YNC4-25}
	\end{figure}

Selanjutnya perhatikan source code berikut,

	\begin{verbatim}
		import matplotlib.pyplot as plt
		from mpl_toolkits.mplot3d import Axes3D
		from matplotlib import cm
		fig = plt.figure()
		fig.clf()
		ax = fig.gca(projection='3d')
		x = rf_params[:,0]
		y = rf_params[:,1]
		z = rf_params[:,2]
		ax.scatter(x, y, z)
		ax.set_zlim(0.8, 1)
		ax.set_xlabel('Max features')
		ax.set_ylabel('Num estimators')
		ax.set_zlabel('Avg accuracy')
		plt.show()
	\end{verbatim}

Hasil output dari source code tersebut adalh menampilkan data pengulangan yang sudah di vektorisasi dalam bentuk grafik. Untuk output dalam spyder dapat dilihat pada figure \ref{YNC4-26}

	\begin{figure}[ht]
		\centering{\includegraphics[scale=0.5]{figures/YN/Chapter4/C4No8/YNC4-26.png}}
		\caption{Grafik Data Pengulangan}
		\label{YNC4-26}
	\end{figure}

\end{enumerate}

\subsection{Penangan Error / Yusniar Nur Syarif Sidiq / 1164089}
\begin{enumerate}

\item Screenshoot Error
Dimana error yang saya dapat ditunjukan pada figure \ref{YNC4-27}

	\begin{figure}[ht]
		\centering{\includegraphics[scale=0.5]{figures/YN/Chapter4/C4PE/YNC4-27.png}}
		\caption{Syntax Error}
		\label{YNC4-27}
	\end{figure}

\item Tuliskan kode error dan jenisnya

	\begin{verbatim}
		from sklearn import svm
		clfsvm = svm.SVR()
		clfsvm.fit(yn_train_att, yn_train_label)
		clfsvm.score(yn_test_att, yn_test_label)
	\end{verbatim}

Dimana error tersebut merupakan syntax error yaitu terjadi kekurangan source code atau salah penulisannya.

\item Solusi pemecahan masalah error
Solusi dari error tersebut adalah dengan melengkapi source codenya. Untuk source code lengkapnya dapat dilihat dibawah ini

	\begin{verbatim}
		from sklearn import svm
		clfsvm = svm.SVR(gamma='auto')
		clfsvm.fit(yn_train_att, yn_train_label)
		clfsvm.score(yn_test_att, yn_test_label)
	\end{verbatim}

\end{enumerate}
\chapter{Conclusion}
brief of conclusion

\section{Conclusion of Problems}
Tell about solving the problem

\section{Conclusion of Method}
Tell about solving using method

\section{Conclusion of Experiment}
Tell about solving in the experiment

\section{Conclusion of Result}
tell about result for purpose of this research.

\section{Yusniar Nur Syarif Sidiq/1164089}
\subsection{Pemahaman Teori / Yusniar Nur Syarif Sidiq / 1164089}

\begin{enumerate}

\item Jelaskan kenapa kata-kata harus dilakukan vektorisasi. Dilengkapi dengan ilustrasi atau gambar.
	\subitem Dikarenakan mesin hanya mampu membaca data dengan bentuk angka maka dari itu diperlukan vektorisasi kata atau bisa disebut dengan mengubah kata menjadi bentuk vektor agar mesin seolah-olah paham apa yang kita maksudkan. Kal inii saya memberikan ilustrasi sederhana, dimana ada sebuah data mengenai kucing, tikus, dan pulpen. Bagaimana cara mesin untuk membaca data tersebut ?, yaitu dengan cara dilakukannya vektorisasi kata. Fungsi dari vektorisasi itu sendiri ialah sebagai indentitas, misalnya di dalam dokumen kucing terdapat kata-kata yang sudah di vektorisasi dan hasilnya adalah 0.012,0.024,.....,0.300, pada dokumen tikus menghasilkan 0.015,0.026,....,0.0276, dan pada pulpen menghasilkan -0.191,...,-0.045. Data vektor tersebut merupakan identitas dari data kucing, tikus dan pulpen.Apabila nilai vektor cenderung sama, maka data tersebut memiliki similarity atau data yang memiliki konten kata yang sama akan tetapi berbeda dengan pulpen yang memiliki data vektorisasi minus, sehingga dapat disimpulkan pulpen memiliki konten kata yang berbeda. Dengan adanya vektorisasi tersebut maka mesin seolah-olah mengerti bahwa kucing dan tikus memiliki konten yang sama yaitu merupakan objek dari binatang sedangkan pulpen merupakan objek benda mati. Mengenai contoh gambarnya dapat dilihat pada figure \ref{YNC5-1}

	\begin{figure}[!htbp]
		\centering{\includegraphics[scale=0.5]{figures/YN/Chapter5/Teori/YNC5-1.png}}
		\caption{Kata-Kata Hasil Vektorisasi}
		\label{YNC5-1}
	\end{figure}

\item Jelaskan mengapa dimensi dari vektor dataset google bisa sampe 300. Dilengkapi dengan ilustrasi atau gambar.
	\subitem Dikarenakan pada setiap dataset didalamnya memiliki identitas masing-masing yang menyebabkan jumlah vektor dataset google bisa sampe 300. Saya akan memberikan sebuah ilustrasi yang saya dapat yaitu bagaimana pembagian dataset pada google. Didalam dataset google tersebut memiliki beberapa objek yaitu spatula, cat, dan dog. Dimana ketiga dataset tersebut akan dilakukan proses perbandingan dataset sehingga diproleh hasil antara cat dan dog yaitu 76\% dikarekan pada dataset cat dan dog memiliki kesamaan data sedangkan untuk hasil perbandingan antara cat dan spatula yaitu 12\%. Mengapa lebih sedikit, dikarenakan data yang dimiliki oleh cat dan spatula tidak memiliki kesamaan data. Hal ini dapat membuktikan setelah dilakukan vektorisasi mesin jadi dapat membedakan mana dataset yang memiliki kesamaan dan mana yang bukan. Mengenai pengertian dan ikustrasi tersebut dapat kita liat kedalam bentuk figure \ref{YNC5-5}

	\begin{figure}[!htbp]
		\centering{\includegraphics[scale=0.5]{figures/YN/Chapter5/Teori/YNC5-5.png}}
		\caption{Dataset Google}
		\label{YNC5-5}
	\end{figure}

\item Jelaskan konsep vektorisasi untuk kata. Dilengkapi dengan ilustrasi atau gambar.
	\subitem Konsep pada vektorisasi kata yaitu dimana kata tersebut merupakan sebuah hasil pengolahan kata dari sebuah kalimat-kalimat yang telah kita olah. Misalnya saat kita membuka sosial media dan terdapat banyak komentar didalamnya. Ada sebuah kalimat yang mengatakan \" Follow instagram aku yah teman-teman \" , dimana dalam kalimat tersebut memiliki kata kunci instagram dan kata tersebut akan dijadikan data training untuk mesin. Diamana nantinya mesin akan menampilkan kata-kata yang ada kaitannya dengan kata kunci tersebut. Figure \ref{YNC5-3} merupakan sebuah ilustrasi pada vektorisasi kata.

	\begin{figure}[!htbp]
		\centering{\includegraphics[scale=0.5]{figures/YN/Chapter5/Teori/YNC5-2.png}}
		\caption{Vektorisasi Kata}
		\label{YNC5-2}
	\end{figure}

\item Jelaskan konsep vektorisasi untuk dokumen. Dilengkapi dengan ilustrasi atau gambar.
	\subitem Sebenarnya konsep pada vektorisasi dokumen dan vektorisasi kata itu sama saja, hal yang membedakannya itu hanya proses awalnya. Dimana pada vektorisasi kata akan membaca kalimat per kalimat namun pada vektorisasi dokumen akan membaca keseluruhan kalimat yang terdapat pada sebuah dokumen yang nantinya kalimat-kalimat tersebut akan dipecah menjadia kata per kata. Pada figure \ref{YNC5-3} merupakan contoh ilustrasi sederhana.

	\begin{figure}[!htbp]
		\centering{\includegraphics[scale=0.5]{figures/YN/Chapter5/Teori/YNC5-3.png}}
		\caption{Vektorisasi Document}
		\label{YNC5-3}
	\end{figure}

\item Jelaskan apa mean dan standar deviasi. Dilengkapi dengan ilustrasi.
	\subitem Mean merupakan nilai rata-rata dari suatu data. Mean dapat dicari dengan cara membagi jumlah data dengan banyak data sehingga diperoleh lah nilai rata-rata dari suatu data. Sedangkan standar deviasi merupakan sebuah teknik statistik yang digunakan dalam menjelaskan homogenitas kelompok. Contoh sederhananya dimana kita menjumpai dataset yang memiliki 10 data yang berbeda dan 5 atribut yang berbeda. Untuk memperolah nilai mean kita harus menghitung keseluruhan data tersebut lalu hasilnya akan dibagi dengan jumlah datanya. Setelah kita mendapatkan nilai mean maka kita bisa menghitung standar deviasi untuk memperoleh nilai statisknya.

\item Jelaskan apa itu skip-gram. Dilengkapi dengan ilustrasi atau gambar.
	\subitem Skip-gram merupakan sebuah teknik yang digunakan pada area speech processing yang dimana n-gram dibentuk lalu ditambah dengan tindakan skip. Misalkan ada sebuah kalimat yaitu \" I hit the tennis ball \", kita akan membuatnya menjadi skip gram 3 kata maka akan menjadi,
		\begin{itemize}
			\item I hit the
			\item Hit the tennis
			\item The tennis ball
		\end{itemize}
Figure \ref{YNC5-4} merupakan ilustrasi sederhana.
		\begin{figure}[!htbp]
		\centering{\includegraphics[scale=0.5]{figures/YN/Chapter5/Teori/YNC5-4.png}}
		\caption{Kata-Kata Hasil Vektorisasi}
		\label{YNC5-4}
	\end{figure}

\end{enumerate}

\subsection{Praktek Pemrograman / Yusniar Nur Syarif Sidiq / 1164089}
\begin{enumerate}
\item Cobalah dataset google, dan jelaskan vektor dari kata love, faith, fall, sick, clear, shine, bag, car, wash, motor, cycle dan cobalah untuk melakukan perbandingan similirati dari masing-masing kata tersebut. Jelaskan arti dari outputan similaritas.
	\subitem Dimana output dari source code tersebut merupakan data vektor untuk kata love. Perhatikan figure \ref{YNC5-6}.

		\begin{verbatim}
			gmodel['love']
		\end{verbatim}

		\begin{figure}[!htbp]
			\centering{\includegraphics[scale=0.5]{figures/YN/Chapter5/Praktek/YNC5-6.png}}
			\caption{Vektorisasi Kata}
			\label{YNC5-6}
		\end{figure}

Output source code dibawah akan memunculkan data vektor untuk kata faith. Perhatikan figure \ref{YNC5-7}

		\begin{verbatim}
			gmodel['faith']
		\end{verbatim}

		\begin{figure}[!htbp]
			\centering{\includegraphics[scale=0.5]{figures/YN/Chapter5/Praktek/YNC5-7.png}}
			\caption{Vektorisasi Kata}
			\label{YNC5-7}
		\end{figure}

Output source code dibawah akan memunculkan data vektor untuk kata fall. Perhatikan figure \ref{YNC5-8}

		\begin{verbatim}
			gmodel['fall']
		\end{verbatim}

		\begin{figure}[!htbp]
			\centering{\includegraphics[scale=0.5]{figures/YN/Chapter5/Praktek/YNC5-8.png}}
			\caption{Vektorisasi Kata}
			\label{YNC5-8}
		\end{figure}

Output source code dibawah akan memunculkan data vektor untuk kata sick. Perhatikan figure \ref{YNC5-9}

		\begin{verbatim}
			gmodel['sick']
		\end{verbatim}

		\begin{figure}[!htbp]
			\centering{\includegraphics[scale=0.5]{figures/YN/Chapter5/Praktek/YNC5-9.png}}
			\caption{Vektorisasi Kata}
			\label{YNC5-9}
		\end{figure}

Output source code dibawah akan memunculkan data vektor untuk kata clear. Perhatikan figure \ref{YNC5-10}

		\begin{verbatim}
			gmodel['clear']
		\end{verbatim}

		\begin{figure}[!htbp]
			\centering{\includegraphics[scale=0.5]{figures/YN/Chapter5/Praktek/YNC5-10.png}}
			\caption{Vektorisasi Kata}
			\label{YNC5-10}
		\end{figure}

Output source code dibawah akan memunculkan data vektor untuk kata shine. Perhatikan figure \ref{YNC5-11}

		\begin{verbatim}
			gmodel['shine']
		\end{verbatim}

		\begin{figure}[!htbp]
			\centering{\includegraphics[scale=0.5]{figures/YN/Chapter5/Praktek/YNC5-11.png}}
			\caption{Vektorisasi Kata}
			\label{YNC5-11}
		\end{figure}

Output source code dibawah akan memunculkan data vektor untuk kata bag. Perhatikan figure \ref{YNC5-12}

		\begin{verbatim}
			gmodel['bag']
		\end{verbatim}

		\begin{figure}[!htbp]
			\centering{\includegraphics[scale=0.5]{figures/YN/Chapter5/Praktek/YNC5-12.png}}
			\caption{Vektorisasi Kata}
			\label{YNC5-12}
		\end{figure}

Output source code dibawah akan memunculkan data vektor untuk kata car. Perhatikan figure \ref{YNC5-13}

		\begin{verbatim}
			gmodel['car']
		\end{verbatim}

		\begin{figure}[!htbp]
			\centering{\includegraphics[scale=0.5]{figures/YN/Chapter5/Praktek/YNC5-13.png}}
			\caption{Vektorisasi Kata}
			\label{YNC5-13}
		\end{figure}

Output source code dibawah akan memunculkan data vektor untuk kata wash. Perhatikan figure \ref{YNC5-14}

		\begin{verbatim}
			gmodel['wash']
		\end{verbatim}

		\begin{figure}[!htbp]
			\centering{\includegraphics[scale=0.5]{figures/YN/Chapter5/Praktek/YNC5-14.png}}
			\caption{Vektorisasi Kata}
			\label{YNC5-14}
		\end{figure}
Output source code dibawah akan memunculkan data vektor untuk kata motor. Perhatikan figure \ref{YNC5-15}

		\begin{verbatim}
			gmodel['motor']
		\end{verbatim}

		\begin{figure}[!htbp]
			\centering{\includegraphics[scale=0.5]{figures/YN/Chapter5/Praktek/YNC5-15.png}}
			\caption{Vektorisasi Kata}
			\label{YNC5-15}
		\end{figure}

Output source code dibawah akan memunculkan data vektor untuk kata cycle. Perhatikan figure \ref{YNC5-16}

		\begin{verbatim}
			gmodel['cycle']
		\end{verbatim}

		\begin{figure}[!htbp]
			\centering{\includegraphics[scale=0.5]{figures/YN/Chapter5/Praktek/YNC5-16.png}}
			\caption{Vektorisasi Kata}
			\label{YNC5-16}
		\end{figure}

	\subitem Pada source code dibawah menunjukkan hasil score perbandingan kata apakah kata cycle dan motor memiliki ke samaan atau tidak, perhatikan figure \ref{YNC5-17}

		\begin{verbatim}
			gmodel.similarity('cycle','motor')
		\end{verbatim}

		\begin{figure}[!htbp]
			\centering{\includegraphics[scale=0.5]{figures/YN/Chapter5/Praktek/YNC5-17.png}}
			\caption{Perbandingan Kata}
			\label{YNC5-17}
		\end{figure}

Pada source code dibawah menunjukkan hasil score perbandingan kata apakah kata car dan wash memiliki ke samaan atau tidak, perhatikan figure \ref{YNC5-18}

		\begin{verbatim}
			gmodel.similarity('car','wash')
		\end{verbatim}

		\begin{figure}[!htbp]
			\centering{\includegraphics[scale=0.5]{figures/YN/Chapter5/Praktek/YNC5-18.png}}
			\caption{Perbandingan Kata}
			\label{YNC5-18}
		\end{figure}

Pada source code dibawah menunjukkan hasil score perbandingan kata apakah kata bag dan shine memiliki ke samaan atau tidak, perhatikan figure \ref{YNC5-19}

		\begin{verbatim}
			gmodel.similarity('bag','shine')
		\end{verbatim}

		\begin{figure}[!htbp]
			\centering{\includegraphics[scale=0.5]{figures/YN/Chapter5/Praktek/YNC5-19.png}}
			\caption{Perbandingan Kata}
			\label{YNC5-19}
		\end{figure}

Pada source code dibawah menunjukkan hasil score perbandingan kata apakah kata clear dan sick memiliki ke samaan atau tidak, perhatikan figure \ref{YNC5-20}

		\begin{verbatim}
			gmodel.similarity('clear','sick')
		\end{verbatim}

		\begin{figure}[!htbp]
			\centering{\includegraphics[scale=0.5]{figures/YN/Chapter5/Praktek/YNC5-20.png}}
			\caption{Perbandingan Kata}
			\label{YNC5-20}
		\end{figure}

Pada source code dibawah menunjukkan hasil score perbandingan kata apakah kata fall dan faith memiliki ke samaan atau tidak, perhatikan figure \ref{YNC5-21}

		\begin{verbatim}
			gmodel.similarity('fall','faith')
		\end{verbatim}

		\begin{figure}[!htbp]
			\centering{\includegraphics[scale=0.5]{figures/YN/Chapter5/Praktek/YNC5-21.png}}
			\caption{Perbandingan Kata}
			\label{YNC5-21}
		\end{figure}

\item Jelaskan dengan kata dan ilustrasi fungsi dari extact\_words dan PermuteSentences.
		
		\begin{verbatim}
		import re
		def extract_words(sent):
    			sent = sent.lower()
   		 	sent = re.sub(r'<[^>]+>', ' ', sent) 
    			sent = re.sub(r'(\w)\'(\w)', ' ', sent) 
    			sent = re.sub(r'\W', ' ', sent) 
    			sent = re.sub(r'\s+', ' ', sent) 
    		return sent.split()
		\end{verbatim}
	\subitem Dimana source code tersebut akan membersihkan tag-tag yang tidak perlu dan mengubahnya ke dalam bentuk array list perkata menggunakan split.

		\begin{verbatim}
		import random
		class PermuteSentences(object):
    			def __init__(self, sents):
        				self.sents = sents
        
    			def __iter__(self):
        		shuffled = list(self.sents)
        		random.shuffle(shuffled)
        			for sent in shuffled:
            		yield sent
		\end{verbatim}
	\subitem Dimana source code diatas berfungsi untuk melakukan pengacakan data supaya memperoleh data yang teratur.

\item Jelaskan fungsi library gensim dan Doc2Vec

	\begin{verbatim}
		from gensim.models.doc2vec import TaggedDocument
		from gensim.models import Doc2Vec
	\end{verbatim}	

	\subitem Dimana fungsi dari library gensim sendiri yaitu untuk memodelkan bahasa dengan sistem unsupervised yang artinya tanpa label. Douc2Vec itu sendiri berfungsi untuk melakukan vectorisasi kata dalam satu document. Pada source code diatas dapat dijelaskan bahwa pada baris pertama akan melakukan import TaggedDocument yang artinya memasukkan document beserta tag-tagnya. Pada baris kedua yaitu akan melakukan import model Doc2Vec dari library gensim.

\item Jelaskan dengan kata dan praktek cara menambahan data training dari file yang dimasukkan kepada variabel dalam rangka melatih model Doc2Vec.


	\begin{verbatim}
	import os
	unsup_sentences = []
	for dirname in ["train/pos","train/neg","train/unsup","test/pos","test/neg"]:
    	for fname in sorted(os.listdir("aclImdb/"+dirname)):
        	if fname[-4:] == '.txt':
           with open("aclImdb/"+dirname+"/"+fname,encoding='UTF-8') as f:
           sent = f.read()
           words = extract_words(sent)
           unsup_sentences.append(TaggedDocument(words,[dirname+"/"+fname]))
	\end{verbatim}

	\subitem Pada source code diatas akan dilakukan import data training dari folder Imdb. Dalam source code tersebut terdapat modul os yang dimana artinya program python yang ada pada pc kita dapat berinteraksi dengan sistem operasi yang pc kita gunakan. Setelah melakukan import pada modul os disini kita akan membuat sebuah variabel baru bernama unsup\_senteses. Selanjutnya kita akan mencari dimana folder Imdb disimpan. Selanjutnya kita akan melakukan open terhadap folder Imdb tersebut dengan menggunakan perintah with open. Selanjutnya kita akan membuat variabel baru yang bernama sent dan akan kita isikan perintah untuk membaca data folder Imdb. Selanjutnya kita akan membaut variabel baru lagi yang bernama word dan akan di isikan dengan perintah extract\_word yang dimana artinya kita akan melakukan extract kata yang ada pada variabel sent. Hasil output source code tersebut dapat dilihat pada figure \ref{YNC5-22}.

	\begin{figure}[!htbp]
		\centerline{\includegraphics[width=0.5\textwidth]{figures/YN/Chapter5/Praktek/YNC5-22.jpeg}}
		\caption{Import Data Training.}
		\label{YNC5-22}
	\end{figure}

	\begin{verbatim}
	 for dirname in ["txt_sentoken/pos","txt_sentoken/neg"]:
    	 for fname in sorted(os.listdir(dirname)):
        	 if fname[-4:] == '.txt':
           with open(dirname+"/"+fname,encoding='UTF-8') as f:
           for i, sent in enumerate(f):
           words = extract_words(sent)
           unsup_sentences.append(TaggedDocument(words,["%s/%s-%d" % (dirname,fname,i)]))
	\end{verbatim}

	\subitem Pada source code diatas kita akan melakukan import data training kembali sehingga membuat data training kita bertambah. Jika source code sebelumnya kita telah menambahkan data training dari folder Imdb, kali ini kita akan menambahkan data training dari folder txt\_sentoken. Untuk setiap fungsi source code nya sama saja seperti sebelumnya. Untuk output dari source code tersebut dapat dilihat pada figure \ref{YNC5-23}

	\begin{figure}[!htbp]
		\centerline{\includegraphics[width=0.5\textwidth]{figures/YN/Chapter5/Praktek/YNC5-23.PNG}}
		\caption{Import Data Training.}
		\label{YNC5-23}
	\end{figure}

	\begin{verbatim}
	with open("stanfordSentimentTreebank/original_rt_snippets.txt",encoding='UTF-8') as f:
    	for i, sent in enumerate(f):
        	words = extract_words(sent)
        	unsup_sentences.append(TaggedDocument(words,["rt-%d" % i]))
	\end{verbatim}

	\subitem Pada source code sebelumnya kita telah menambahkan data training dari folder Imdb dan txt\_stoken kali ini kita akan menambahkan data training kembali dengan menggunakan folder stanfordSentimentTreebank sehingga data training kita kembali bertambah. Untuk output dari source code tersebut dapat dilihat pada figure \ref{YNC5-24}.

	\begin{figure}[!htbp]
		\centerline{\includegraphics[width=0.5\textwidth]{figures/YN/Chapter5/Praktek/YNC5-24.PNG}}
		\caption{Import Data Training.}
		\label{YNC5-24}
	\end{figure}

\item Jelaskan dengan kata dan praktek mengapa harus dilakukan pengocokan dan pembersihan data.

	\begin{verbatim}
		mute=PermuteSentences(unsup_sentences)
	\end{verbatim}

	\subitem Pada source code diatas kita akan melakukan pengocokan atau pengacakan data. Sebelum dijadikan model kita perlu melakukan pengocokan agar mendapatkan hasil yang random dan bagus. Output dari source code diatas dapat dilihat pada figure \ref{YNC5-25}.

	\begin{figure}[!htbp]
		\centerline{\includegraphics[width=0.5\textwidth]{figures/YN/Chapter5/Praktek/YNC5-25.PNG}}
		\caption{Proses Instansiasi Pengocokan Data.}
		\label{YNC5-25}
	\end{figure}

	\subitem Untuk pembersihan data perhatikan source code berikut,

	\begin{verbatim}
		model.delete_temporary_training_data(keep_inference=True)
	\end{verbatim}

	\subitem Dikarenakan kita telah melakukan running data google yang jumlahnya mencapai 300000 data, maka perlu dilakukannya pembersihan data. Mengapa harus di bersihkan ?, agar memory pada pc kita memiliki space yang lega. Output dari source code tersebut dapat dilihat pada figure \ref{YNC5-26}.

	\begin{figure}[!htbp]
		\centerline{\includegraphics[width=0.5\textwidth]{figures/YN/Chapter5/Praktek/YNC5-26.PNG}}
		\caption{Proses Instansiasi Pembersihan Data.}
		\label{YNC5-26}
	\end{figure}

\item Jelaskan dengan kata dan praktek kenapa model harus di save dan kenapa temporari training harus dihapus.

	\begin{verbatim}
		model.save('haci.d2v')
	\end{verbatim}
	
	\subitem Pada source code diatas kita akan melakukan save pada data training yang sudah kita latih. Mengapa dilakukan save ?, agar kita dapat melakukan load saat ingin melatih datanya lagi. Data yang kita simpan akan terlihat seperti pada figure \ref{YNC5=27}

	\begin{figure}[!htbp]
		\centerline{\includegraphics[width=0.5\textwidth]{figures/YN/Chapter5/Praktek/YNC5-27.PNG}}
		\caption{Save Data Training.}
		\label{YNC5-27}
	\end{figure}

	\subitem Apabila kita ingin menghapus tempory data maka dapat menggunakan source code dibawah,

	\begin{verbatim}
		model.delete_temporary_training_data(keep_inference=True)
	\end{verbatim}		

		 Tujuan dari penghapusan temporary data yaitu untuk membuat memory pada pc kita menjadi lega dikarenakan memory itu terbatas. Untuk ouput dari source code tersebut dapat dilihat pada figure \ref{YNC5-26}.

	\begin{figure}[!htbp]
		\centerline{\includegraphics[width=0.5\textwidth]{figures/YN/Chapter5/Praktek/YNC5-26.PNG}}
		\caption{Proses Instansiasi Penghapusan Temporary Training Data.}
		\label{YNC5-26}
	\end{figure}

\item Jelaskan dengan kata dan praktek maksud dari infer\_code.

	\begin{verbatim}
		model.infer_vector(extract_words("I will go home"))
	\end{verbatim}

	\subitem Pada source code diatas kuta akan melakukan infer\_code menggunakan model vector. Dimana fungsinya untuk menghitung atau mengkalkulasikan vektor dari kata yang dinputkan.  Untuk output dari source code tersebut dapat dilihat pada figure \ref{YNC5-28}

	\begin{figure}[!htbp]
		\centerline{\includegraphics[width=0.5\textwidth]{figures/YN/Chapter5/Praktek/YNC5-28.PNG}}
		\caption{Hasil Infer Vector.}
		\label{YNC5-28}
	\end{figure}

\item Jelaskan dengan kata dan praktek maksud dari consine\_similarity.

	\begin{verbatim}
	from sklearn.metrics.pairwise import cosine_similarity
	cosine_similarity(
        	[model.infer_vector(extract_words("she going to school, after wash hand"))],
       	 [model.infer_vector(extract_words("Services sucks."))])
	\end{verbatim}

	\subitem Dimana consine\_similarity ini akan melakukan perbandingan vektorisasi terhadap dua kata, apabila hasil vektorisasi dari kedua kata tersebut adalah 50\% hal ini dapat diperkirakan data kata tersebut memiliki kesamaan, dan apabila kurang dari 50\% maka kata tersebut tidak memiliki kesamaan. Output dari source code diatas dapat dilihat pada figure \ref{YNC5-29}.

	\begin{figure}[!htbp]
		\centerline{\includegraphics[width=0.5\textwidth]{figures/YN/Chapter5/Praktek/YNC5-29.PNG}}
		\caption{Consine Similarity.}
		\label{YNC5-29}
	\end{figure}

\item Jalankan dengan praktek score dari cross validation masing-masing metode.

	\subitem Untuk source code cross validation pada metode CLF dapat dilihat dibawah ini,

	\begin{verbatim}
		scores = cross_val_score(clf, sentvecs, sentiments, cv=5)
		np.mean(scores), np.std(scores)
	\end{verbatim}

		    Dimana score yang diperoleh dari cross validation clf dapat dilihat pada figure \ref{YNC5-30}

	\begin{figure}[!htbp]
		\centerline{\includegraphics[width=0.5\textwidth]{figures/YN/Chapter5/Praktek/YNC5-30.PNG}}
		\caption{Cross Validation CLF.}
		\label{YNC5-30}
	\end{figure}

	\subitem Untuk source code cross validation pada metode Randon Forest, dapat dilihat dibawah ini,

	\begin{verbatim}
		scores = cross_val_score(clfrf, sentvecs, sentiments, cv=5)
		np.mean(scores), np.std(scores)
	\end{verbatim}

		  Dimana score yang diperoleh dari cross validation Random Forest dapat dilihat pada figure \ref{YNC5-31}

	\begin{figure}[!htbp]
		\centerline{\includegraphics[width=0.5\textwidth]{figures/YN/Chapter5/Praktek/YNC5-31.PNG}}
		\caption{Cross Validation Random Forest.}
		\label{YNC5-31}
	\end{figure}	

	\subitem Untuk source code cross validation pada metode CountVectorizer, dapat dilihat dibawah ini,

	\begin{verbatim}
		scores = cross_val_score(pipeline,sentences,sentiments, cv=5)
		np.mean(scores), np.std(scores)
	\end{verbatim}

		  Dimana score yang diperoleh dari cross validation CountVectorizer dapat dilihat pada figure \ref{YNC5-32}

	\begin{figure}[!htbp]
		\centerline{\includegraphics[width=0.5\textwidth]{figures/YN/Chapter5/Praktek/YNC5-32.PNG}}
		\caption{Cross Validation CountVectorizer.}
		\label{YNC5-32}
	\end{figure}	

\end{enumerate}

\subsection{Penanganan Error / Yusniar Nur Syarif Sidiq / 1164089}
\begin{enumerate}

\item Lakukan screenshot code yang error saat melakukan praktikum, tuliskan codenya dan nama errornya, lalu bagaimana cara menanganinya.
	\subitem Untuk error yang saya dapat dapat dilihat pada figure \ref{YNC5-34}.
	
	\begin{figure}[!htbp]
		\centerline{\includegraphics[width=0.5\textwidth]{figures/YN/Chapter5/Praktek/YNC5-34.PNG}}
		\caption{Name Error.}
		\label{YNC5-34}
	\end{figure}	
	
	\subitem Jenis error tersebut merupakan name error atau bisa juga disebut variabel error, dikarenakan variabel tersebut tidak terbaca,
		
	\begin{verbatim}
		model.delet_temporary_training_data(keep_inference=True)
	\end{verbatim}

	\subitem Solusi dari error tersebut yaitu kita perlu membuat variabel model terlebih dahulu, maka kitakan source code pada figure \ref{YNC5-35} diatas source code errornya.

	\begin{figure}[!htbp]
		\centerline{\includegraphics[width=0.5\textwidth]{figures/YN/Chapter5/Praktek/YNC5-35.PNG}}
		\caption{Penangan Error.}
		\label{YNC5-35}
	\end{figure}	

\end{enumerate}

\section{Imron Sumadireja / 1164076}
\subsection{Teori}
\begin{enumerate}
\item Jelaskan kenapa kata-kata harus dilakukan vektorisasi. Dilengkapi dengan ilustrasi \par
Karena untuk mengetahui presentase kata yang sering muncul dalam setiap kalimatnya, yang berguna untuk menetukan kata kunci, serta untuk memberikan kemudahan bagi mesin untuk mempelajari kata yang bentuk aslinya serupa namun tak sama seperti bebek dan ayam. Selain itu fungsi dari vektorisasi kata ini untuk membaca setiap kata-katanya dengan merubah kata menjadi sebuah angka atau identitas itu sendiri \ref{vek1}.
		\begin{figure}[!htbp]
		\centerline{\includegraphics[width=0.5\textwidth]{figures/im/vek1.png}}
		\caption{Ilustrasi Vektorisasi Kata.}
		\label{vek1}
		\end{figure}

\item Jelaskan mengapa dimensi dari vektor dataset google bisa sampai 300. Dilengkapi dengan ilustrasi \par
Karena pada masing-masing objek itu memiliki identitas tersendiri, contoh sederhananya. Pada dataset google ini memiliki 3 buah objek diantaranya, cat, dog, dan spatula. Lalu dari masing-masing objek itu dibandingkan datasetnya antara cat dan dog lalu cat dan spatula. Hasil yang didapatkan untuk cat dan dog itu sekitar 76\% sedangkan untuk cat dan spatula itu memiliki presentase 12\% itu artinya bahwa mesin dapat membedakan objek yang hampir serupa namun tak sama. Untuk ilustrasi sederhananya bisa dilihat pada gambar \ref{vek2}.
		\begin{figure}[!htbp]
		\centerline{\includegraphics[width=0.5\textwidth]{figures/im/vek2.png}}
		\caption{Ilustrasi Vektorisasi Dataset Google.}
		\label{vek2}
		\end{figure}

\item Jelaskan konsep vektorisasi untuk kata. Dilengkapi dengan ilustrasi \par
Konsep untuk vektorisasi kata ini sama halnya dengan kita input suatu kata pada mesin pencari. Lalu hasilnya akan mengeluarkan berupa referensi mengenai kata tersebut. Jadi data kata tersebut didapatkan dari hasil pengolahan pada kalimat-kalimat sebelumnya yang telah diolah. Contoh sederhananya pada kalimat berikut ( Please subscribe my channel thank you guys ), pada kalimat tersebut terdapat konteks yakni channel, kata tersebut akan dijadikan data latih untuk mesin. Jadi ketika kita inputkan kata channel, maka mesin akan menampilkan keterkaitannya dengan kata tersebut. Ilustrasinya bisa dilihat pada gambar berikut \ref{vek3}.
		\begin{figure}[!htbp]
		\centerline{\includegraphics[width=0.5\textwidth]{figures/im/vek3.png}}
		\caption{Ilustrasi Vektorisasi Kata.}
		\label{vek3}
		\end{figure}

\item Jelaskan konesep vektorisasai untuk dokumen. Dilengkapi dengan ilustrasi \par
Sama halnya dengan vektorisasi kata, yang membedakan hanya pada proses awalnya. Untuk vektorisasi dokumen ini, mesin akan membaca semua kalimat yang terdapat pada dokumen tersebut, lalu kalimat yang terdapat pada dokumen akan di pecah menjadi kata-kata. Untuk ilustrasinya dapat dilihat pada gambar berikut \ref{vek4}.
		\begin{figure}[!htbp]
		\centerline{\includegraphics[width=0.5\textwidth]{figures/im/vek4.png}}
		\caption{Ilustrasi Vektorisasi Dokumen.}
		\label{vek4}
		\end{figure}

\item Jelaskan apa mean dan standar deviasi. \par
Mean merupakan nilai rata-rata. Untuk mendapatkan mean ini kita tinggal menjumlahkan data yang tersedia lalu dibagi dengan banyaknya data tersebut. Sedangkan standar deviasi adalah nilai statistik yang digunakan untuk menentukan bagaimana sebaran data dalam sampel, dan seberapa dekat titik data individu dengan rata-rata nilai sampel.

\item Jelaskan apa itu skip-gram. Dilengkapi dengan ilustrasi \par
Skip-gram sama halnya dengan vektorisasi kata, namun untuk skip-gram ia dibalik prosesnya. Yang sebelumnya dari kalimat lalu di olah untuk menemukan salah satu kata, kali ini dari keyword tersebut akan diolah menjadi suatu kalimat yang memiliki keterkaitannya dengan keyword tersebut. Ilustrasinya bisa dilihat pada gambar berikut \ref{vek5}.
		\begin{figure}[!htbp]
		\centerline{\includegraphics[width=0.5\textwidth]{figures/im/vek5.png}}
		\caption{Ilustrasi Skip-Gram.}
		\label{vek6}
		\end{figure}

\end{enumerate}

\subsection{Praktikum / Imron Sumadireja / 1164076}
\begin{enumerate}
\item Cobalah dataset google, dan jelaskan vektor dari kata love, faith, fall, sick, clear, shine, bag, car, wash, motor, cycle dan cobalah untuk melakukan perbandingan similirati dari masing-masing kata tersebut \par
\begin{verbatim}
import gensim, logging
logging.basicConfig(format='%(asctime)s : %(levelname)s : %(message)s', level=logging.INFO)
gmodel = gensim.models.KeyedVectors.load_word2vec_format('F:/Imron/Kuliah/Semester 6/../GoogleNews-vectors-negative300.bin', binary=True, limit=500000)
\end{verbatim}
Hasil keluaran pada gambar \ref{sim1} untuk keluaran ke 99 itu untuk import library gensim. Gensim itu sendiri berguna untuk melakukan pemodelan dengan dataset atau topik yang telah ditentukan. Untuk keluaran 100 logging itu library opsional karena logging hanya untuk menampilkan berupa log untuk setiap code yang dijalankan. Dan keluaran 101 itu hasil load data dari file vektor google itu, disini saya menggunakan limit karena kondisi laptop yang tidak mempuni untuk melakukan running data sebesar 3 juta file, jadi saya batasi hanya melakukan running 500 ribu data saja.
		\begin{figure}[!htbp]
		\centerline{\includegraphics[width=0.5\textwidth]{figures/im/sim1.png}}
		\caption{Hasil import gensim dan logging.}
		\label{sim1}
		\end{figure}

\begin{verbatim}
gmodel['love']
\end{verbatim}
Keluaran pada gambar \ref{sim2} ini untuk menampilkan data hasil vektorisasi data dari kata love, hasil tersebut berjumlah kurang lebih 300 data. Untuk kata faith, fall, sick, clear, dan lain-lain itu sama saja untuk menampilkan hasil vektorisasi dari setiap katanya. Selanjutnya akan dibandingkan setiap katanya untuk menentukan presentase kemiripan pada setiap kata tersebut.
		\begin{figure}[!htbp]
		\centerline{\includegraphics[width=0.5\textwidth]{figures/im/sim2.png}}
		\caption{Data vektor dari kata love.}
		\label{sim2}
		\end{figure}

\begin{verbatim}
gmodel.similarity('love','faith')
\end{verbatim}
Hasil keluaran pada gambar \ref{sim3} merupakan presentase dari perbandingan kata love dan faith dengan hasil 37\%. Hasil tersebut didapatkan dengan 37\% karena mesin dapat membaca bahwa kata love dengan faith ini hampir memiliki arti yang sama.
		\begin{figure}[!htbp]
		\centerline{\includegraphics[width=0.5\textwidth]{figures/im/sim3.png}}
		\caption{Hasil similaritas.}
		\label{sim3}
		\end{figure}

\begin{verbatim}
gmodel.similarity('fall','sick')
\end{verbatim}
Hasil keluaran pada gambar \ref{sim4} adalah hasil presentase dari perbandingan antara kata fall dan sick dengan hasil 8\%. Hasil tersebut tidak terlalu baik, karena mesin menentukan bahwa kata fall dengan sick itu tidak serupa karena memiliki presentase yang kecil.
		\begin{figure}[!htbp]
		\centerline{\includegraphics[width=0.5\textwidth]{figures/im/sim4.png}}
		\caption{Hasil similaritas.}
		\label{sim4}
		\end{figure}

\begin{verbatim}
gmodel.similarity('clear','shine')
\end{verbatim}
Hasil keluaran pada gambar \ref{sim5} merupakan hasil presentase dari perbandingan antara kata clear dengan shine dengan hasil presentase 11\%. Presentase tersebut dapat memberikan kita gambaran bahwa mesin dapat menentukan bahwa kata clear dan shine itu berbeda.
		\begin{figure}[!htbp]
		\centerline{\includegraphics[width=0.5\textwidth]{figures/im/sim5.png}}
		\caption{Hasil similaritas.}
		\label{sim5}
		\end{figure}

\begin{verbatim}
gmodel.similarity('bag','wash')
\end{verbatim}
Hasil keluaran pada gambar \ref{sim6} adalah hasil presentase dari perbandingan kata bag dengan wash dan hasil presentasenya 18\%. Presentase tersebut cukup bagus karena mesin dapat membedakan kata antara bag dan wash.
		\begin{figure}[!htbp]
		\centerline{\includegraphics[width=0.5\textwidth]{figures/im/sim6.png}}
		\caption{Hasil similaritas.}
		\label{sim6}
		\end{figure}

\begin{verbatim}
gmodel.similarity('car','motor')
\end{verbatim}
Hasil keluaran pada gambar \ref{sim7} merupakan hasil dari perbandingan kata car dengan motor dan hasil presentasenya 48\%. Ini terbilang bagus dan bisa dikatakan mirip karena mesin dapat menentukan presentase yang cukup baik.
		\begin{figure}[!htbp]
		\centerline{\includegraphics[width=0.5\textwidth]{figures/im/sim7.png}}
		\caption{Hasil similaritas.}
		\label{sim7}
		\end{figure}

\item Jelaskan dengan kata dan ilustrasi fungsi dari extract words dan PermuteSentences
\begin{verbatim}
import re
def extract_words(sent):
    sent = sent.lower()
    sent = re.sub(r'<[^>]+>', ' ', sent) #hapus tag html
    sent = re.sub(r'(\w)\'(\w)', ' ', sent) #hapus petik satu
    sent = re.sub(r'\W', ' ', sent) #hapus tanda baca
    sent = re.sub(r'\s+', ' ', sent) #hapus spasi yang berurutan
    return sent.split()
\end{verbatim}
Code tersebut berguna untuk menghapus tag-tag html yang tidak diperlukan pada dokumen yang telah dilakukan pelatihan dan vektorisasi, hasilnya pun tidak mengeluarkna apa-apa, hanya pemberitahuan bahwa code tersebut telah berhasil dijalankan. Dan code tersebut akan merubah kalimat yang didalamnya menjadi list kata-kata dalam bentuk array dengan menggunakan split.

\begin{verbatim}
import random
class PermuteSentences(object):
    def __init__(self, sents):
        self.sents = sents
        
    def __iter__(self):
        shuffled = list(self.sents)
        random.shuffle(shuffled)
        for sent in shuffled:
            yield sent
\end{verbatim}
Code tersebut untuk melakukan pengacakan data agar data tersebut dapat menghasilkan hasil yang mempuni. Code tersebut bisa digunakan atau tidak, tetapi alangkah baiknya untuk digunakan agar hasil akhir yang didapatkan bisa prima.

\item Jelaskan fungsi dari librari gensim TaggedDocument dan Doc2Vec disertai praktek pemakaiannya
\begin{verbatim}
from gensim.models.doc2vec import TaggedDocument
from gensim.models import Doc2Vec
\end{verbatim}
Fungsi dari library gensim untuk pemodelan topik tanpa pengawasan dan pemrosesan bahasa alamai, atau bisa kita sebut dengan unsupervised. Fungsi dari doc2vec itu sendiri ialah untuk membandingkan bobot data yang terdapat pada dokumen yang lainnya, apakah kata-kata didalamnya ada yang sama atau tidak. Lalu untuk tagged document itu memasukan kata-kata pada setiap dokumennya untuk di vektorisasi. Dan hasil dari running coding tersebut tidak mengeluarkan apa-apa, seperti gambar berikut \ref{sim8}
		\begin{figure}[!htbp]
		\centerline{\includegraphics[width=0.5\textwidth]{figures/im/sim8.png}}
		\caption{Import library.}
		\label{sim8}
		\end{figure}

\item Jelaskan dengan kata dan praktek cara menambahkan data training dari file yang dimasukkan kepada variabel dalam rangka melatih model doc2vec
\begin{verbatim}
import os
unsup_sentences = []
for dirname in ["train/pos","train/neg","train/unsup","test/pos","test/neg"]:
    for fname in sorted(os.listdir("aclImdb/"+dirname)):
        if fname[-4:] == '.txt':
            with open("aclImdb/"+dirname+"/"+fname,encoding='UTF-8') as f:
                sent = f.read()
                words = extract_words(sent)
                unsup_sentences.append(TaggedDocument(words,[dirname+"/"+fname]))
\end{verbatim}
Untuk menambahkan data training kita melakukan import library os, library os itu sendiri berfungsi untuk melakukan interaksi antara python dengan os laptop kita masing-masing, setelah itu kita buat variable unsup sentences. Selanjutnya pilih direktori tempat data kita disimpan. Selanjutnya itu untuk menyortir data yang terdapat pada folder aclImdb dan membaca file tersebut dengan ektensi .txt. Hasil dari code pertama tersebut ialah terdapatnya data hasil running dari folder aclImdb \ref{sim9}
		\begin{figure}[!htbp]
		\centerline{\includegraphics[width=0.5\textwidth]{figures/im/sim9.png}}
		\caption{Menambahkan data training.}
		\label{sim9}
		\end{figure}

\begin{verbatim}
with open("stanfordSentimentTreebank/original_rt_snippets.txt",encoding='UTF-8') as f:
    for i, sent in enumerate(f):
        words = extract_words(sent)
        unsup_sentences.append(TaggedDocument(words,["rt-%d" % i]))
\end{verbatim}
Untuk code tersebut sama saja seperti yang sebelumnya, yang membedakan hanya file yang akan digunakan untuk dilakukan training. Hasil dari code tersebut menambahkan data sekitar 60 ribu data, seperti gambar berikut \ref{sim10}
		\begin{figure}[!htbp]
		\centerline{\includegraphics[width=0.5\textwidth]{figures/im/sim10.png}}
		\caption{Menambahkan data training.}
		\label{sim10}
		\end{figure}

\begin{verbatim}
for dirname in ["review_polarity/txt_sentoken/pos","review_polarity/txt_sentoken/neg"]:
    for fname in sorted(os.listdir(dirname)):
        if fname[-4:] == '.txt':
            with open(dirname+"/"+fname,encoding='UTF-8') as f:
                for i, sent in enumerate(f):
                    words = extract_words(sent)
                    unsup_sentences.append(TaggedDocument(words,["%s/%s-%d" % (dirname,fname,i)]))
\end{verbatim}
Untuk code tersebut sama seperti code sebelumnya yakni untuk menambahkan data training, bagian ketiga ini menambahkan data training sekitar 10 ribu data. Hasilnya seperti gambar berikut \ref{sim11}
		\begin{figure}[!htbp]
		\centerline{\includegraphics[width=0.5\textwidth]{figures/im/sim11.png}}
		\caption{Menambahkan data training.}
		\label{sim11}
		\end{figure}

\item Jelaskan dengan kata dan praktek kenapa harus dilakukan pengocokan dan pembersihan data
\begin{verbatim}
# Pengacakan data
mute = PermuteSentences(unsup_sentences)

# Pembersihan data
model.delete_temporary_training_data(keep_inference=True)
\end{verbatim}
Untuk bagian pengacakan data itu berguna untuk mengacak data supaya pada saat data di running bisa berjalan lebih baik dan hasil presentase akhirnya bisa lebih baik. Sedangkan untuk pembersihan data untuk memberikan ruang bagi ram laptop kita setelah melakukan running data sebanyak 3 juta lebih, agar lebih ringan saat proses selanjutnya. Hasil dari pengacakan data tidak ditampilkan pada spyder, namun untuk hasil dari pembersihan data pada gambar berikut \ref{sim12}. Dan sebelumnya memori yang terpakai itu sekitar 80\% lebih, setelah dikosongkan jadi 64\%.
		\begin{figure}[!htbp]
		\centerline{\includegraphics[width=0.5\textwidth]{figures/im/sim12.png}}
		\caption{Random dan Clear Data.}
		\label{sim12}
		\end{figure}

\item Jelaskan dengan kata dan praktek kenapa model harus di save dan kenapa temporari training harus dihapus
\begin{verbatim}
# Save data
model.save('ocean.d2v')

# Delete temporary data
model.delete_temporary_training_data(keep_inference=True)
\end{verbatim}
Save data ini berfungsi untuk menyimpan file hasil dari proses pelatihan data sebelumnya, model tersebut dilakukan penyimpanan untuk memberikan keringanan pada ram agar saat kita akan melakukan pelatihan lagi, model tersebut tinggal di load saja tanpa harus melakukan pelatihan dari awal dan bisa menghemat waktu untuk hasilnya bisa dilihat pada gambar \ref{sim13}. Sedangkan untuk delete temporary training data ini berguna untuk menghapus data latihan yang sebelumnya sudah dilakukan dan disimpan, bertujuan untuk memberikan keringanan pada ram. Karena setelah melakukan proses pelatihan ram biasanya jadi tercekik sampai laptop jadi lag. Itulah fungsi dari delete temporary training data.
		\begin{figure}[!htbp]
		\centerline{\includegraphics[width=0.5\textwidth]{figures/im/sim13.png}}
		\caption{Save dan Delete Temporary Training Data.}
		\label{sim13}
		\end{figure}

\item Jalankan dengan kata dan praktek maksud dari infer code
\begin{verbatim}
model.infer_vector(extract_words("jangan lupa tobat guys"))
\end{verbatim}
Infer vector itu sendiri berguna untuk membandingkan kata yang tercantum dengan vektor yang mana pada dokumen yang sudah di load pada step sebelumnya. Selain itu infer vector juga untuk menghitung atau mengkalkulasikan vektor dari kata yang dicantumkan dari model yang telah kita buat. Alangkah baiknya kata yang dicantumkan itu lebih panjang lagi agar hasilnya bisa lebih baik lagi. Hasilnya seperti gambar berikut \ref{sim14}
		\begin{figure}[!htbp]
		\centerline{\includegraphics[width=0.5\textwidth]{figures/im/sim14.png}}
		\caption{Infer Code.}
		\label{sim14}
		\end{figure}

\item Jelaskan dengan praktek dan kata maksud dari cosine similarity
\begin{verbatim}
from sklearn.metrics.pairwise import cosine_similarity
cosine_similarity(
        [model.infer_vector(extract_words("katakatatatata"))],
        [model.infer_vector(extract_words("Services sucks."))])
\end{verbatim}
Cosine similarity ini berfungsi untuk membandingkan vektorisasi data diantara kedua kata yang di inputkan, jika hasil presentase dari kedua kata tersebut lebih dari 50\% itu memiliki kemungkinan kata tersebut terdapat dalam 1 file. Namun jika kurang dari 50\% itu kemungkinan kata tersebut tidak terdapat dalam 1 file. Hasil yang didapatkan pada code tersebut hanya 0.8\% itu dikarenakan kata pertama dan kedua tidak memiliki kesamaan vektorisasi dan tidak terdapat pada salah satu dokumen. Untuk hasil presentasenya seperti gambar berikut \ref{sim15}
		\begin{figure}[!htbp]
		\centerline{\includegraphics[width=0.5\textwidth]{figures/im/sim15.png}}
		\caption{Cosine Similarity.}
		\label{sim15}
		\end{figure}

\end{enumerate}
\chapter{Discussion}
Please tell more about conclusion and how to the next work of this study.

\section{Imron Sumadireja / 1164076}
\subsection{Teori}
\begin{enumerate}

\item Jelaskan kenapa file suara harus di lakukan MFCC. Dilengkapi dengan ilustrasi atau gambar. \par
MFCC merupakan koefisien yang merepresentasikan audio. Ekstraksi ciri dalam proses ini ditandai dengan pengubahan data suara menjadi citra berupa spektrum gelombang. File audio dilakukan MFCC itu agar objek suara dapat diubah menjadi bentuk matrix. Suara tersebut akan menjadi vektor yang nantinya akan diolah sebagai keluaran. Untuk ilustrasi sederhananya bisa dilihat pada gambar \ref{cc1}. Gambar tersebut menjelaskan tahapan-tahapan kenapa file suara harus dilakukan MFCC. Selain itu untuk memberikan kemudahan kepada mesin dalam mempelajari suara tersebut karena mesin tidak dapat membaca teks maka dari itu diperlukan MFCC untuk merubah suara tersebut menjadi vektor.
		\begin{figure}[!htbp]
		\centerline{\includegraphics[width=0.5\textwidth]{figures/im/cc1.png}}
		\caption{MFCC.}
		\label{cc1}
		\end{figure}

\item Jelaskan konsep dasar neural network. Dilengkapi dengan ilustrasi atau gambar. \par
Konsep sederhana dari neural network hampir mirip dengan proses belajar pada anak-anak yakni dengan memetakan pola baru yang didapatkan dari inputan untuk membuat pola baru pada keluaran. Contoh sederhana tersebut menganalogikan kinerja otak manusia. Neural network itu sendiri terdiri dari sebuah unit pemroses yang disebut neuron yang berisi adder dan fungsi aktivasi. Fungsi aktivasi itu sendiri untuk mengatur keluaran yang diberikan oleh neuron. Neural network ini mengadopsi mekanisme berpikir sebuah sistem atau aplikasi yang menyerupai otak manusia, baik untuk pemrosesan berbagai sinyal elemen yang diterima, toleransi terhadap kesalahan/error, dan juga prallel processing. Karakteristik dari neural network dilihat dari pola hubungan antar neuron, metode penentuan bobot dari tiap koneksi, dan fungsi aktivasinya. Untuk ilustrasinya bisa dilihat pada gambar \ref{cc2}
		\begin{figure}[!htbp]
		\centerline{\includegraphics[width=0.5\textwidth]{figures/im/cc2.png}}
		\caption{Konsep Dasar Neural Network.}
		\label{cc2}
		\end{figure}

\item Jelaskan konsep pembobotan dalam neural network. Dilengkapi dengan ilustrasi atau gambar. \par
Pembobotan ini akan menentukan serta penanda dari sebuah konektivitas. Pada proses neural network dimulai dari input yang diterima oleh neuron beserta dengan nilai bobot dari tiap-tiap input yang ada. Setelah masuk ke dalam neuron, nilai input yang ada akan dijumlahkan oleh suatu fungsi penambahan. Hasil penjumlahan tersebut akan diproses oleh fungsi aktivasi oleh setiap neuron, hasil penjumlahan tersebut akan dibandingkan dengan nilai ambang tertentu. Jika nilai dari hasil penjumlahan tersebut melebihi nilai ambang maka aktivasi neuron akan dibatalkan, namun sebaliknya jika hasil penjumlahan dibawah nilai ambang maka neuron akan diaktifkan. Setelah neuron aktif selanjutnya akan mengirimkan nilai output melalui bobot-bobot keluarannya ke semua neuroon yang berhubungan. Ilustrasinya bisa dilihat pada gambar \ref{cc3}
		\begin{figure}[!htbp]
		\centerline{\includegraphics[width=0.5\textwidth]{figures/im/cc3.png}}
		\caption{Konsep Pembobotan Neural Network.}
		\label{cc3}
		\end{figure}

\item Jelaskan konsep fungsi aktifasi dalam neural network. Dilengkapi dengan ilustrasi atau gambar. \par
Fungsi aktivasi ini merupakan operasi matematik yang dikenakan pada sinyal output. Fungsi ini digunakan untuk mengaktifkan atau menonaktifkan neuron. Fungsi aktivasi ini terbagi setidaknya menjadi 6, dianranya sebagai berikut:
\begin{itemize}
\item a. Fungsi Undak Biner Hard Limit, fungsi ini biasanya digunakan oleh jaringan lapiran tunggal untuk mengkonversi nilai input dari suatu variabel yang bernilai kontinu ke suatu nilai output biner 0 atau 1.
\item b. Fungsi Undak Biner Threshold, fungsi ini menggunakan nilai ambang sebagai batasnya.
\item c. Fungsi Bipolar Symetric Hard Limit, fungsi ini memiliki output bernilai 1, 0 atau -1.
\item d. Fungsi Bipolar dengan Threshold, fungsi ini mempunyai output yang bernilai 1, 0 atau -1 untuk batas nilai ambang tertentu.
\item e. Fungsi Linear atau Identitas.
\end{itemize}
Untuk ilustrasi sederhananya bisa dilihat pada gambar \ref{cc4}. Gambar tersebut merupakan salah satu contoh dari fungsi aktivasi bipolar.
		\begin{figure}[!htbp]
		\centerline{\includegraphics[width=0.5\textwidth]{figures/im/cc4.png}}
		\caption{Konsep Fungsi Aktivasi Neural Network.}
		\label{cc4}
		\end{figure}

\item Jelaskan cara membaca hasil plot dari MFCC. Dilengkapi dengan ilustrasi atau gambar. \par
Membaca plot MFCC ini bisa kita lihat pada gambar \ref{cc5}. Gambar tersebut menjelaskan bahwa pada waktu ke 5 daya atau desible yang dikelurakan pada nada tersebut paling keras pada 20 Hz, selain itu pada 40 -120 Hz itu daya atau desible yang dikeluarkan pada musik yang telah di plotting. Begitupun seterusnya bahwa warna yang paling gelap itu merupakan daya atau desible yang paling tinggi dibandingkan dengan warna yang cerah. Untuk yang berwarna merah itu suara dibawah pendengaran frekuensi manusia, jadi tidak dapat terdengar secara langsung. 
		\begin{figure}[!htbp]
		\centerline{\includegraphics[width=0.5\textwidth]{figures/im/cc5.png}}
		\caption{Membaca Plot MFCC.}
		\label{cc5}
		\end{figure}

\item Jelaskan apa itu one-hot encoding. Dilengkapi dengan ilustrasi kode dan atau gambar. \par
Sederhananya one-hot encoding ini untuk merubah hasil data vektorisasi menjadi bilangan biner 0 dan 1 serta membuat keterangan pada atribut tersebut menjadi label. Unutuk ilustrasi sederhananya bisa dilihat pada gambar \ref{cc6}
		\begin{figure}[!htbp]
		\centerline{\includegraphics[width=0.5\textwidth]{figures/im/cc6.png}}
		\caption{One-Hot Encoding.}
		\label{cc6}
		\end{figure}

\item Jelaskan apa fungsi dari np.unique dan to categorical dalam kode program. Dilengkapi dengan ilustrasi atau gambar. \par
Fungsi dari np.unique adalah untuk menemukan elemen yang berbeda atau unik array, dan dapat mengembalikan elemen unik array tersebut yang diurutkan. Untuk ilustrasi sederhananya bisa dilihat pada gambar \ref{cc7}. Gambar tersebut menjelaskan bahwa unique itu sendiri akan mengambil data yang berbeda dari variabel a yang berada dalam fungsi array dan hasilnya seperti gambar tersebut.
		\begin{figure}[!htbp]
		\centerline{\includegraphics[width=0.5\textwidth]{figures/im/cc7.png}}
		\caption{np.unique.}
		\label{cc7}
		\end{figure}

Fungsi dari to\_categorical untuk mengubah vektor yang berupa integer menjadi matrix dengan kelas biner. Untuk ilustrasinya bisa dilihat pada gambar \ref{cc71}
		\begin{figure}[!htbp]
		\centerline{\includegraphics[width=0.5\textwidth]{figures/im/cc71.png}}
		\caption{to\_categorical.}
		\label{cc71}
		\end{figure}

\item Jelaskan apa fungsi dari Sequential dalam kode program. Dilengkapi dengan ilustrasi atau gambar.\par
Salah satu jenis model yang digunakan dalam perhitungan. Sequential ini membangun tumpukan linear yang berurutan. Contoh sederhananya sebagai berikut \ref{cc8}
		\begin{figure}[!htbp]
		\centerline{\includegraphics[width=0.5\textwidth]{figures/im/cc8.png}}
		\caption{Sequential.}
		\label{cc8}
		\end{figure}

\end{enumerate}

\section{Yusniar Nur Syarif Sidiq / 1164089}
\subsection{Pemahaman Teori Yusniar Nur Syarif Sidiq / 1164089}
\begin{enumerate}

\item Jelaskan kenapa file suara harus di lakukan MFCC. dilengkapi dengan ilustrasi !
	\subitem MFCC (Mel Frequency Cepstrum Coefficients) merupakan metode untuk melakukan feature extraction, sebuah proses yang mengkonversikan signal suara menjadi beberapa parameter. Dimana dalam python MFCC digunakan untuk melakukan etraction suara menjadi bentuk vektor. Mengapa perlu dilakukan ?, dikarenakan mesin tidak dapat membaca data selain bilangan biner dan vektor, maka dari itu untuk mempermudah pembacaan data oleh mesi perlu dilakukannya MFCC. Disini saya akan memberikan ilustrasi sederhana dimana ada sebuah Mechine Learning yang ingin membaca data dalam bentuk gelombang suara. Mechine Learning tidak akan bisa membaca data tersebut, Mengapa ?, karena data tersebut masih berbentuk gelombang suara, untuk memahami Mechine Learning perlu data dalam bentuk vektor, maka gelombang suara tersebut akan diubah menjadi bentuk vektor.

\item Jelaskan konsep dasar neural network. Dilengkapi dengan ilustrasi atau gambar !
	\subitem Neural Network atau Jaringan Saraf dimana memiliki neuron dan pada tiap neuron akan saling terhubung pada lapisan - laposan berikutnya. Pada lapisan pertama dimana terjadinya proses menerima input sedangkan pada lapisan terakhir akan memberikan output. Neural Network akan mengadopsi mekanisme berpikir sebuah sistem atau aplikasi yang menyerupai otak manusia, baik dalam pemrosesan berbagai sinyal elemen yang diterima, toleransi kesalahan atau error, dan parallel processing. Dimana terdapat dua data apel dan jeruk, data tersebut akan diinputkan pada setiap layer. Data tersebut akan dilakukan proses hidden pada hidden layer. Data yang dinputkan tadi akan diubah kedalam bentuk binner, dan outputnya akan terjadi misalnya 01 itu adalah apel dan 10 itu adalah jeruk. Untuk contoh figure dapat dilihat pada figure \ref{YNC6-1}.

	\begin{figure}[!htbp]
		\centering{\includegraphics[scale=0.5]{figures/YN/Chapter6/Teori/YNC6-1.png}}
		\caption{Konsep Nural Network}
		\label{YNC6-1}
	\end{figure}

\item Jelaskan konsep pembobotan dalam neural network. Dilengkapi dengan ilustrasi atau gambar !
	\subitem Dimana Bobot akan mewakili koneksi antar unit. Apabila bobot dari node 1 ke node 2 memiliki nilai yang kebih besar, hal ini menandakan bahwa neuron 1 memiliki pengaruh yang lebih besar terhadap neuron 2. Apabila nilai bobot mendekati nol hal ini menandakan akan merubah input namun tidak mengubah output dan apabila nilai bobot negatif hal ini menandakan akan meningkatkan input namun mengurangi output yang artinya bobot akan menentukan seberapa besar pengaruh input terhadap output. Untuk ilustrasi figure dapat dilihat pada figure \ref{YNC6-2}.

	\begin{figure}[!htbp]
		\centering{\includegraphics[scale=0.5]{figures/YN/Chapter6/Teori/YNC6-2.png}}
		\caption{Konsep Pembobotan}
		\label{YNC6-2}
	\end{figure}	

\item Jelaskan konsep fungsi aktivasi dalam neural network. Dilengkapi dengan ilustrasi atau gambar !
	\subitem Dimana fungsi aktivasi ini merupakan operasi dalam matematika yang digunakan pada sinyal output y. Fungsi ini sering digunakan untuk mengaktifkan atau menonaktifkan neuron. Dimana perilaku dari nural network ini ditentukan oleh bobot dan input-output fungsi aktivasi yang telah ditetapkan. Contohnya dimana jaringan lapisan tunggal akan menkonversi nilai input dari suatu variabel yang bernilai continue ke suatu nilai output biner yaitu angka 0 dan 1. Untuk contoh figure dapat dilihat pada pada figure \ref{YNC6-3}

	\begin{figure}[!htbp]
		\centering{\includegraphics[scale=0.5]{figures/YN/Chapter6/Teori/YNC6-3.png}}
		\caption{Fungsi Aktivasi}
		\label{YNC6-3}
	\end{figure}	
	
\item Jelaskan cara membaca hasil plot dari MFCC. Dilengkapi dengan ilustrasi atau gambar !
	\subitem Perhatikan figure \ref{YNC6-4}, dimana terlihat bahwa warna biru merupakan suara terendah, mengapa ?, dikarenakan warna biru bernilai -300, sedangkan warna merah merupakah suara tertinggi dan bernilai 200. Jika diilustrasikan ada sebuah musik yang sedang dimainkan dan diperoleh datanya lalu diplot, maka akan terlihat seperti pada figure \ref{YNC6-4}

	\begin{figure}[!htbp]
		\centering{\includegraphics[scale=0.5]{figures/YN/Chapter6/Teori/YNC6-4.png}}
		\caption{Plot MFCC}
		\label{YNC6-4}
	\end{figure}	

\item Jelaskan apa itu one-hot encoding. Dilengkapi dengan ilustrasi kode dan atau gambar !
	\subitem One-hot encoding merupakan representasi variabel berkategorikan sebagai vektor biner yang artinya hanya akang 0 dan 1. Dimana mengharuskan nilai categorinya berbentuk biner dan setiap nilai integer akan dipresentasikan kedalam vektor biner. Contoh figure dapat dilihat pada figure \ref{YNC6-5}

	\begin{figure}[!htbp]
		\centering{\includegraphics[scale=0.5]{figures/YN/Chapter6/Teori/YNC6-5.png}}
		\caption{One-Hot Encoding}
		\label{YNC6-5}
	\end{figure}	

\item Jelaskan apa fungsi dari np.unique dan to\_categorical dalam kode program . Dilengkapi dengan ilustrasi atau gambar !
	\subitem Fungsi dari np.unique yaitu sebagai indeks array input yang memberikan nilai unik, sebagai indeks array unik yang merekonstruksi array input, dan menghitung berapa kali setiap nilai unik muncul dalam array input. Contoh source code dalam pemrograman dapat dilihat pada figure \ref{YNC6-6}

	\begin{figure}[!htbp]
		\centering{\includegraphics[scale=0.5]{figures/YN/Chapter6/Teori/YNC6-6.png}}
		\caption{np.unique}
		\label{YNC6-6}
	\end{figure}	

	\subitem Fungsi dari to\_categorical yaitu untuk mengubah vektor ke dalam matriks class biner. Contoh source code to\_categorical dapat dilihat pada figure \ref{YNC6-7}

	\begin{figure}[!htbp]
		\centering{\includegraphics[scale=0.5]{figures/YN/Chapter6/Teori/YNC6-7.png}}
		\caption{categorical}
		\label{YNC6-7}
	\end{figure}	

\item Jelaskan apa fungsi dari Sequential dalam kode program. Dilengkapi dengan ilustrasi atau gambar !
	\subitem Dimana fungsi dari Sequential dalam source code program hanyalah sebagai tumpukan linear lapisan, dimana contih source code dalam program dapat dilihat pada figure \ref{YNC6-8}

	\begin{figure}[!htbp]
		\centering{\includegraphics[scale=0.5]{figures/YN/Chapter6/Teori/YNC6-8.png}}
		\caption{Sequential}
		\label{YNC6-8}
	\end{figure}

\end{enumerate}
\chapter{Discussion}
Please tell more about conclusion and how to the next work of this study.

\section{Imron Sumadireja / 1164076}
\subsection{Teori}
\begin{enumerate}
\item Jelaskan kenapa file teks harus dilakukan tokenizer. Dilengkapi dengan ilustrasi atau gambar \par
Tokenizer merupakan proses membagi teks yang dapat berupa kalimat, paragraf atau dokumen menjadi kata-kata atau bagian-bagian tertentu dalam kalimat tersebut. Sebagai contoh dari kalimat `Aku mau istirahat dulu ya untuk hari ini', menjadi `Aku', `Mau', `'Istirahat',`Dulu',`Ya',`Untuk',`Hari',`Ini'. Yang menjadi acuan yakni tanda baca dan spasi. Untuk ilustrasinya bisa dilihat pada gambar \ref{toke1}
		\begin{figure}[!htbp]
		\centerline{\includegraphics[width=0.5\textwidth]{figures/im/toke1.png}}
		\caption{Ilustrasi Tokenizer.}
		\label{toke1}
		\end{figure}

\item Jelaskan konsep dasar K Fold Cross Validation pada dataset komentar Youtube pada kode listing 7.1.dilengkapi dengan ilustrasi atau gambar \par
\lstinputlisting[firstline=1, lastline=2]{src/ron/tujuh.py}
Pada code tesebut terdapat kfold yang bertujuan untuk melakukan split data menjadi 5 bagian dari dataset komentar Youtube tersebut. Sehingga dari setiap data yang sudah dibagi tersebut akan menghasilkan presentase dari setiap bagiannya, untuk menghasilkan hasil akhir dengan presentase yang cukup baik. Untuk ilustrasi sederhananya bisa dilihat pada gambar \ref{toke2}
		\begin{figure}[!htbp]
		\centerline{\includegraphics[width=0.5\textwidth]{figures/im/toke2.png}}
		\caption{Ilustrasi K-Fold Cross Validation.}
		\label{toke2}
		\end{figure}

\item Jelaskan apa maksudnya kode program for train, test in splits. Dilengkapi dengan ilustrasi atau gambar \par
For train ini untuk membagi data tersebut menjadi data training. Sedangkan test in splits ini untuk menguji apakah dataset tersebut sudah dibagi menjadi beberapa bagian atau masih menumpuk. Untuk ilustrasi sederhananya bisa dilihat pada gambar \ref{toke3}
		\begin{figure}[!htbp]
		\centerline{\includegraphics[width=0.5\textwidth]{figures/im/toke3.png}}
		\caption{For Train dan Test In Splits.}
		\label{toke3}
		\end{figure}

\item Jelaskan apa maksudnya kode program train content = d['CONTENT'].iloc[train idx] dan test content = d['CONTENT'].iloc[test idx]. dilengkapi dengan ilustrasi atau gambar \par
Code tersebut untuk mengambil data pada kolom atau index CONTENT yang merupakan bagian dari train\_idx dan test\_idx. Contoh sederhananya ketika data telah diubah menjadi data train dan data test maka kita dapat memilihnya untuk ditampilkan pada kolom yang diinginkan. Untuk ilustrasinya bisa dilihat pada gambar \ref{toke4}
		\begin{figure}[!htbp]
		\centerline{\includegraphics[width=0.5\textwidth]{figures/im/toke4.png}}
		\caption{Ilustrasi Content.}
		\label{toke4}
		\end{figure}

\item Jelaskan apa maksud dari fungsi tokenizer = Tokenizer(num words=2000) dan tokenizer fit on texts(train content), dilengkapi dengan ilustrasi atau gambar \par
Variabel tokenizer ini berfungsi untuk melakukan vektorisasi data kedalam bentuk token sebanyak 2000 kata. Dan selanjutnya akan melakukan fit tokenizer hanya untuk data training saja tidak dengan data testingnya. Untuk ilustrasinya bisa dilihat pada gambar \ref{toke5}
		\begin{figure}[!htbp]
		\centerline{\includegraphics[width=0.5\textwidth]{figures/im/toke5.png}}
		\caption{Ilustrasi Fungsi Tokenizer.}
		\label{toke5}
		\end{figure}

\item Jelaskan apa maksud dari fungsi d train inputs = tokenizer.texts to matrix(train content, mode='tdf') dan d test inputs = tokenizer.texts to matrix(test content, mode='tdf'), dilengkapi dengan ilustrasi kode dan atau gambar \par
Code pertama yakni untuk merubah atau vektorisasi dari data training yang berupa teks menjadi matrix dengan menggunakan model tfidf. Code kedua sama halnya seperti yang pertama, yang membedakan hanya data yang dirubah. Untuk yang ini merubah data testing yang berupa teks menjadi matrix dengan menggunakan model tfidf. Ilustrasinya bisa dilihat pada gambar \ref{toke6}
		\begin{figure}[!htbp]
		\centerline{\includegraphics[width=0.5\textwidth]{figures/im/toke6.png}}
		\caption{Ilustrasi d\_train\_inputs.}
		\label{toke6}
		\end{figure}

\item Jelaskan apa maksud dari fungsi d train inputs = d train inputs/np.amax(np.absolute(d train dan d test inputs = d test inputs/np.amax(np.absolute(d test inputs)), dilengkapi dengan ilustrasi atau gambar \par
Code tersebut akan membagi matrix tfidf dengan amax yakni mengembalikan nilai maksimum array, dan hasilnya akan dimasukan kedalam variabel d train inputs untuk data train dan d test inputs untuk data testing dengan nominal absolut atau tanda adanya bilangan negatif dan koma. Ilustrasinya bisa dilihat pada gambar \ref{toke7}
		\begin{figure}[!htbp]
		\centerline{\includegraphics[width=0.5\textwidth]{figures/im/toke7.png}}
		\caption{Ilustrasi d\_train\_inputs.}
		\label{toke7}
		\end{figure}

\item Jelaskan apa maksud fungsi dari d train outputs = np utils.to categorical(d['CLASS'].iloc[train dan d test outputs = np utils.to categorical(d['CLASS'].iloc[test idx]) dalam kode program, dilengkapi dengan ilustrasi atau gambar \par
Code tersebut ditujukan untuk melakukan one-hot encoding agar dapat digunakan pada proses neural network. One-hot encoding tersebut diambil dari `CLASS' yang terdapat 2 neuron, yakni (1,0) dan (01) karena pilihannya hanya 2 yakni spam atau bukan spam.

\item Jelaskan apa maksud dari fungsi di listing 7.2. Gambarkan ilustrasi Neural Network nya dari model kode tersebut \par
\lstinputlisting[firstline=4, lastline=9]{src/ron/tujuh.py}
Code tersebut bertujuan untuk melakukan sequential, membandingkan setiap satu elemen dengan cara satu persatu secara beruntun. Dimana terdapat 512 neurons inputan dengan jumlah shape 2000 yang sebelumnya sudah dinormalisasi. Lalu model tersebut dilakukan aktivasi dengan menggunakan fungsi relu. Kemudian dilakukan pemotongan model tree sebesar 50\% agar tidak terjadi overfitting. Pada outputnya terdapat 2 neurons yakni (1,0) dan (0,1) yang akan di aktivasi dengan menggunakan fungsi softmax.

\item Jelaskan apa maksud dari fungsi di listing 7.3 dengan parameter tersebut \par
\lstinputlisting[firstline=11, lastline=11]{src/ron/tujuh.py}
Code tersebut akan melakukan compile model dengan fungsi optimizer, loss dan matrix.

\item Jelaskan apa itu Deep Learning \par
Deep Learning adalah salah satu cabang dari ilmu machine learning yang terdiri dari algoritma pemodelan abstraksi tingkat tinggi pada data menggunakan sekumpulan fungsi transformasi non-linear yang ditata berlapis-lapis dan mendalam. Teknik dan algoritma dalam machine learning dapat digunakan baik untuk supervised learning dan unsupervised learning.

\item Jelaskan apa itu Deep Neural Network, dan apa bedanya dengan Deep Learning \par
DNN adalah salah satu algoritma berbasis jaringan saraf tiruan yang memiliki dari 1 lapisan saraf tersembunyi yang dapat digunakan untuk pengambilan keputun. Perbedaannya dengan deep learning, yakni: DNN dapat menentukan dan mencerna karakteristik tertentu di suatu rangkaian data, kapabilitas lebih kompleks untuk mempelajari, mencerna, dan mengklasifikasikan data, serta dibagi ke dalam berbagai lapisan dengan fungsi yang berbeda-beda.

\item Jelaskan dengan ilustrasi gambar buatan sendiri(langkah per langkah) bagaimana perhitungan algoritma konvolusi dengan ukuran stride (NPM mod3+1) x (NPM mod3+1) yang terdapat max pooling \par
Konvolusi pada gambar dilakukan dalam image processing untuk menerapkan operator yang memiliki nilai output dari piksel gambar yang berasal dari kombinasi linear nilai input piksel, semakin nilai piksel tersebut maka kualitas gambar bisa semakin bagus. Untuk ilustrasinya bisa dilihat pada gambar \ref{toke13}
		\begin{figure}[!htbp]
		\centerline{\includegraphics[width=0.5\textwidth]{figures/im/toke13.png}}
		\caption{Ilustrasi Algoritma Konvolusi.}
		\label{toke13}
		\end{figure}
\end{enumerate}

\subsection{Praktikum / Imron Sumadireja / 1164076}
\begin{enumerate}
\item Jelaskan kode program pada blok In[1]. Jelaskan arti dari setiap baris kode yang dibuat(harus beda dengan teman sekelas) dan hasil luarannya dari komputer sendiri \par
\lstinputlisting[firstline=2, lastline=4]{src/ron/Math.py}
Hasil keluaran dari code tersebut dapat dilihat pada gambar \ref{math1}. Hasil tersebut tidak mengeluarkan apa-apa karena hanya melakukan import library saja.
		\begin{figure}[!htbp]
		\centerline{\includegraphics[width=0.5\textwidth]{figures/im/math1.png}}
		\caption{Keluaran blok In[1].}
		\label{math1}
		\end{figure}

\item Jelaskan kode program pada blok In[2]. Jelaskan arti dari setiap baris kode yang dibuat(harus beda dengan teman sekelas) dan hasil luarannya dari komputer sendiri \par
\lstinputlisting[firstline=8, lastline=21]{src/ron/Math.py}
Untuk keluarannya bisa dilihat pada gambar \ref{math2}. Code tersebut akan memanggil dan membaca file .csv yang di dalamnya terdapat data-data yang akan dilakukan vektorisasi.
		\begin{figure}[!htbp]
		\centerline{\includegraphics[width=0.5\textwidth]{figures/im/math2.png}}
		\caption{Keluaran blok In[2].}
		\label{math2}
		\end{figure}

\item Jelaskan kode program pada blok In[3]. Jelaskan arti dari setiap baris kode yang dibuat(harus beda dengan teman sekelas) dan hasil luarannya dari komputer sendiri \par
\lstinputlisting[firstline=25, lastline=29]{src/ron/Math.py}
Keluaran dari code tersebut dapat dilihat pada gambar \ref{math3}. Pada gambar tersebut akan melakukan pengacaka data agar hasil akhir dari data tersebut dapat menghasilkan hasil yang terbaik.
		\begin{figure}[!htbp]
		\centerline{\includegraphics[width=0.5\textwidth]{figures/im/math3.png}}
		\caption{Keluaran blok In[3].}
		\label{math3}
		\end{figure}

\item Jelaskan kode program pada blok In[4]. Jelaskan arti dari setiap baris kode yang dibuat(harus beda dengan teman sekelas) dan hasil luarannya dari komputer sendiri \par
\lstinputlisting[firstline=25, lastline=29]{src/ron/Math.py}
Keluaran dari code tersebut dapat dilihat pada gambar \ref{math4}. Pada code tersebut akan melakukan proses pelatihan data dengan data telah dibagi menjadi data training dengan menggunakan library numpy.
		\begin{figure}[!htbp]
		\centerline{\includegraphics[width=0.5\textwidth]{figures/im/math4.png}}
		\caption{Keluaran blok In[4].}
		\label{math4}
		\end{figure}

\item Jelaskan kode program pada blok In[5]. Jelaskan arti dari setiap baris kode yang dibuat(harus beda dengan teman sekelas) dan hasil luarannya dari komputer sendiri \par
\lstinputlisting[firstline=25, lastline=29]{src/ron/Math.py}
Keluaran dari code tersebut dapat dilihat pada gambar \ref{math5}. Pada code tersebut akan melakukan import librayr sklearn dengan metode labelencoder dan onehotencoder.
		\begin{figure}[!htbp]
		\centerline{\includegraphics[width=0.5\textwidth]{figures/im/math5.png}}
		\caption{Keluaran blok In[5].}
		\label{math5}
		\end{figure}
\end{enumerate}



\section{Yusniar Nur Syarif Sidiq / 1164089}
\subsection{Teori / Yusniar Nur Syarif Sidiq / 1164089}
\begin{enumerate}

\item Jelaskan kenapa file teks harus di lakukan tokenizer. Dilengkapi dengan ilustrasi atau gambar !
	\subitem Sebelumnya kita harus tau terlebih dahulu apa itu Tokenizer, yaitu sebuah proses pembagian terhadap kalimat yang berada dalam dokumen sehingga menjadi sebuah bagian - bagian kata atau bisa kita sebut denga token. Dalam dataset Youtube Tokenizer digunakan untuk melakukan vektorisasi data, sehingga dapat kita simpulkan bahwa data yang telah kita buat dokumen pada chapter 6 yaitu data spam dan bukan spam akan dilakukan vektorisasi dengan menggunakan Tokenizer ini. Ilustrasi sederhana mengenai Tokenizer ini misalkan saya memiliki sebuah kalimat Nama Saya Adalah Yusniar jika gunakan fungsi Tokenizer ini akan dipecah menjadi kata per kata, perhatikan figure \ref{YNC7-1}.

	\begin{figure}[!htbp!]
		\centerline{\includegraphics[width=0.5\textwidth]{figures/YN/Chapter7/YNC7-1.png}}
		\caption{Contoh Tokenizer.}
		\label{YNC7-1}
	\end{figure}

\item Jelaskan konsep dasar K Fold Cross Validation pada dataset komentar Youtube pada source code dibawah. Dilengkapi dengan ilustrasi atau gambar !
	\lstinputlisting[firstline=3, lastline=4]{src/yns/sc7.py}
	\subitem Pada variabel kfold berisikan StratifieldKFold yang dimana akan diisikan sebuah sempel dan dibagi menjadi 5 dengan nsplits pada setiap class. Lalu akan dibuat variabel baru yang dinamakan dengan 2splits dan disikan class dari dataset Youtube. Pada data yang telah dilakukan split akan diperoleh hasil presentase akhir. Ilustrasi yang saya berikan misalkan ada sebuah data lalu data tersebut akan dibagi menjadi data testing dan training lalu dilakukan fungsi Cross Validation untuk memperoleh hasil presentase akhir. Perhatikan figure \ref{YNC7-2}.

	\begin{figure}[!htbp!]
		\centerline{\includegraphics[width=0.5\textwidth]{figures/YN/Chapter7/YNC7-2.png}}
		\caption{Konsep Dasar K Fold Validation.}
		\label{YNC7-2}
	\end{figure}

\item Jelaskan apa maksudnya kode program for train, test in splits. Dilengkapi dengan ilustrasi atau gambar !
	\subitem Dimana fungsi tersebut digunakan untuk melakukan pengujian terhadap data yang sudah di split atau belum dalam dataset dan tidak akan terjadi penumpukan, maksudnya pada setiap class tidak akan menampilkan id yang sama. Kali ini saya akan memberikan ilustrasi sederhana, misalkan ada seseorang yang ingin menyumbangkan buku cerita kepada perpustakan, lalu pihak perpustakaan akan menerimanya, dikarenakan yang di sumbangkan adalah buku cerita dan bermacam - macam maka tidak akan ada buku yang sama hal ini sama saja bisa dibilang tidak adanya id yang sama. Untuk ilustrasi nya perhatikan figure \ref{YNC7-3}.

	\begin{figure}[!htbp!]
		\centerline{\includegraphics[width=0.5\textwidth]{figures/YN/Chapter7/YNC7-3.png}}
		\caption{Train And Test In Split.}
		\label{YNC7-3}
	\end{figure}

\item Jelaskan apa yang dimaksud kode program dibawah. Dilengkapi dengan ilustrasi atau gambar !
	\lstinputlisting[firstline=21, lastline=22]{src/yns/sc7.py}
	\subitem Dimana source code tersebut akan mengambil data dari kolom Content yaitu kolom tersebut merupakan bagian dari train\_idx dan test\_idx. Ilustrasi yang saya berikan yaitu apabila kita telah mengubah suatu data menjadi data training atau data testing maka kita dapat menampilkan data tersebut dengan isi kolom yang kita mau.

\item Jelaskan apa maksud dari fungsi tokenizer = Tokenizer(num\_words=2000) dan tokenizer.fit\_on\_texts(train\_content), dilengkapi ilustrasi atau gambar !
	\subitem Variabel tokenizer tersebut akan melakukan proses vektorisasi kata sebanyak 2000 kata dengan menggunakan fungsi Tokenizer. Pada source code tokenizer.fit\_on\_texts(train\_content) akan melakukan fit dengan menggunakan fungsi tokenizer akan tetapi hanya pada data training saja pada kolom content. Untuk ilustrasi dapat dilihat pada figure \ref{YNC7-4}

	\begin{figure}[!htbp!]
		\centerline{\includegraphics[width=0.5\textwidth]{figures/YN/Chapter7/YNC7-4.png}}
		\caption{Fungsi Tokenizer.}
		\label{YNC7-4}
	\end{figure}

\item Jelaskan apa yang dimaksud dari fungsi source code dibawah !
	\lstinputlisting[firstline=42, lastline=43]{src/yns/sc7.py}
	\subitem Pada source code baris pertama akan melakukan vektorisasi dari data training yang dimana data tersebut berbentuk string atau teks dan diubah kedalam bentuk matrix menggunakan model tdf. Untuk source code baris kedua akan melakukan vektorisasi data testing yang akan diubah kedalam bentuk matrix dengan mode tdf. Dimana kita ilustrasi source code nya adalah seperti yang ditampilkan pada figure \ref{YNC7-5}.

	\begin{figure}[!htbp!]
		\centerline{\includegraphics[width=0.5\textwidth]{figures/YN/Chapter7/YNC7-5.png}}
		\caption{Vektorisasi TDF Data Training Dan Testing.}
		\label{YNC7-5}
	\end{figure}

\item Jelaskan apa yang dimaksud dari fungsi d\_train\_inputs = d\_train\_inputs/np.amax(np.absolute(d\_train)) dan  d\_test\_inputs = d\_test\_inputs/np.amax(np.absolute(d\_test)), dilengkapi dengan ilustrasi atau gambar !
	\subitem Pada source code tersebut akan melakukan pembagian data matrix yang sudah kita buat pada data trainig dan testing dan akan mengembalikan nilai maksimum array menggunakan class amax. Untuk ilustrasi dapat dilihat pada figure \ref{YNC7-6}.

	\begin{figure}[!htbp!]
		\centerline{\includegraphics[width=0.5\textwidth]{figures/YN/Chapter7/YNC7-6.png}}
		\caption{Pembagian Data Matrix.}
		\label{YNC7-6}
	\end{figure}

\item Jelaskan apa yang dimaksud fungsi d\_train\_outputs = np\_utils.to\_categorical(d[CLASS].iloc[train\_idx] dan  d\_test\_outputs = np\_utils.to\_categorical(d[CLASS].iloc[test\_idx] dalam kode program dilengkapi dengan ilustrasi atau gamabar !
	\subitem Source code tersebut digunakan untuk melakuakan fungsi yang dinamakan One-hot encoding yang nantinya data tersebut dapat digunakan dalam proses Neural Network. Data yang diambil terdiri dari CLASS yaitu Spam dan No Spam dari data training dan data testing. Untuk ilustrasi dapat dilihat pada figure \ref{YNC7-7}.

	\begin{figure}[!htbp!]
		\centerline{\includegraphics[width=0.5\textwidth]{figures/YN/Chapter7/YNC7-7.png}}
		\caption{One-hot Encoding.}
		\label{YNC7-7}
	\end{figure}

\item Jelaskan apa yang dimaksud dengan dari fungsi source code dibawah,
	\lstinputlisting[firstline=8, lastline=13]{src/yns/sc7.py}
	\subitem Source Code tersebut akan melakukan sequential yang diamana akan melakukan perbandingan pada setiap elemen secara satu persatu dan berurut. Terdapat 512 Neurons dengan jumlah input shape sebesar 2000 yang dimana sudah dilakukan normalisasi. Mpdel tersebut akan dilakukan aktivasi menggunakan model relu lalu dilakukan dropout atau pemotongan sebesar 50\%. Pada neurons selanjutnya terdapat 2  yang kemudian akan dilakukan aktivasi menggunakan fungsi softmax.

\item Jelaskan apa yang dimaksud dari fungsi pada source code dibawah dengan parameter tersebut !
	\lstinputlisting[firstline=17, lastline=17]{src/yns/sc7.py}
	\subitem Source code tersebut akan melakukan proses compile dengan menggunakan fungsi optimizer dan memunculkan data loss dan akurasi matrix.

\item Jelaskan apa itu Deep Learning !
	\subitem Deep Learning merupakan sebuah cabang dari Mechine Learning dimana konesep yang deep learning ini hampir serupa dengan Mechine Learning hanya saja deep learning dilakukan dengan metode yang lebih cerdas, contohnya dalam menditeksi wajah itu termasuk deep learning.

\item Jelaskan apa itu Deep Neural Network dan apa bedanya dengan Deep Learning !
	\subitem Deep Neural Network merupakan sebuah algoritma berbasis Neural Netowork yang digunakan dalam pengambilan sebuah keputusan. Perbedaan DNN dengan DL antara lain DNN merupakan algoritma yang digunakan terhadap DL sedangkan DL yang akan menggunakan algoritma tersebut.

\item Jelaskan dengan ilustrasi gambar langkah per langkah bagaimana perhitungan algoritma konvolusi dengan ukuran stide (NPM mod 3 + 1) x (NPM mod 3 + 1) yang terdapat max pooling !

	\begin{itemize}
		\item Pertama tentukan nilai (x,y) dan (x1,y1)
		\item Nilai tersebut dibuat kedalam bentuk matrix
		\item Jika sudah berbentuk matrix lakukan perkalian antar baris dan deret
		\item Hasil perkalian tersebut dijumlahkan sehingga akan menghasilkan nilai matrix (3x3)
	\end{itemize}

	\subitem Untuk ilustrasi dapat dilihat pada figure \ref{YNC7-8}

	\begin{figure}[!htbp!]
		\centerline{\includegraphics[width=0.5\textwidth]{figures/YN/Chapter7/YNC7-8.png}}
		\caption{Algoritma Konvolusi Dengan Matrix (3x3).}
		\label{YNC7-8}
	\end{figure}	


\end{enumerate}
\include{section/chapter8}
\include{section/chapter9}
\include{section/chapter10}
\include{section/chapter11}
\include{section/chapter12}
\include{section/chapter13}
\include{section/chapter14}

%now enable appendix numbering format and include any appendices
\appendix
\include{section/appendix1}
\include{section/appendix2}

%next line adds the Bibliography to the contents page
\addcontentsline{toc}{chapter}{Bibliography}
%uncomment next line to change bibliography name to references
%\renewcommand{\bibname}{References}
\bibliography{references}        %use a bibtex bibliography file refs.bib
\bibliographystyle{plain}  %use the plain bibliography style

\end{document}

