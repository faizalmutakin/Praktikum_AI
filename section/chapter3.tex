\chapter{Methods}

\section{The data}
PLease tell where is the data come from, a little brief of company can be put here.

\section{Method 1}
Definition, steps, algoritm or equation of method 1 and how to apply into your data
\section{Method 2}
Definition, steps, algoritm or equation of method 2 and how to apply into your data

\section{Yusniar Nur Syarif Sidiq/1164089}
\begin{enumerate}

\item Random Forest merupakan algoritma yang digunakan terhadapap klasifikasi data dalam jumlah yang besar. Klasifikasi pada random forest dilakukan dengan penggabungan dicision tree dengan melakuakn training terhadap sempel data yang dimiliki. Semakin banyak dicision tree maka data yang di dapat akan semakin akurat. Untuk gambar Random Forest dapat dilihat pada figure \ref{YNRF}

	\begin{figure}[ht]
	\centerline{\includegraphics[width=1\textwidth]{figures/YN/RF.PNG}}
	\caption{Random Forest.}
	\label{YNRF}
	\end{figure}

\item Pertama download dataset terlebih dahulu lalu buka dengan menggunakan software spyder guna melihat isi dari dataset tersebut. Data tersebut memiliki extensi file bernama .txt dan didalamnya terdapat class dari field. Misalnya saja pada data jenis burung memiliki file index dan angka, dimana index berisi angka yang memiliki makna berupa jenis burung atau bahkan nama burung sedangkan field memiliki isi nilai berupa 0 dan 1 yang dimana sifatnya boolean atau Ya dan Tidak. Hal ini dikarenakan komputer hanya dapat membaca bilangan biner maka dari itu field yang di isikan berupa angka. Artinya angka 0 berarti tidak dan angka 1 berarti Ya.

\item Cross Validation adalah sebuah teknik validasi model yang digunakan untuk menilai bagaimana hasil analisis statistik akan digeneralisasi ke kumpulan data independen. Cross validation digunakan dengan tujuan prediksi, dan bila kita ingin memperkirakan seberapa akurat model model prediksi yang dilakukan dalam sebuah praktek. Tujuan dari cross validation yaitu untuk mendefinisikan dataset guna menguju dalam fase pelatihan untuk membatasi masalah seperti overfitting dan underfitting serta mendapatkan wawasan tentang bagaimana model akan digeneralisasikan ke set data independen.

\item Dimana Score 44 \% diperoleh dari hasil pengelohan dataset jenis burung. Dimana akan dilakukan proses pembagian data testing dan data training lalu diproses dan menghasilkan score sebanyak 44 \% dimana menjelaskan bahwa score tersebut digunakan sebagai pembanding dalam tingkat keakuratannya. Pada dicision tree akan memperoleh data lebih kecil yaitu sebanyak 27 \% hal ini dikarenakan data yang diolah menggunakan dicision tree dibagi menjadi beberapa tree dan lalu disimpulkan untuk mendapatkan data yang akurat. Pada SVM akan memperoleh score sebanyak 29 \% hal ini dikarenakan data yang dimiliki masih bernilai netral sehingga tingkat keakuratannya masih belum jelas.

\item Untuk membaca confusion matriks dapat menggunakan source code sebagai berikut,
	\begin{verbatim}
		import numpy as np
		np.set_printoptions(precision=2)
		plt.figure(figsize=(60,60), dpi=300)
		plot_confusion_matrix(cm, classes=birds, normalize=True)
		plt.show()
	\end{verbatim}

Dimana numpy akan mengurus semua data yang berhubungan dengan matrix. Pada source code tersebut digunakan dalam melakukan read pada dataset burung dengan menggunakan metode confusion matrix. Dalam confusion matrix memiliki 4 istilah yaitu True Positive yang merupakan data posotif yang terditeksi benar, True Negatif yang merupakan data negatif akan tetapi terditeksi benar, False Positif merupakan data negatif namun terditeksi sebagai data positif, False Negatif merupakan data posotif namun terditeksi sebagai data negatif. Adapun contoh hasil read dataset menggunakan confusion matrix dapat dilihat pada figure \ref{YNCM}
	
	\begin{figure}[ht]
	\centerline{\includegraphics[width=1\textwidth]{figures/YN/YNCM.PNG}}
	\caption{Confusion Matrix.}
	\label{YNCM}
	\end{figure}

\item Voting merupakan proses pemilihan dari tree yang dimana akan dimunculkan hasilnya dan disimpulkan menjadi informasi yang pasti. Untuk kebih jelasnya saya akan memberikan sebuah contoh bagaimana voting beerja.
	
	\begin{figure}[ht]
	\centerline{\includegraphics[width=1\textwidth]{figures/YN/YNVoting.PNG}}
	\caption{Voting.}
	\label{YNV}
	\end{figure}

Dimana ditunjukkan pada figure \ref{YNV} terdapat 3 tree. Dalam tree tersebut akan dilakukan proses voting. Saya akan memberikan contoh kasus, dimana akan diadakan voting untuk menentukan sebuah mobil. Dalam tree akan diberikan sejumlah data misalnya saja data tersebut berupa gambar, yang dimana data tersebut akan dipilih dengan cara voting. Hasil voting akhir dari setiap tree menunjukkan mobil jazz, yang berarti kesimpulan dari data yang telah diberikan menyatakan gambar tersebut adalah mobil jazz. Bagaimana apabila terjadi perbedaan data misalnya saja pada tree 1 dan 2 menunyatakan mobil jazz sedangkan pada tree 3 menyatakan mobil yaris, maka kesimpulan yang di ambil adalah mobil jazz dikarenakan hasil voting terbanyak adalah mobil jazz.

\end{enumerate}