\chapter{Mengenal Kecerdasan Buatan dan Scikit-Learn}
Buku umum yang digunakan adalah \cite{russell2016artificial} dan  
untuk sebelum UTS menggunakan buku \textit{Python Artificial Intelligence Projects for Beginners}\cite{eckroth2018python}.
Dengan praktek menggunakan python 3 dan editor anaconda dan library python scikit-learn.
Tujuan pembelajaran pada pertemuan pertama antara lain:
\begin{enumerate}
\item
Mengerti definisi kecerdasan buatan, sejarah kecerdasan buatan, perkembangan dan penggunaan di perusahaan
\item
Memahami cara instalasi dan pemakaian sci-kit learn
\item
Memahami cara penggunaan variabel explorer di spyder
\end{enumerate}
Tugas dengan cara dikumpulkan dengan pull request ke github dengan menggunakan latex pada repo yang dibuat oleh asisten riset.

\section{Teori}
Praktek teori penunjang yang dikerjakan :
\begin{enumerate}
\item
Buat Resume Definisi, Sejarah dan perkembangan Kecerdasan Buatan, dengan bahasa yang mudah dipahami dan dimengerti. Buatan sendiri bebas plagiat[hari ke 1](10)
\item
Buat Resume mengenai definisi supervised learning, klasifikasi, regresi dan unsupervised learning. Data set, training set dan testing set.[hari ke 1](10)
\end{enumerate}

\section{Instalasi}
Membuka https://scikit-learn.org/stable/tutorial/basic/tutorial.html. Dengan menggunakan bahasa yang mudah dimengerti dan bebas plagiat. 
Dan wajib skrinsut dari komputer sendiri.
\begin{enumerate}
\item
Instalasi library scikit dari anaconda, mencoba kompilasi dan uji coba ambil contoh kode dan lihat variabel explorer[hari ke 1](10)
\item
Mencoba Loading an example dataset, menjelaskan maksud dari tulisan tersebut dan mengartikan per baris[hari ke 1](10)
\item
Mencoba Learning and predicting, menjelaskan maksud dari tulisan tersebut dan mengartikan per baris[hari ke 2](10)
\item
mencoba Model persistence, menjelaskan maksud dari tulisan tersebut dan mengartikan per baris[hari ke 2](10)
\item 
Mencoba Conventions, menjelaskan maksud dari tulisan tersebut dan mengartikan per baris[hari ke 2](10)
\end{enumerate}


\section{Penanganan Error}
Dari percobaan yang dilakukan di atas, apabila mendapatkan error maka:

\begin{enumerate}
	\item
	skrinsut error[hari ke 2](10)
	\item
Tuliskan kode eror dan jenis errornya [hari ke 2](10)
	\item
Solusi pemecahan masalah error tersebut[hari ke 2](10)

\end{enumerate}

\section{Imron Sumadireja / 1164076}
\subsection{Teori}
\begin{enumerate}
\item
Pengertian
\subitem
Kecerdasan Buatan Artificial Intelligence merupakan salah satu bagian dari ilmu komputer yang mempelajari cara membuat mesin komputer dapat melakukan pekerjaan sebaik bahkan lebih baik dari yang dilakukan oleh manusia. Agar mesin dapat bekerja layaknya manusia maka perlu diberi bekal pengetahuan, sehingga mempunyai kemampuan untuk menalar. Menurut para ahli kecerdasan buatan seperti berikut:
\begin{itemize}
\item
H. A. Simon:
Kecerdasan buatan Artificial Intelligence merupakan kawasan penelitian, aplikasi dan instruksi yang terkait dengan pemrograman komputer untuk melakukan sesuatu hal yang dalam pandangan manusia adalah cerdas.
\item
Rich and Knight:
Kecerdasan buatan Artificial Intelligence merupakan sebuah studi tentang bagaimana membuat komputer melakukan hal-hal yang pada saat ini dapat dilakukan lebih baik oleh manusia.
\end{itemize}

\item
Sejarah dan Perkembangan
\subitem
Kata intelligence berasal dari bahasa latin intelligo yang memiliki arti saya paham. Arti dasar dari intelligence merupakan kemampuan untuk memahami dan melakukan aksi. Area Kecerdasan Buatan Artificial Intelligence, bermula pada saat kemunculan komputer sekitar tahun 1940-an, walaupun sejarah perkembangannya dapat dilacak sejak zaman Mesir kuno. Pada masa saat ini, perhatian difokuskan pada kemampuan komputer mengerjakan sesuatu yang dapat dilakukan oleh manusia. Dalam hal ini, komputer tersebut dapat meniru kemampuan kecerdasan dan perilaku manusia dengan akurasi yang cukup baik \cite{warwick2013artificial}.
\subitem
Pada akhir tahun 1955, Newell dan Simon mengembangkan The Logic Theorist, program AI pertama, program ini merepresentasikan masalah sebagai model pohon, lalu penyelesaiannya dengan memilih cabang yang akan menghasilkan kesimpulan yang paling benar. Pada tahun 1956 John McCarthy dari Massacuhetts Institute of Technology dianggap sebagai bapak AI, menyelenggarakan konferensi untuk menarik para ahli komputer bertemu, dengan nama kegiatan The Dartmouth Summer Research Project on Artificial Intelligence. Konferensi Dartmouth itu mempertemukkan para pendiri AI, dan bertugas untuk meletakkan dasar bagi masa depan pengembangan dan penelitian AI. John McCarthy saat itu mengusulkan definisi AI adalah AI merupakan cabang dari ilmu komputer yang berfokus pada pengembangan komputer untuk dapat memiliki kemampuan dan berprilaku seperti manusia\cite{bassil2012expert}.

\item
Supervised Learning dan Unsupervised Learning
\subitem
Supervised Learning merupakan suatu pendekatan dimana sudah terdapat data yang dilatih, dan terdapat variable yang ditargetkan sehingga tujuan dari pendekatan ini adalah mengelompokan suatu data ke data yang sudah ada. Sebagai contoh, ketika Anda memiliki sejumlah buku yang sudah dibeli dengan beberapa kategori. Misalnya, kategori buku akademik, dan buku novel. Selanjutnya Anda membeli sejumlah buku baru, maka Anda harus mengindentifikasi buku tersebut, dan memasukannya dalam kategori yang sudah ada.
\subitem
Unsupervised Learning merupakan suatu pendekatan namun tidak memiliki data yang dilatih, sehingga dari data yang ada, kita dapat mengelompokan data tersebut menjadi 2 bagian atau 3 bagian dan seterusnya. Sebagai contoh, Anda belum pernah membeli sejumlah buku, suatu hari Anda membeli sejumlah buku dan ingin membaginya kedalam beberapa kategori agar mudah dicari. Anda akan mengidentifikasi buku mana yang mirip. Dalam hal ini, kita memilih buku berdasarkan isinya.


\item
Klasifikasi dan Regresi
\subitem
Klasifikasi merupakan penempatan objek-objek ke salah satu dari beberapa kategori yang telah ditentukan sebelumnya. Klasifikasi banyak digunakan untuk memprediksi kelas pada suatu label atau atribut tertentu, yaitu dengan mengklasifikasi data membangun model berdasarkan training set dan nilai-nilai dalm mengklasifikasikan data yang baru.
Regresi dibedakan menjadi 2, diantaranya regresi linear dan regresi nonlinear.
\begin{itemize}
\item
Regresi Linear
Regresi Linear merupakan bentuk hubungan di mana variabel bebas x maupun variabel tergantung y sebagai faktor yang berpangkat satu.
\item
Regresi Nonlinear
Regresi Nonlinear merupakan bentuk hubungan atau fungsi di mana variabel x dan variabel tidak bebas y dapat berfungsi sebagai faktor atau variabel dengan pangkat tertentu.
\end{itemize}

\item
Data set, Training set, dan Testing set
\subitem
Untuk melakukan data set, training set, dan testing set diperlukan beberapa langkah, diantaranya:
\begin{itemize}
\item
Membuat model atau mesin untuk memeriksa data,
\item
Membuat model atau mesin belajar dari kesalahannya,
\item
Membuat kesimpulan tentang sebarapa baik kinerja model atau mesin tersebut.
\end{itemize}

\begin{enumerate}
\item
Data set
\subitem
Data set ini mencakup sekumpulan contoh input yang modelnya akan cocok atau dilatih dengan menyesuaikan parameter.
\item
Training set
\subitem
Training set diperlukan oleh model atau mesin agar dapat dilatih. Dengan menghitung kerugian tingkat kesalahan yang dilakukan model atau mesin menghasilkan pada set validasi pada titik tertentu, agar kita tahu seberapa akuratnya. Selanjutnya, model akan menyesuaikan parameternya berdasarkan hasil evaluasi yang sering pada training set ini.
\item
Testing set
\subitem
Testing set sangat penting untuk menguji generelasi model atau mesin. Dengan testing set ini, kita bisa mendapatkan akurasi kinerja model atau mesin.
\end{enumerate}

\end{enumerate}

\subsection{Instalasi}
\subsubsection{Proses Instalasi Anaconda dan Library Scikit}
\begin{enumerate}
\item Pertama kita unduh terlebih dahulu aplikasi anaconda, seperti gambar berikut
\begin{figure}[ht]
\centering
\includegraphics[scale=0.3]{figures/1.png}
\caption{Download Aplikasi Anaconda}
\end{figure}

\item Setelah di unduh, selanjutnya buka aplikasi tersebut. Lalu klik next untuk melanjutkan.
\begin{figure}[ht]
\centering
\includegraphics[scale=0.7]{figures/2.png}
\caption{Proses Instalasi Aplikasi}
\end{figure}

\item Lalu klik I Agree untuk melanjutkan.
\begin{figure}[ht]
\centering
\includegraphics[scale=0.7]{figures/3.png}
\caption{Proses Instalasi Aplikasi}
\end{figure}

\item Selanjutnya pilih Just me agar aplikasi tersebut hanya dapat digunakan oleh user yang login pada laptop tersebut.
\begin{figure}[ht]
\centering
\includegraphics[scale=0.7]{figures/4.png}
\caption{Proses Instalasi Aplikasi}
\end{figure}

\item Lalu tentukan direktori penyimpanan file tersebut
\begin{figure}[ht]
\centering
\includegraphics[scale=0.7]{figures/5.png}
\caption{Proses Instalasi Aplikasi}
\end{figure}

\item Selanjutnya akan muncul pop up box tentang advance installation options, ceklis keduanya.
\begin{figure}[ht]
\centering
\includegraphics[scale=0.7]{figures/6.png}
\caption{Proses Instalasi Aplikasi}
\end{figure}

\item Tunggu hingga proses install selesai
\begin{figure}[ht]
\centering
\includegraphics[scale=0.7]{figures/7.png}
\caption{Proses Instalasi Aplikasi}
\end{figure}

\item Setelah proses instalasi selesai, klik next
\begin{figure}[ht]
\centering
\includegraphics[scale=0.7]{figures/8.png}
\caption{Proses Instalasi Aplikasi}
\end{figure}

\item Pada bagian selanjutnya akan muncul box dengan memberikan pilihan untuk install VS Code,  jika tidak klik skip.
\begin{figure}[ht]
\centering
\includegraphics[scale=0.7]{figures/9.png}
\caption{Proses Instalasi Aplikasi}
\end{figure}

\item Setelah selesai, klik finish
\begin{figure}[ht]
\centering
\includegraphics[scale=0.7]{figures/10.png}
\caption{Proses Instalasi Aplikasi}
\end{figure}

\item Setelah proses instalasi selesai, selanjutnya buka cmd dan ketikan seperti berikut.
\begin{figure}[ht]
\centering
\includegraphics[scale=0.5]{figures/16.png}
\caption{Instalasi Library}
\end{figure}

\item Selanjutnya ketikan perintah berikut untuk mengunduh library scikit
\begin{figure}[ht]
\centering
\includegraphics[scale=0.5]{figures/17.png}
\caption{Instalasi Library}
\end{figure}

\item Jika sudah berhasil selanjutnya, ketikan perintah seperti gambar berikut untuk malakukan cek versi conda dan python
\begin{figure}[ht]
\centering
\includegraphics[scale=0.7]{figures/18.png}
\caption{Instalasi Library}
\end{figure}

\item Mencoba dan mengcompile source code, hasilnya seperti berikut
\begin{figure}[ht]
\centering
\includegraphics[scale=0.5]{figures/19.png}
\caption{Instalasi Library}
\end{figure}
\end{enumerate}

\subsection{Mencoba Loading Dataset}
\begin{enumerate}
\item Berikut source code yang menjelaskan tentang loading dataset. Pada baris pertama code tersebut berfungsi untuk import library datasets dari sklearn. Baris kedua berfungsi untuk menampilkan data secara berurutan. Baris ketiga untuk menampilkan data tersebut berupa angka dan baris keempat untuk menampilkan data tersebut.
\begin{figure}[ht]
\centering
\includegraphics[scale=0.5]{figures/20.png}
\caption{Loading dataset}
\end{figure}
\end{enumerate}


